%!TEX root = ../template.tex
%%%%%%%%%%%%%%%%%%%%%%%%%%%%%%%%%%%%%%%%%%%%%%%%%%%%%%%%%%%%%%%%%%%%
%% chapter4.tex
%% NOVA thesis document file
%%%%%%%%%%%%%%%%%%%%%%%%%%%%%%%%%%%%%%%%%%%%%%%%%%%%%%%%%%%%%%%%%%%%

\typeout{NT FILE chapter4.tex}%

\chapter{Design}
\label{cha:design}

\glsresetall

\section{Preliminary Protocol Concept}
\label{sec:preliminary_protocol_concept}

TODO

\section{Security and Governance Requirements}
\label{sec:security_governance_requirements}

Para garantir que o protocolo atenda às demandas de segurança e
governança horizontal, os seguintes requisitos são definidos:

\begin{itemize}
    \item \textbf{Transparência:} Todos os processos de autorização
devem ser registrados de forma imutável, permitindo auditorias
completas.
    \item \textbf{Participação Democrática:} A tomada de decisão deve
ser inclusiva, com processos que permitam contribuições de todos os
membros.
    \item \textbf{Flexibilidade:} O protocolo deve suportar mudanças
rápidas entre modos de governança centralizados e descentralizados,
dependendo das necessidades da organização.
    \item \textbf{Escalabilidade:} Deve ser capaz de operar
eficientemente em organizações de diferentes tamanhos e níveis de
complexidade.
\end{itemize}

\section{Evaluation Strategy}
\label{sec:evaluation_strategy}

A estratégia de avaliação será baseada em estudos de caso e simulações
que envolvam cenários realistas de organizações horizontais. O
desempenho do protocolo será analisado considerando os seguintes
critérios:

\begin{itemize}
    \item \textbf{Eficiência:} Tempo necessário para concluir
processos de autorização e tomadas de decisão.
    \item \textbf{Eficácia:} Capacidade de identificar e mitigar
ameaças relevantes.
    \item \textbf{Aceitação pelos Usuários:} Facilidade de uso e nível
de adoção pelos membros da organização.
    \item \textbf{Resiliência a Ameaças:} Desempenho sob ataques
simulados, incluindo manipulações de quórum e falhas de consenso.
\end{itemize}

\section{Experimental Design}
\label{sec:experimental_design}

TODO

\section{Research Questions}
\label{sec:research_questions}

\begin{enumerate}
    \item Como o protocolo pode equilibrar eficiência e participação
democrática em organizações horizontais?
    \item Quais são as melhores práticas para integrar segurança e
governança em estruturas descentralizadas?
    \item Como o protocolo pode se adaptar a diferentes níveis de
horizontalidade e dinâmicas organizacionais?
    \item De que forma ele pode melhorar a resiliência contra ameaças
internas e externas, mantendo a transparência e a participação
inclusiva?
\end{enumerate}

