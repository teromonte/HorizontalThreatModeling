%!TEX root = ../template.tex
%%%%%%%%%%%%%%%%%%%%%%%%%%%%%%%%%%%%%%%%%%%%%%%%%%%%%%%%%%%%%%%%%%%%
%% chapter4.tex
%% NOVA thesis document file
%%%%%%%%%%%%%%%%%%%%%%%%%%%%%%%%%%%%%%%%%%%%%%%%%%%%%%%%%%%%%%%%%%%%

\typeout{NT FILE chapter4.tex}%

\chapter{Solution}
\label{cha:solution}

\glsresetall
 
\section{Threat Modeling Protocol for Horizontal Organizations}
\label{sec:protocol}

Horizontal cooperatives face unique security challenges. Without a central
authority, they are vulnerable to attacks that exploit their open, democratic
nature. For example, an attacker might create fake member identities to sway
decisions (a Sybil attack) or abuse the consensus process to cause chaos. At the
same time, horizontality can be a security strength: distributing decisions
means there's no single point of failure. This protocol treats horizontality
as an asset, making security a collective endeavor rather than a top-down
mandate.

\subsection{Key Design Principles}
\label{subsec:key_principles}

\begin{itemize}
    \item \textbf{Transparency}: Security activities (like decisions,
configurations, and incidents) should be visible to members. Open logs and
auditable records ensure nothing happens "behind closed doors, building trust
and accountability among the group.
    \item \textbf{Decentralization}: No single person should have unchecked
power over systems or data. Access and control are distributed. This prevents a
"single admin from being a weak link and avoids creating a digital vanguard
(where a tech-savvy few hold all the keys).
    \item \textbf{Democratic Participation}: All members can participate in
identifying and addressing threats. Security decisions are made through
inclusive discussions or votes, so measures have collective buy-in. This keeps
the process aligned with the coop's democratic governance.
    \item \textbf{Traceability}: Every important action (granting access, making
a change, etc.) leaves an immutable trail. For example, changes can be logged on
tamper-proof ledgers and digitally signed by those who approved them. This way,
if something goes wrong, the coop can trace what happened and who was involved,
without relying on memory or hearsay.
    \item \textbf{Resilience}: The protocol aims to strengthen the coop's
ability to withstand and recover from threats. By eliminating single failure
points and planning for crises (with backup plans and rapid response
mechanisms), the organization stays resilient even under attack. If one
safeguard fails, others are in place to limit damage and bounce back quickly.
\end{itemize}

These principles ensure that improving security will not undermine the
organization nature of the group. Instead, security measures will reinforce
collaboration, shared responsibility, and trust. In practice, this means
building security into everyday cooperative workflows and governance. What
follows is a step-by-step threat modeling process designed with these values in
mind. It's written in accessible language so that any member can take part.
Each step includes guidance and checklists for participatory activities,
and the protocol can be scaled or adapted for organizations of different
sizes and structures.

(Note: While this protocol is inspired by established frameworks like PASTA,
it does not use the formal seven-stage PASTA terminology. Instead, it
presents an equivalent logic in a more accessible format.)

\subsection{Target Audience}
\label{subsec:target_audience}

The protocol is designed for members of horizontal organizations without
specialized cybersecurity expertise. Unlike traditional expert-oriented models,
our protocol emphasizes simplicity and accessibility. It supports stakeholders
involved in decision-making, operational activities, conflict resolution, and
coordination tasks, as well as informal community groups, by providing clear
guidance and intuitive methods for effectively dealing with security threats.


\subsection{Ethics and Protection of Organizations}
\label{subsec:ethics_protection}

When constructing and applying the threat modeling protocol in non-hierarchical
organizations, it is imperative to consider ethical principles as structuring
elements that transcend the merely technical aspect of cybersecurity. Ethical
concerns are based on the explicit commitment to protecting not only
technological integrity, but also the individuals and groups involved in
organizational processes.

A crucial aspect in this context is the responsibility regarding the
confidentiality of sensitive information and the protection of the privacy of
organizational members. The inadvertent exposure of security flaws can cause
significant damage, not only operational, but also personal and social.
Therefore, the protocol must incorporate strict guidelines on the ethical
treatment of identified vulnerabilities, ensuring that such information is
managed in a restricted manner and shared only with authorized individuals or
groups, always respecting the principle of least privilege.

Additionally, it is vital to establish clear internal communication mechanisms
to ensure that any identified vulnerabilities are immediately reported,
mitigated and documented without unnecessary public exposure.

\section{Threat Modeling Process Overview}
\label{sec:threat_modeling_process_overview}

The threat modeling process is broken into eight collaborative steps. In a small
cooperative, most steps can be taken in all-hands meetings or workshops with
everyone. In larger groups, you might delegate initial work to
working groups, but every member should have a chance to review and contribute
at each stage. The process is iterative and modular, so you can adjust the depth
or format of each step based on your organization's size and needs. For each
step below we describe the process and its activities
along with tips to adapt to different situations.

\subsection{Step 1: Establish Context and Goals}
\label{subsec:Step1}

What Are We Protecting?

\subsubsection{Purpose}

Set the stage by agreeing on what assets and operations you need to protect, and
what your security objectives are. This ensures everyone is on the same page
about why you are doing threat modeling and what "success looks like". 

\subsubsection{Activities}

\begin{itemize}
    \item \textbf{Identify Critical Assets}: In a group, list out what is most
valuable to your organization. This can include digital assets (member data,
documents, the website, chat platforms), physical assets (office space, devices,
servers), and intangible assets (the organization's reputation, member trust).
Ask yourselves what would hurt the most if it were stolen, destroyed, or made
public.
    \item \textbf{Outline Key Operations/Workflows}: Describe in simple terms
what the organization does everyday. For example, "We coordinate orders through an
online platform", or "We have weekly meetings to make decisions", or "We run a
community space with an entry badge system". Understanding these workflows helps
identify where disruptions would be most damaging.
    \item \textbf{State Security Objectives and Requirements}: Discuss what
security means for your organization. Do you need to keep member data private? Ensure
your service is always available? Meet any legal regulations like
GDPR? Also consider organization statutes or policies about confidentiality and data
handling. For instance, if your statutes say all financial info must be accessible
to members, that influences how you balance transparency with confidentiality.
    \item \textbf{Define the Scope and Boundaries}: Decide what will and won't
be covered in this threat modeling exercise. Maybe you want to focus on a
particular system and not on unrelated areas.
Or include only digital systems but not physical office
security or vice versa. Clearly defining scope prevents the discussion from
going off-track. It's okay to start with a narrow scope and expand later if needed.
    \item \textbf{Agree on Terminology}: Ensure everyone understands basic terms
you will use. For example, define what you mean by asset, threat, vulnerability,
etc., in plain language. A quick glossary on a whiteboard can help.
\end{itemize}

\subsection{Step 2: Map Systems and Trust Boundaries}
\label{subsec:Step2}

How Do We Work?

\subsubsection{Purpose}

Create a shared understanding of how information and processes flow in your
organization, and where important trust boundaries are. Essentially, draw a map of your
organization's sociotechnical system including people, tech, and their
interactions. In threat modeling, this is similar to diagramming your system
architecture and identifying entry points. For a cooperative, it also means
noting social trust assumptions like who or what we trust and in what ways.

\subsubsection{Activities}

List components and assets like hardware, Software, Data Stores,
People, Roles and Processes. Write these out, possibly in categories.
Essentially, you are enumerating what pieces make up your organization "system".

After that, diagram the Workflow: On a large paper or using a simple online diagram,
sketch how these components connect. Draw who interacts with what:
for example members (people) log into the chat platform (software) to discuss,
or the website communicates with a payment processor. Draw arrows for data flow
or interaction: emails sent, files shared, money transferred, etc. Keep it
understandable. Mark any external services clearly since those are partly
outside your control.

Identify Trust Boundaries, the points in the system where the
level of trust changes. For instance: between an external user and your
internal system like a public website and your internal database; between
a regular member and personal data; social trust boundaries for example
the trust members will not to leak info from private discussions.

On your diagram, draw a dotted line where these boundaries are.
Essentially ask: At what points do we assume things are safe on one side
and potentially risky on the other?

Document Who Has Access to What: Alongside the map, list which roles or people have
access to which assets. E.g., "Only tech team members can access the server", or "All
members can post in the forum", or "Treasurer has the bank account login". This
helps spotlight any concentrations of access and areas where trust is placed in
individuals. Write down services or partners you rely on and mark them on the
diagram for example, your website host, email service, or any software
provider. These are outside your organization but critical; threats can come through them.

\subsection{Step 3: Identify Threats Collaboratively}
\label{subsec:Step3}

What Could Go Wrong?

\subsubsection{Purpose}

Brainstorm all the potential threats and bad things that could happen to the
assets and processes you identified. The goal is a comprehensive list of threat
scenarios, covering both technical attacks and governance risks. At this
stage, quantity is more important than quality. We want to surface as many
ideas as possible, without judging them yet. This step harnesses the diverse
perspectives in your organization: digitally skilled members might think
of hacking scenarios, whereas others might point out process failures or insider
issues that a pure tech focus could miss.

\subsubsection{Activities}

\begin{itemize}

    \item \textbf{Brainstorm in a Safe Environment}:
    
    Gather a group of members, ideally representing different roles or viewpoints,
    for a threat brainstorming session. Set some ground rules: no idea is
    too small and everyone's input is valued. It's important
    people feel comfortable mentioning even unpleasant hypotheticals.
    Assure everyone this is about hypothetical situations,
    not personal distrust.

    \item \textbf{Use Prompts and Creative Tools}:
    Use prompts and creative techniques to explore potential threats by walking
    through the system map from Step 2 and pausing at each component or boundary to
    ask what could go wrong. For example, when examining the database, consider
    whether someone could steal or delete important information, and think about who
    might do this and how. At a boundary like the internet connection, reflect on
    the possibility of an attacker intercepting data in transit or flooding the
    system with excessive traffic. It can also be helpful to introduce the classic
    \gls{stride} as a simple checklist that prompts questions such as
    whether there is a risk of someone impersonating users,
    manipulating records, denying actions they actually performed, exposing private
    data, disrupting service, or somehow gaining privileged access.

    Another useful approach is to incorporate Security Cards, which
    encourage you to think broadly about different attackers
    motivations, methods, and the impacts they could have.
    Motivations might include financial gain, political objectives, disgruntled
    ex-members, or simple mischief. Methods and resources could range from phishing
    emails and malware to physical break-ins, social engineering, or legal threats.
    The potential impacts include service downtime, loss of member trust, financial
    losses, or legal ramifications. By posing these questions and exploring various
    scenarios, the group can avoid focusing solely on obvious IT threats and instead
    uncover possibilities that include internal misuse, human error, or even
    external events that could pose significant risks.

    \item \textbf{Distinguish Different Threat Sources}:
    As ideas emerge, note whether each threat
    scenario is external like a hacker, a virus, a competitor
    or internal coming from within the organization like a member error,
    an insider attack, conflict and miscoordination. There could be a hybrid threat
    where an external actor exploits an internal weakness, like a hacker using
    social engineering to trick a member into giving up access.

    \item \textbf{Write Down Concrete Scenarios}:
    For each idea, capture it as a short scenario description.
    For example:

    \begin{itemize}
        \item "Sybil Attack on decision-making:" An attacker creates multiple fake member
        accounts to gain extra votes in an online poll, influencing the group decision illicitly.
        \item "Insider data leak:" A discontented member with access to sensitive data decides to leak member
        emails and addresses to the public.
        \item "Ransomware on shared drive:" Malware infects a member's computer and encrypts the shared cloud
        drive files, making them inaccessible until a ransom is paid.
        \item "Lost credentials:" A member who manages the Twitter account leaves suddenly, and no one else has
        the password The group loses control of its own social media.
        \item "Miscoordination outage:" In a crisis, no one is designated to respond, because everyone thinks
        someone else will, and a small issue like a certificate expiry escalates, taking the website offline
        for days.
        \item "Service provider failure:" The third-party platform goes
        down or is compromised, affecting the organization's operations.
    \end{itemize}

    Aim for a broad list, covering cyber-attacks, human mistakes, physical events
    like the office gets robbed or a server gets wet, and governance failures. Don't
    worry at this stage if some scenarios seem very unlikely. List them as if someone
    is concerned about it.

    \item \textbf{Ensure Social/Process Threats are Included}:
    Cooperatives might face threats like quorum manipulation (exploiting the rules
    of consensus/voting), abuse of emergency powers (someone invoking a crisis to
    grab authority), or "digital vanguard" accumulation (one person quietly gaining
    control of many digital assets because no one else steps up). Include these in
    your brainstorming. For example, "Member X holds all the keys and if they quit
    or go rogue, we're locked out" is a valid threat scenario to record (it's an
    internal risk).

\end{itemize}

\subsection{Step 4: Profile Adversaries}
\label{subsec:Step4}

Who Might Attack Us?

\subsubsection{Purpose}

Humanize the threats by creating adversary personas that represent fictional
characters that represent types of attackers or sources of threats. This helps the
group to think from an attacker's perspective and ensures you consider the
motivations and capabilities behind the threats. In this step we focus on personas for
intentional actors. Developing these profiles makes later analysis more
concrete and relatable.

\subsubsection{Activities}

\begin{itemize}
    \item \textbf{Identify Key Threat Actors}: Look at the threat scenarios from Step 3 and ask, "Who would
    carry out these actions?". You will find a few recurring archetypes. For example:
    a hacker or vandal with no connection to the organization, motivated by profit or
    entertainment; a state or corporate actor who opposes the organization's
    mission; a disgruntled or former member with inside knowledge and intent to
    cause harm; or a careless insider who, even without bad intentions, may
    introduce risks through mistakes or negligence. It is also important to
    distinguish between adversaries with technical skills—such as a mid-level
    hacker—and those who act in a non-technical manner, such as an insider who
    manipulates rules or processes to his or her own advantage.

    \item \textbf{For each type of actor}, it is recommended to create a brief profile that
    includes their name and role, for example: "Mallory, the Malicious Insider" or "Ingrid,
    the Unaware member", as well as describing their motivations such as profit,
    revenge, or simply wanting to harm the organization and their capabilities or
    resources, from intrusion techniques to privileged insider knowledge. In
    addition, it is important to indicate what methods this actor might use for example:
    phishing, vulnerability exploitation, manipulation of legitimate credentials.
    and relate each persona to specific scenarios from the threat list, highlighting
    how their actions fit into the group dynamics.

    \item \textbf{When documenting personas}, it is recommended to dedicate a page or slide to each
    adversary, including an illustrative image to make it more memorable, avoiding
    photographs of real people, in order to avoid generating bias.
    The text should be clear and accessible, emphasizing that the purpose is
    purely internal, without the need for excessive formality. For example, when
    describing Mallory, the resentful ex-member, it would be useful to indicate her
    background (she left after a conflict, but still retains some credentials), her
    motivations (revenge, demonstrating security flaws), her skills (she knows the
    system well, although she is not an expert in hacking), the most likely actions
    (using leftover credentials, masquerading as another member, spreading
    disinformation), and the assets targeted (member directory, internal
    communication channels, decision-making platforms). The same pattern can be
    applied for each main type of adversary identified.
    
    \item \textbf{Include at Least One Insider Persona}: It may be uncomfortable but include a scenario of a malicious or
    careless insider. Cooperatives thrive on trust, yet history shows sometimes insiders can cause harm (intentionally
    or not). By creating, say, "Insider Irene" who is well meaning but prone to bypassing rules.
    Make it clear this is hypothetical to improve security for everyone.
    
    \item \textbf{Use Personas in Discussion}: Once you have personas, you can use them in future steps. For example, when
    thinking about mitigations, you might ask "Would this stop Mallory?" or "How would we detect Oscar's actions?"
    Personas help ground these discussions.
\end{itemize}

\subsection{Step 5: Analyze Attack Scenarios}
\label{subsec:Step5}

How Could Attacks Happen?

\subsubsection{Purpose}

Now take your threat list and personas, and dive deeper into how those threats
could play out step-by-step. This scenario analysis helps you understand the
sequence of events in an attack and where your weak points are. In practice,
this means building attack narratives or attack trees and possibly simulating
some scenarios in an exercise. This step turns abstract threats into
concrete stories, revealing exactly what vulnerabilities enable an attack and
how severe the consequences would be. It's a bridge from brainstorming to
action, by visualizing attacks, you prepare to figure out defenses.

\subsubsection{Activities}

\begin{itemize}   
    \item \textbf{Construct Attack Trees}: Pick a high-priority threat scenario (you will prioritize
    formally in the next step but start with one that seems obviously serious or emblematic).
    For example, "unauthorized access to the member list", an attack tree is
    constructed with that objective at the root. From there, possible paths are identified.
    For example, an external attacker exploits a vulnerability in the system, gains administrator privileges, and
    extracts the data; or a malicious insider uses their legitimate access to copy
    and disclose the information; or an attacker convinces someone to give up their
    password, logs in with the victim's account, and navigates to the data. The
    figure below illustrates these branches. At each stage, it highlights where
    defenses already exist, where they fail, and which points require attention. The
    level of detail should be sufficient to reveal relevant decisions and
    vulnerabilities, while maintaining the clarity of the analysis.

    \item \textbf{Conduct Tabletop Simulations}: For some scenarios, especially ones involving multiple people or
    processes, do a role-play tabletop exercise. Assemble a small group and narratively walk through the
    scenario: Assign someone to play the adversary (using one of your personas) and others to play
    defenders or just observers. Example: Simulate "Sybil attack during an online vote." Narrator says:
    "It's the day of a big proposal vote. Unknown to the group, Mallory has created three fake member identities over
    the last month." Then step by step: "Vote opens. Mallory votes as herself and as Alice, Bob, and
    Charlie (her fake profiles). The system tallies four votes from what appears to be four people."
    Discuss: Would anything at that moment flag an issue? How would the organization notice? Perhaps another
    member finds it odd that there were three new members who never spoke but voted. Or maybe nobody
    notices until later. Continue: "The vote passes with those extra votes. Later, someone questions the
    outcome…" This kind of storytelling helps highlight if your current processes have detection or not.
    Participants can chime in with "At that point, I would check the member list…" or ask "Do we verify
    new online voters somehow?" Write down these insights. The idea is to practice an attack in theory
    to see where your response or system breaks. It's much cheaper to find gaps this way than during
    a real incident.
    
    \item \textbf{Identify Vulnerabilities at Each Step}: As you chart out these scenarios, explicitly list the
    vulnerabilities or weak points that make the attack possible. These could be technical (e.g.
    "Outdated plugin allows injection", "No backup exists for database"), or organizational (e.g. "New
    members are not verified, allowing fakes", "Only one person knows how to reset the server"). Also
    note any existing controls and whether they work: For each step in the scenario, ask "What should
    stop this? Do we have something to stop it? Does it actually stop it or can it be bypassed?"
    E.g., in the Sybil scenario: Vulnerability = lack of strict member verification in the
    voting system. Existing control = new accounts require admin approval (does that happen? maybe
    someone auto-approved without checks).
    In the hacker scenario: Vulnerability = software not patched; Control = we have a
    firewall, but if the attack comes through a web port, firewall doesn't stop it; or Control = we rely
    on strong passwords, which might not help if exploit exists.
    Write these vulnerabilities next to the steps or in a separate list mapped to the
    scenario. This will directly feed into deciding mitigations.

    \item \textbf{Assess Impact and Likelihood for Scenarios}: As part of analysis, discuss for each scenario,
    how bad would it be if this happened? and how likely is it to happen? Use qualitative terms for example:
    For ransomware encrypting files the Impact is High because of data loss but likelihood maybe Medium
    if members are generally careful. For sybil attack the impact is High on governance
    legitimacy but likelihood may be Low to Medium depending on how easy it is to create fake accounts in your
    system.

    \item \textbf{Leverage Past Incidents}: If your coop or similar groups have experienced incidents,
    incorporate those into scenarios. "This happened before, could it happen again in a worse way?"
    Learning from near-misses or history makes scenarios very concrete. For example, "Last year someone
    guessed our Twitter password, what if they had tweeted offensive stuff? Let's play out that
    scenario".
\end{itemize}

\subsection{Step 6: Prioritize Risks Together}
\label{subsec:Step6}

Which Problems Matter Most?

\subsubsection{Purpose}

Not all threats are equal. In this step, the group evaluates all the
identified threat scenarios and decides which ones to address first. This is
essentially a risk analysis and ranking. Risk is usually judged by two factors:
how severe the impact would be and how likely the threat is to occur. By scoring
or discussing these, the group can focus on the most critical issues.
Importantly, this is done participatorily, everyone's perspective on what is
important is considered, keeping the process democratic. The output will be a
clear list of top-priority risks that the organization will invest effort in mitigating.

\subsubsection{Activities}

\begin{itemize}

    \item \textbf{Define Risk Criteria (Impact and Likelihood)}: First, agree on clear
    definitions for impact and likelihood. High impact risks threaten the
    organization's survival, break laws, or deeply harm member trust; medium impacts
    create noticeable but manageable problems; low impacts result in minor
    inconveniences. High likelihood means there's strong evidence a risk can easily
    happen, medium is conditional, and low likelihood means it's rare or requires
    sophisticated effort. Use simple terms (high, medium, low), but ensure everyone
    understands them clearly.
    
    \item \textbf{Evaluate Each Threat Scenario}: Review your scenarios (from Step 3/5) one by one,
    asking two questions: "If this happens, how bad is it? (Impact)" and "How likely
    is it? (Likelihood)." Different perspectives matter,for example, tech members
    might know a threat is difficult to execute (low likelihood), while governance
    experts could emphasize the severe impact on trust. Aim for informal consensus
    or use a voting method to agree on each scenario's risk level.
    
    \item \textbf{Rank the Risks}: Identify the most critical risks needing immediate
    action, usually those with high impact, high likelihood, or both. Pay attention
    also to "easy fixes", issues that might be medium risk but simple to address
    immediately. Classify risks into categories like Critical (fix urgently),
    Moderate (plan action), and Low (acknowledge without immediate action).
    
    \item \textbf{Document Your Reasoning}: Clearly note why each top risk was
    prioritized. For example: "Sybil attack on voting, High Impact (could damage
    governance legitimacy), Low Likelihood (approval required for new members), still
    prioritized due to potential impact on trust". This transparency helps others
    understand the choices later.
    
    \item \textbf{Check Against Goals}: Revisit Step 1's initial assets and objectives.
    Ensure the prioritized risks align with your core mission and values. If
    important assets lack high-ranked threats, double-check if something was
    overlooked or truly low risk. Likewise, ensure you're not overly prioritizing
    threats unrelated to your central goals.
    
    \item \textbf{Get Group Approval}: Finally, confirm group agreement on your chosen
    top threats. For smaller groups, informal agreement may suffice; larger groups
    might prefer a quick formal vote. This step reinforces collective responsibility
    and accountability for the chosen priorities.

\end{itemize}
    
\subsection{Step 7: Mitigations and Governance Decisions}
\label{subsec:Step7}

How Do We Fix or Prevent Issues?

\subsubsection{Purpose}

For each of the top-priority threats, decide on countermeasures and integrate
those decisions into the organization's action plan. In other words, figure out what
security measures to implement and how to approve and enforce them
democratically. This step is where you turn analysis into concrete changes:
technical fixes, new or improved policies, training, etc. It's also where you
make sure that implementing these fixes doesn't accidentally centralize power or
violate cooperative principles.

\subsubsection{Activities}

\begin{itemize}

    \item \textbf{Brainstorm Mitigation Options:} For each high-priority threat,
    list potential mitigations. These might include technical solutions (software
    updates, two-factor authentication, encryption, multi-signature financial
    controls), workflow improvements (clear onboarding/offboarding processes,
    regular backups, peer reviews for critical tasks), training and education
    (phishing awareness, password management sessions, annual security policy
    orientations), and governance policies (formal rules for password storage,
    emergency decision protocols, member identity verification).
    
    \item \textbf{Evaluate Feasibility:} Consider the practicality of each
    mitigation. Discuss how difficult or costly they are, if external resources are
    required, if they introduce significant workflow friction, and if they align
    with organizational values. Use adversary personas to check effectiveness—ask if
    specific measures effectively deter identified threats.
    
    \item \textbf{Deliberate and Decide Collectively:} Engage members or relevant
    subgroups to choose appropriate mitigations. Include those responsible for
    implementation and those affected by changes. Address concerns through
    discussion, propose alternative solutions, and aim for consensus. Document
    clearly why each chosen mitigation was prioritized.
    
    \item \textbf{Assign Responsibilities Clearly:} Define clear responsibilities
    for each mitigation task, distributing them among relevant members or working
    groups. Technical tasks should include tech-skilled members paired with others
    for transparency. Assign policy drafting and educational roles to suitable teams
    or individuals. Clearly state budget needs for specific tasks to ensure
    alignment with cooperative budgeting processes.
    
    \item \textbf{Implement with Democratic Oversight:} Keep the organization
    updated throughout the implementation process. Regularly report progress in
    meetings or group communications. Drafted policies should circulate for
    collective review and formal adoption. Ensure full cooperation from members,
    emphasizing that security measures result from collective decisions.
    
    \item \textbf{Prepare for Emergencies:} Establish clear protocols for crisis
    situations requiring rapid response. Formally designate a small emergency
    response team authorized to take quick decisions within agreed limits. Define
    the scope and timeframe for their actions, and ensure accountability by
    mandating transparent post-incident reporting and collective decision-making on
    permanent actions.
    
    \item \textbf{Integrate into Governance Documents:} Codify agreed-upon
    mitigations and emergency protocols into the organization's official
    documentation. Update handbooks or wikis to institutionalize these practices,
    ensuring new members receive proper orientation.
    
    \item \textbf{Set Review Milestones:} Schedule periodic reviews to assess the
    effectiveness and implementation status of security measures. Quick check-ins at
    regular meetings, formal reviews after a set period, or integration into annual
    reviews help maintain accountability and adapt strategies as needed.
    
\end{itemize}

\subsection{Step 8: Ongoing Improvement and Monitoring}
\label{subsec:Step8}

How Do we Keep Alive Security?

\subsubsection{Purpose}

Threat modeling isn't a one-time project – especially in a cooperative that
evolves. Step 8 is about making security a continuous, integrated part of your
coop's governance and operations. This means regularly monitoring for new
threats, reviewing the effectiveness of your defenses, and adjusting to changes,
all without losing the cooperative spirit. In short: build a living security
program that grows and adapts with your organization.

\subsubsection{Activities}

\begin{itemize}

    \item \textbf{Include Security in Routine Governance:} Regularly discuss
    security as part of routine meetings. Monthly or weekly, briefly review
    incidents, vulnerabilities, or new tools. Conduct deeper quarterly or
    semi-annual reviews to update your threat model. Rotate or delegate preparation
    tasks to maintain collective involvement.
    
    \item \textbf{Maintain Logs and Audit Trails:} Regularly inspect logs and
    records to ensure adherence to agreed practices. Monthly reviews of
    administrative logs and periodic verification of multi-signature requirements or
    account security settings help maintain compliance. Conduct an annual peer audit
    of critical policies, openly sharing results to maintain transparency and
    accountability.
    
    \item \textbf{Periodic Retraining and Onboarding:} Create and regularly update a
    concise security handbook covering key practices and their rationale. Provide
    annual refresher workshops and simulate occasional phishing drills, discussing
    outcomes constructively. Incorporate security orientation into new member
    onboarding processes to instill a collective security culture from the start.
    
    \item \textbf{Adapt to Organizational Changes:} Regularly reassess your threat
    model to accommodate organizational growth or new projects. Expanded membership
    or new activities can introduce different risks. Schedule comprehensive threat
    modeling updates annually or biennially, adjusting assumptions to align with
    current realities.
    
    \item \textbf{Engage with the Wider Community:} Share non-sensitive security
    strategies with other cooperatives or participate in community forums. This
    interaction helps identify common threats, discover new tools, and contribute to
    collective knowledge. Bring insights from external interactions back to enhance
    your internal threat modeling sessions.
    
    \item \textbf{Monitor and Respond to Evolving Threats:} Stay updated on
    cybersecurity news relevant to your cooperative. Establish clear channels for
    collective information sharing, such as dedicated group chats or email lists,
    ensuring timely alerts and responses to emerging threats.
    
    \item \textbf{Governance Feedback Loop:} Regularly evaluate your security
    governance process for effectiveness and inclusivity. After incidents or
    periodically, review emergency response efficiency and member satisfaction with
    security protocols. Adjust policies through collective decision-making to
    balance security needs with practical usability.
    
\end{itemize}

\subsection{Participation Tips}
\label{subsec:participation_tips}

Successful threat modeling requires that everyone in the organization feels comfortable
and motivated to contribute. Creating a culture of collaboration and respect is
crucial. In small groups, completing activities in a single meeting is
effective, fostering unity and avoiding repeat sessions. For larger groups, use
brief questionnaires or small group discussions, consolidating ideas later in
plenary to ensure that everyone can contribute without feeling judged.

It is important that members with technical expertise do not dominate the
conversation. Encourage more reserved participants by asking them directly for
their opinions on specific threats or solutions. When prioritizing risks, open
discussions are valuable, but if consensus is difficult, consider anonymous
voting or point scoring to ease the pressure.

Keep a relaxed and creative environment when role-playing scenarios or
developing adversarial personas. Make it clear that these scenarios are
hypothetical and not accusations against anyone. Document decisions
transparently, explaining the reasons clearly, allowing for future review and
improvements.

Once security measures are in place, encourage open feedback on any challenges
or improvements needed. Share responsibilities equally among team members to
avoid overload, foster collective ownership, and maintain a genuinely flat
organizational structure.

\section{Security and Governance Requirements}
\label{sec:security_governance_requirements}

The proposed protocol for horizontal organizations must meet security and
governance requirements that ensure both participatory inclusion and robustness
against internal and external threats. In decentralized structures, where the
absence of formal hierarchy can be seen as both an advantage and a challenge, it
is imperative that the protocol be designed to integrate mechanisms that
preserve horizontality without compromising organizational resilience.
Transparency, in this context, emerges as one of the central pillars
\cite{Colbac}. All events related to the authorization of actions or
modifications must be recorded in an immutable manner, ensuring that actions can
be reliably audited by any member of the organization. These immutable records,
built on technologies such as blockchain or similar cryptographic structures,
ensure traceability and eliminate the possibility of unauthorized changes,
maintaining cohesion between organizational principles and technological
mechanisms.

Democratic participation also plays a fundamental role in the design of the
protocol \cite{Colbac}. In a horizontal organization, decisions must reflect the
collective will, and to this end, the protocol must integrate digital voting
systems that guarantee both the privacy and security of votes \cite{Colbac}. In
addition, it must be possible to temporarily delegate authority to individuals
or groups to deal with situations that require specific expertise, always
ensuring that such delegation can be revoked and that records of actions are
accessible for collective audit \cite{Colbac}.

Another crucial aspect is the flexibility of the protocol, especially in
scenarios where the organization needs to switch between centralized and
distributed governance modes. This transition capacity must be implemented in a
way that the levels of centralization are temporary and properly tracked,
ensuring that control can quickly return to the collective. To this end, the
protocol must provide mechanisms such as emergency tokens, which allow the
execution of critical actions in exceptional situations, as long as such actions
are recorded and subject to retroactive validation by the members of the
organization \cite{Colbac}.

For the protocol to be scalable, it is essential that it supports organizations
of different sizes and degrees of complexity. This includes implementing
distributed validation systems that allow collaborative verification of actions
without burdening individuals or single points of control \cite{Colbac}.
Furthermore, threat modeling should be iterative and adaptive, incorporating
techniques such as simulations based on real-world scenarios to identify
emerging vulnerabilities and adjust countermeasures accordingly. These
mechanisms ensure that the protocol remains functional and effective even when
the organizational scale increases or its internal dynamics change.

\section{Evaluation Strategy}
\label{sec:evaluation_strategy}

The evaluation of the proposed protocol will be conducted through an
experimental and comparative approach, involving real case studies in
organizations with different degrees of horizontality. The main objective is to
validate the effectiveness of the protocol in identifying, mitigating and
preventing threats in horizontal organizational contexts, directly comparing its
performance with established frameworks, such as \gls{stride}.

Candidate organizations will be selected to represent a variety of horizontal
structures, ensuring diversity in power distribution, size and operational
complexity. Participants from each organization will receive structured training
sessions covering both the proposed protocol and \gls{stride}, ensuring
familiarity and a level playing field for comparison.

\textbf{Parallel Threat Modeling Sessions:}
Each organization will conduct simultaneous threat modeling sessions using both
the proposed protocol and \gls{stride} in identical scenarios. These sessions
will be observed to systematically document the process and collect data,
analyzed according to the following clearly defined metrics:

\begin{itemize}
\item \textbf{Precision}: The ratio of valid threats correctly identified to the
total number of threats identified by the protocol.
\item \textbf{Coverage (Recall)}: The proportion of existing threats identified
by the protocol relative to a reference set established by experts.
\item \textbf{Operational Efficiency}: Total time (latency) required to complete
each phase of the threat modeling process, including identification, analysis,
and mitigation planning.
\item \textbf{Usability}: Participants' opinions on the ease of use, clarity,
and overall effort required to learn and effectively apply the protocol.
\end{itemize}

By systematically applying these clearly defined metrics and evaluation steps,
this strategy ensures a comprehensive comparative analysis, providing robust
empirical evidence on the effectiveness of the protocol relative to \gls{stride}
in horizontal organizations.

\section*{}
Chapter 4 presented the preliminary design of a protocol specifically aimed at
threat modeling in horizontal organizations, detailing fundamental security and
governance requirements, such as mechanisms for democratic participation,
temporary delegation of authority, and auditable records. The importance of
aligning technological and organizational processes to ensure robustness against
internal and external threats was highlighted. Based on this conceptual
development, Chapter 5 summarizes the main results achieved by this research,
critically evaluating the proposed protocol, identifying its contributions in
relation to traditional methodologies, and exploring practical scenarios for
future validations in real environments.
