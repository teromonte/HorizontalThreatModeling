%!TEX root = ../template.tex
%%%%%%%%%%%%%%%%%%%%%%%%%%%%%%%%%%%%%%%%%%%%%%%%%%%%%%%%%%%%%%%%%%%%
%% chapter4.tex
%% NOVA thesis document file
%%%%%%%%%%%%%%%%%%%%%%%%%%%%%%%%%%%%%%%%%%%%%%%%%%%%%%%%%%%%%%%%%%%%

\typeout{NT FILE chapter4.tex}%

\chapter{Design}
\label{cha:design}

\glsresetall
 
\section{Conceito Preliminar do Protocolo}
\label{sec:preliminary_protocol_concept}

O protocolo a ser desenvolvido será um modelo estruturado para a modelagem de
ameaças em organizações não-hierárquicas, funcionando de forma semelhante ao
STRIDE, mas adaptado às especificidades da governança distribuída
\cite{ThreatModelingdesigningForSecurity}.
Ele não será um software, mas um conjunto formal de diretrizes e metodologias documentadas
para identificação, análise e mitigação de ameaças em ambientes horizontais \cite{Colbac}.
Seu formato será textual e estruturado como um documento acadêmico e técnico,
oferecendo um referencial prático e teórico para segurança em estruturas
descentralizadas .

A audiência primária inclui membros da organização que utilizam o protocolo para
analisar ameaças em seu contexto específico, bem como especialistas em segurança
da informação e governança distribuída que buscam métodos adaptados a ambientes
sem hierarquia. O protocolo servirá para aprimorar a resiliência organizacional
sem comprometer a participação democrática, garantindo que a segurança seja
integrada à dinâmica colaborativa das organizações horizontais.

\section{Requisitos de Segurança e Governança}
\label{sec:security_governance_requirements}

O protocolo proposto para organizações horizontais deve atender a requisitos de
segurança e governança que garantam tanto a inclusão participativa quanto a
robustez contra ameaças internas e externas. Em estruturas descentralizadas,
onde a ausência de hierarquia formal pode ser vista tanto como uma vantagem
quanto como um desafio, é imperativo que o protocolo seja projetado para
integrar mecanismos que preservem a horizontalidade sem comprometer a
resiliência organizacional. A transparência, nesse contexto, emerge como um dos
pilares centrais \cite{Colbac}. Todos os eventos relacionados à autorização de ações ou
modificações devem ser registrados de maneira imutável, garantindo que as ações
possam ser auditadas de forma confiável por qualquer membro da organização.
Esses registros imutáveis, construídos sobre tecnologias como blockchain ou
estruturas criptográficas semelhantes, asseguram a rastreabilidade e eliminam a
possibilidade de alterações não autorizadas, mantendo a coesão entre os
princípios organizacionais e os mecanismos tecnológicos.

A participação democrática também desempenha um papel fundamental no desenho do
protocolo \cite{Colbac}. Em uma organização horizontal, as decisões devem refletir a vontade
coletiva, e, para isso, o protocolo deve integrar sistemas de voto digital que
garantam tanto a privacidade quanto a segurança dos votos \cite{Colbac}. Além disso, deve ser
possível delegar temporariamente a autoridade a indivíduos ou grupos para lidar
com situações que requeiram expertise específica, sempre assegurando que tal
delegação possa ser revogada e que os registros das ações estejam acessíveis
para auditoria coletiva \cite{Colbac}.

Outro aspecto crucial é a flexibilidade do protocolo, especialmente em cenários
onde a organização precise alternar entre modos de governança centralizada e
distribuída. Essa capacidade de transição deve ser implementada de forma que os
níveis de centralização sejam temporários e devidamente rastreados, garantindo
que o controle possa retornar rapidamente ao coletivo. Para isso, o protocolo
deve prever mecanismos como tokens de emergência, que permitam a execução de
ações críticas em situações excepcionais, desde que tais ações sejam registradas
e sujeitas a validação retroativa pelos membros da organização \cite{Colbac}.

Para que o protocolo seja escalável, é essencial que ele suporte organizações de
diferentes tamanhos e graus de complexidade. Isso inclui a implementação de
sistemas de validação distribuída que permitam a verificação colaborativa das
ações sem sobrecarregar indivíduos ou pontos únicos de controle \cite{Colbac}. Além disso, a
modelagem de ameaças deve ser iterativa e adaptável, incorporando técnicas como
simulações baseadas em cenários reais para identificar vulnerabilidades
emergentes e ajustar as contramedidas de acordo. Esses mecanismos
asseguram que o protocolo permaneça funcional e eficaz mesmo diante
de ampliação da escala organizacional ou de alterações em suas dinâmicas
internas.

\section{Estratégia de Avaliação}
\label{sec:evaluation_strategy}

A avaliação do protocolo proposto será conduzida de forma experimental e
comparativa, utilizando grupos organizacionais com diferentes graus de
horizontalidade. Este processo tem como objetivo principal validar a capacidade
do protocolo em identificar, mitigar e prevenir ameaças em ambientes
horizontais, enquanto compara sua eficácia com frameworks consolidados, como o
\gls{stride}.

Inicialmente, serão selecionadas organizações candidatas representativas de
distintos níveis de distribuição de poder e autonomia. Cada organização será
submetida a sessões de treinamento estruturadas, abordando tanto o protocolo
desenvolvido quanto o \gls{stride}, garantindo que todos os participantes compreendam
as ferramentas utilizadas.

Para operacionalizar a avaliação, será aplicada uma metodologia em três etapas.
Na primeira etapa, cada grupo organizará um modelo de ameaças para sua estrutura
com base no protocolo proposto. Em paralelo, um segundo grupo, dentro da mesma
organização, aplicará o \gls{stride} ao mesmo cenário. Ambas as sessões
serão acompanhadas para documentar o processo e coletar dados
quantitativos e qualitativos.

Os modelos resultantes serão analisados segundo métricas predefinidas, incluindo:

\begin{itemize}

    \item \textbf{Precisão}: avaliação do número de ameaças corretamente identificadas
    em relação ao total detectado.

    \item \textbf{Recall}: proporção de ameaças identificadas pelo protocolo em relação
    às existentes no modelo de referência.

    \item \textbf{Latência Operacional}: tempo total necessário para completar cada fase
    do processo de modelagem.

    \item \textbf{Feedback dos Usuários}: coleta de opiniões qualitativas sobre
    usabilidade e clareza do protocolo por parte dos participantes.

\end{itemize}

A segunda etapa envolverá a simulação de cenários de ameaças, incluindo ataques
Sybil, manipulação de quórum e falhas de consenso, replicando situações
realistas enfrentadas por organizações horizontais. A resposta a essas ameaças
será comparada entre os protocolos, considerando a qualidade das soluções
propostas e o tempo de resposta dos grupos.

Por fim, na terceira etapa, as organizações serão submetidas a estudos de caso,
nos quais o protocolo será implementado de maneira integral e monitorado ao
longo do tempo. O desempenho do protocolo será analisado em termos de
resiliência às ameaças identificadas, sua adaptabilidade a dinâmicas
organizacionais e a capacidade de promover a participação e transparência.

\section{Questões de Pesquisa}
\label{sec:research_questions}

\begin{enumerate}
    \item Como o protocolo pode equilibrar eficiência e participação
democrática em organizações horizontais?
    \item Quais são as melhores práticas para integrar segurança e
governança em estruturas descentralizadas?
    \item Como o protocolo pode se adaptar a diferentes níveis de
horizontalidade e dinâmicas organizacionais?
    \item De que forma ele pode melhorar a resiliência contra ameaças
internas e externas, mantendo a transparência e a participação
inclusiva?
\end{enumerate}

