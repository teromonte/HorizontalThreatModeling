%!TEX root = ../template.tex
%%%%%%%%%%%%%%%%%%%%%%%%%%%%%%%%%%%%%%%%%%%%%%%%%%%%%%%%%%%%%%%%%%%%
%% chapter4.tex
%% NOVA thesis document file
%%%%%%%%%%%%%%%%%%%%%%%%%%%%%%%%%%%%%%%%%%%%%%%%%%%%%%%%%%%%%%%%%%%%

\typeout{NT FILE chapter4.tex}%

\chapter{Design}
\label{cha:design}

\glsresetall
 
\section{Protocol Concept}
\label{sec:protocol_concept}

The protocol to be developed will be a structured model for threat modeling in
non-hierarchical organizations, functioning similarly to STRIDE, but adapted to
the specificities of distributed governance
\cite{ThreatModelingdesigningForSecurity}. It will not be a software, but a
formal set of documented guidelines and methodologies for identifying, analyzing
and mitigating threats in horizontal environments \cite{Colbac}. Its format will
be textual and structured as an academic and technical document, offering a
practical and theoretical framework for security in decentralized structures.

The primary audience includes members of the organization who use the protocol
to analyze threats in their specific context, as well as experts in information
security and distributed governance who seek methods adapted to non-hierarchical
environments. The protocol will serve to improve organizational resilience
without compromising democratic participation, ensuring that security is
integrated into the collaborative dynamics of horizontal organizations.

\subsection{Target Audience}
\label{subsec:target_audience}

The protocol is designed for members of horizontal organizations without
specialized cybersecurity expertise. Unlike traditional expert-oriented models,
our protocol emphasizes simplicity and accessibility. It supports stakeholders
involved in decision-making, operational activities, conflict resolution, and
coordination tasks, as well as informal community groups, by providing clear
guidance and intuitive methods for effectively dealing with security threats.

\section{Security and Governance Requirements}
\label{sec:security_governance_requirements}

The proposed protocol for horizontal organizations must meet security and
governance requirements that ensure both participatory inclusion and robustness
against internal and external threats. In decentralized structures, where the
absence of formal hierarchy can be seen as both an advantage and a challenge, it
is imperative that the protocol be designed to integrate mechanisms that
preserve horizontality without compromising organizational resilience.
Transparency, in this context, emerges as one of the central pillars
\cite{Colbac}. All events related to the authorization of actions or
modifications must be recorded in an immutable manner, ensuring that actions can
be reliably audited by any member of the organization. These immutable records,
built on technologies such as blockchain or similar cryptographic structures,
ensure traceability and eliminate the possibility of unauthorized changes,
maintaining cohesion between organizational principles and technological
mechanisms.

Democratic participation also plays a fundamental role in the design of the
protocol \cite{Colbac}. In a horizontal organization, decisions must reflect the
collective will, and to this end, the protocol must integrate digital voting
systems that guarantee both the privacy and security of votes \cite{Colbac}. In
addition, it must be possible to temporarily delegate authority to individuals
or groups to deal with situations that require specific expertise, always
ensuring that such delegation can be revoked and that records of actions are
accessible for collective audit \cite{Colbac}.

Another crucial aspect is the flexibility of the protocol, especially in
scenarios where the organization needs to switch between centralized and
distributed governance modes. This transition capacity must be implemented in a
way that the levels of centralization are temporary and properly tracked,
ensuring that control can quickly return to the collective. To this end, the
protocol must provide mechanisms such as emergency tokens, which allow the
execution of critical actions in exceptional situations, as long as such actions
are recorded and subject to retroactive validation by the members of the
organization \cite{Colbac}.

For the protocol to be scalable, it is essential that it supports organizations
of different sizes and degrees of complexity. This includes implementing
distributed validation systems that allow collaborative verification of actions
without burdening individuals or single points of control \cite{Colbac}.
Furthermore, threat modeling should be iterative and adaptive, incorporating
techniques such as simulations based on real-world scenarios to identify
emerging vulnerabilities and adjust countermeasures accordingly. These
mechanisms ensure that the protocol remains functional and effective even when
the organizational scale increases or its internal dynamics change.

\section{Evaluation Strategy}
\label{sec:evaluation_strategy}

The evaluation of the proposed protocol will be conducted through an
experimental and comparative approach, involving real case studies in
organizations with different degrees of horizontality. The main objective is to
validate the effectiveness of the protocol in identifying, mitigating and
preventing threats in horizontal organizational contexts, directly comparing its
performance with established frameworks, such as \gls{stride}.

Candidate organizations will be selected to represent a variety of horizontal
structures, ensuring diversity in power distribution, size and operational
complexity. Participants from each organization will receive structured training
sessions covering both the proposed protocol and \gls{stride}, ensuring
familiarity and a level playing field for comparison.

\textbf{Parallel Threat Modeling Sessions:}
Each organization will conduct simultaneous threat modeling sessions using both
the proposed protocol and \gls{stride} in identical scenarios. These sessions
will be observed to systematically document the process and collect data,
analyzed according to the following clearly defined metrics:

\begin{itemize}
\item \textbf{Precision}: The ratio of valid threats correctly identified to the
total number of threats identified by the protocol.
\item \textbf{Coverage (Recall)}: The proportion of existing threats identified
by the protocol relative to a reference set established by experts.
\item \textbf{Operational Efficiency}: Total time (latency) required to complete
each phase of the threat modeling process, including identification, analysis,
and mitigation planning.
\item \textbf{Usability}: Participants' opinions on the ease of use, clarity,
and overall effort required to learn and effectively apply the protocol.
\end{itemize}

By systematically applying these clearly defined metrics and evaluation steps,
this strategy ensures a comprehensive comparative analysis, providing robust
empirical evidence on the effectiveness of the protocol relative to \gls{stride}
in horizontal organizations.

\section{Ethics and Protection of Organizations}
\label{sec:ethics_protection}

When constructing and applying the threat modeling protocol in non-hierarchical
organizations, it is imperative to consider ethical principles as structuring
elements that transcend the merely technical aspect of cybersecurity. Ethical
concerns are based on the explicit commitment to protecting not only
technological integrity, but also the individuals and groups involved in
organizational processes.

A crucial aspect in this context is the responsibility regarding the
confidentiality of sensitive information and the protection of the privacy of
organizational members. The inadvertent exposure of security flaws can cause
significant damage, not only operational, but also personal and social.
Therefore, the protocol must incorporate strict guidelines on the ethical
treatment of identified vulnerabilities, ensuring that such information is
managed in a restricted manner and shared only with authorized individuals or
groups, always respecting the principle of least privilege.

Additionally, it is vital to establish clear internal communication mechanisms
to ensure that any identified vulnerabilities are immediately reported,
mitigated and documented without unnecessary public exposure.

\section{Research Questions}
\label{sec:research_questions}

\begin{enumerate}
    \item How can the protocol balance efficiency and democratic participation
in horizontal organizations?
    \item What are the best practices for integrating security and governance
in decentralized structures?
    \item How can the protocol adapt to different levels of horizontality and
organizational dynamics?
    \item How can it improve resilience against internal and external threats
while maintaining transparency and inclusive participation?
\end{enumerate}

\section*{}
Chapter 4 presented the preliminary design of a protocol specifically aimed at
threat modeling in horizontal organizations, detailing fundamental security and
governance requirements, such as mechanisms for democratic participation,
temporary delegation of authority, and auditable records. The importance of
aligning technological and organizational processes to ensure robustness against
internal and external threats was highlighted. Based on this conceptual
development, Chapter 5 summarizes the main results achieved by this research,
critically evaluating the proposed protocol, identifying its contributions in
relation to traditional methodologies, and exploring practical scenarios for
future validations in real environments.
