%!TEX root = ../template.tex
%%%%%%%%%%%%%%%%%%%%%%%%%%%%%%%%%%%%%%%%%%%%%%%%%%%%%%%%%%%%%%%%%%%%
%% chapter4.tex
%% NOVA thesis document file
%%%%%%%%%%%%%%%%%%%%%%%%%%%%%%%%%%%%%%%%%%%%%%%%%%%%%%%%%%%%%%%%%%%%

\typeout{NT FILE chapter4.tex}%

\chapter{Design}
\label{cha:design}

\glsresetall
 
\section{Preliminary Protocol Concept}
\label{sec:preliminary_protocol_concept}

The protocol to be developed will be a structured model for threat modeling in
non-hierarchical organizations, functioning similarly to STRIDE, but adapted to
the specificities of distributed governance
\cite{ThreatModelingdesigningForSecurity}. It will not be a software, but a
formal set of documented guidelines and methodologies for identifying, analyzing
and mitigating threats in horizontal environments \cite{Colbac}. Its format will
be textual and structured as an academic and technical document, offering a
practical and theoretical framework for security in decentralized structures.

The primary audience includes members of the organization who use the protocol
to analyze threats in their specific context, as well as experts in information
security and distributed governance who seek methods adapted to non-hierarchical
environments. The protocol will serve to improve organizational resilience
without compromising democratic participation, ensuring that security is
integrated into the collaborative dynamics of horizontal organizations.

\subsection{Target Audience}
\label{subsec:target_audience}

The protocol is designed for members of horizontal organizations without
specialized cybersecurity expertise. Unlike traditional expert-oriented models,
our protocol emphasizes simplicity and accessibility. It supports stakeholders
involved in decision-making, operational activities, conflict resolution, and
coordination tasks, as well as informal community groups, by providing clear
guidance and intuitive methods for effectively dealing with security threats.

\section{Security and Governance Requirements}
\label{sec:security_governance_requirements}

The proposed protocol for horizontal organizations must meet security and
governance requirements that ensure both participatory inclusion and robustness
against internal and external threats. In decentralized structures, where the
absence of formal hierarchy can be seen as both an advantage and a challenge, it
is imperative that the protocol be designed to integrate mechanisms that
preserve horizontality without compromising organizational resilience.
Transparency, in this context, emerges as one of the central pillars
\cite{Colbac}. All events related to the authorization of actions or
modifications must be recorded in an immutable manner, ensuring that actions can
be reliably audited by any member of the organization. These immutable records,
built on technologies such as blockchain or similar cryptographic structures,
ensure traceability and eliminate the possibility of unauthorized changes,
maintaining cohesion between organizational principles and technological
mechanisms.

Democratic participation also plays a fundamental role in the design of the
protocol \cite{Colbac}. In a horizontal organization, decisions must reflect the
collective will, and to this end, the protocol must integrate digital voting
systems that guarantee both the privacy and security of votes \cite{Colbac}. In
addition, it must be possible to temporarily delegate authority to individuals
or groups to deal with situations that require specific expertise, always
ensuring that such delegation can be revoked and that records of actions are
accessible for collective audit \cite{Colbac}.

Another crucial aspect is the flexibility of the protocol, especially in
scenarios where the organization needs to switch between centralized and
distributed governance modes. This transition capacity must be implemented in a
way that the levels of centralization are temporary and properly tracked,
ensuring that control can quickly return to the collective. To this end, the
protocol must provide mechanisms such as emergency tokens, which allow the
execution of critical actions in exceptional situations, as long as such actions
are recorded and subject to retroactive validation by the members of the
organization \cite{Colbac}.

For the protocol to be scalable, it is essential that it supports organizations
of different sizes and degrees of complexity. This includes implementing
distributed validation systems that allow collaborative verification of actions
without burdening individuals or single points of control \cite{Colbac}.
Furthermore, threat modeling should be iterative and adaptive, incorporating
techniques such as simulations based on real-world scenarios to identify
emerging vulnerabilities and adjust countermeasures accordingly. These
mechanisms ensure that the protocol remains functional and effective even when
the organizational scale increases or its internal dynamics change.

\section{Evaluation Strategy}
\label{sec:evaluation_strategy}

The evaluation of the proposed protocol will be conducted in an experimental and
comparative manner, using organizational groups with different degrees of
horizontality. The main objective of this process is to validate the protocol's
ability to identify, mitigate and prevent threats in horizontal environments,
while comparing its effectiveness with consolidated frameworks, such as
\gls{stride}.

Initially, candidate organizations representing different levels of power
distribution and autonomy will be selected. Each organization will undergo
structured training sessions, addressing both the developed protocol and
\gls{stride}, ensuring that all participants understand the tools used.

To operationalize the evaluation, a three-stage methodology will be applied. In
the first stage, each group will organize a threat model for its structure based
on the proposed protocol. In parallel, a second group, within the same
organization, will apply \gls{stride} to the same scenario. Both sessions will
be monitored to document the process and collect quantitative and qualitative
data.

The resulting models will be analyzed according to predefined metrics,
including:

\begin{itemize}
    \item \textbf{Precision}: assessment of the number of correctly identified threats
in relation to the total detected.
    \item \textbf{Recall}: proportion of threats identified by the protocol in relation
to those existing in the reference model.
    \item \textbf{Operational Latency}: total time required to complete each phase
of the modeling process.
    \item \textbf{User Feedback}: collection of qualitative opinions on
the usability and clarity of the protocol from participants.
\end{itemize}

The second stage will involve simulating threat scenarios, including Sybil
attacks, quorum manipulation and consensus failures, replicating realistic
situations faced by horizontal organizations. The response to these threats will
be compared between the protocols, considering the quality of the proposed
solutions and the response time of the groups.

Finally, in the third stage, the organizations will be subjected to case
studies, in which the protocol will be implemented in full and monitored over
time. The performance of the protocol will be analyzed in terms of its
resilience to the identified threats, its adaptability to organizational
dynamics and its ability to promote participation and transparency.

\section{Research Questions}
\label{sec:research_questions}

\begin{enumerate}
    \item How can the protocol balance efficiency and democratic participation
in horizontal organizations?
    \item What are the best practices for integrating security and governance
in decentralized structures?
    \item How can the protocol adapt to different levels of horizontality and
organizational dynamics?
    \item How can it improve resilience against internal and external threats
while maintaining transparency and inclusive participation?
\end{enumerate}

\section*{}
Chapter 4 presented the preliminary design of a protocol specifically aimed at
threat modeling in horizontal organizations, detailing fundamental security and
governance requirements, such as mechanisms for democratic participation,
temporary delegation of authority, and auditable records. The importance of
aligning technological and organizational processes to ensure robustness against
internal and external threats was highlighted. Based on this conceptual
development, Chapter 5 summarizes the main results achieved by this research,
critically evaluating the proposed protocol, identifying its contributions in
relation to traditional methodologies, and exploring practical scenarios for
future validations in real environments.
