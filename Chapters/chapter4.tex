%!TEX root = ../template.tex
%%%%%%%%%%%%%%%%%%%%%%%%%%%%%%%%%%%%%%%%%%%%%%%%%%%%%%%%%%%%%%%%%%%%
%% chapter4.tex
%% NOVA thesis document file
%%%%%%%%%%%%%%%%%%%%%%%%%%%%%%%%%%%%%%%%%%%%%%%%%%%%%%%%%%%%%%%%%%%%

\typeout{NT FILE chapter4.tex}%

\chapter{Design}
\label{cha:design}

\glsresetall
 
\section{Conceito Preliminar do Protocolo}
\label{sec:preliminary_protocol_concept}

O protocolo a ser desenvolvido será um modelo estruturado para a modelagem de
ameaças em organizações não-hierárquicas, funcionando de forma semelhante ao
STRIDE, mas adaptado às especificidades da governança distribuída. Ele não será
um software, mas um conjunto formal de diretrizes e metodologias documentadas
para identificação, análise e mitigação de ameaças em ambientes horizontais. Seu
formato será textual e estruturado como um documento acadêmico e técnico,
oferecendo um referencial prático e teórico para segurança em estruturas
descentralizadas.

A audiência primária inclui membros da organização que utilizam o protocolo para
analisar ameaças em seu contexto específico, bem como especialistas em segurança
da informação e governança distribuída que buscam métodos adaptados a ambientes
sem hierarquia. O protocolo servirá para aprimorar a resiliência organizacional
sem comprometer a participação democrática, garantindo que a segurança seja
integrada à dinâmica colaborativa das organizações horizontais.

\section{Requisitos de Segurança e Governança}
\label{sec:security_governance_requirements}

Para garantir que o protocolo atenda às demandas de segurança e governança
horizontal, os seguintes requisitos são definidos e sua integração à modelagem
de ameaças é delineada:
    
\begin{itemize}

    \item \textbf{Transparência}: Todos os processos de autorização devem ser
registrados de forma imutável, permitindo auditorias completas. Para isso, o
protocolo utilizará mecanismos criptográficos de registro imutável, como logs
descentralizados e assinaturas digitais verificáveis, garantindo rastreabilidade
sem comprometer a horizontalidade.
    
    \item \textbf{Participação Democrática}: A tomada de decisão deve ser
inclusiva, permitindo contribuições de todos os membros. O protocolo integrará
mecanismos de consenso inspirados em sistemas de governança distribuída, como
votação segura e delegação dinâmica de autoridade, reduzindo riscos como
manipulação de quórum ou ataques Sybil.
    
    \item \textbf{Flexibilidade}: O protocolo deve suportar mudanças rápidas
entre modos de governança centralizados e descentralizados, conforme as
necessidades da organização. Para isso, serão definidos critérios e mecanismos
para transições seguras entre modelos de tomada de decisão, permitindo ajustes
dinâmicos sem expor a organização a vulnerabilidades.
    
    \item \textbf{Escalabilidade}: Deve ser capaz de operar eficientemente em
organizações de diferentes tamanhos e níveis de complexidade. A modelagem de
ameaças abordará a expansão das interações e dos processos de segurança sem
comprometer a governança distribuída, garantindo que mecanismos como controle de
acesso colaborativo e validação distribuída continuem funcionais em diferentes
contextos organizacionais.

\end{itemize}

\section{Estratégia de Avaliação}
\label{sec:evaluation_strategy}

A estratégia de avaliação do protocolo será conduzida através da sua aplicação
em grupos selecionados que apresentam diferentes níveis de horizontalidade. Isso
permitirá analisar como a estrutura de governança influencia a eficiência, a
segurança e a adesão ao protocolo. A avaliação será baseada em estudos de caso e
simulações que repliquem ameaças realistas dentro desses contextos, considerando
os seguintes critérios:

\begin{itemize}

    \item \textbf{Eficiência}: O tempo necessário para concluir processos de autorização e
tomadas de decisão será medido utilizando logs de auditoria e análise da
latência operacional em diferentes modelos de governança. Serão comparadas
abordagens baseadas em consenso distribuído e em decisões locais para
identificar quais garantem maior agilidade sem comprometer a segurança.

    \item \textbf{Eficácia}: A capacidade do protocolo de identificar e mitigar ameaças
será testada em cada grupo, analisando-se vulnerabilidades estruturais
associadas à sua dinâmica organizacional. Modelos como STRIDE e árvores de
ataque serão utilizados para simular cenários adversos e avaliar a resposta do
sistema a ataques internos e externos.

    \item \textbf{Aceitação pelos Usuários}: A facilidade de uso e o nível de adesão serão
medidos através da interação com diferentes grupos e seus modos de tomada de
decisão. Experimentos controlados incluirão feedback qualitativo e quantitativo
sobre a usabilidade do protocolo em contextos altamente horizontais e híbridos.

    \item \textbf{Resiliência a Ameaças}: O desempenho do protocolo diante de ataques
simulados, incluindo manipulação de quórum, ataques Sybil e falhas de consenso,
será testado em cada nível de horizontalidade. Serão analisadas abordagens de
mitigação específicas para organizações sem liderança central e grupos com
hierarquia flexível, garantindo um protocolo adaptável a diferentes cenários.

\end{itemize}

\section{Questões de Pesquisa}
\label{sec:research_questions}

\begin{enumerate}
    \item Como o protocolo pode equilibrar eficiência e participação
democrática em organizações horizontais?
    \item Quais são as melhores práticas para integrar segurança e
governança em estruturas descentralizadas?
    \item Como o protocolo pode se adaptar a diferentes níveis de
horizontalidade e dinâmicas organizacionais?
    \item De que forma ele pode melhorar a resiliência contra ameaças
internas e externas, mantendo a transparência e a participação
inclusiva?
\end{enumerate}

