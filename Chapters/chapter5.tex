%!TEX root = ../template.tex
%%%%%%%%%%%%%%%%%%%%%%%%%%%%%%%%%%%%%%%%%%%%%%%%%%%%%%%%%%%%%%%%%%%%
%% chapter5.tex
%% NOVA thesis document file
%%%%%%%%%%%%%%%%%%%%%%%%%%%%%%%%%%%%%%%%%%%%%%%%%%%%%%%%%%%%%%%%%%%%

\typeout{NT FILE chapter5.tex}%

\chapter{Conclusion}
\label{cha:conclusion}

\glsresetall

Este capítulo apresentou os fundamentos da modelagem de ameaças e suas
interseções com diferentes estruturas organizacionais, destacando desafios
e soluções em contextos hierárquicos e horizontais. Em estruturas
horizontais, a ausência de hierarquias formais exige mecanismos claros de
segurança coletiva, enquanto o centralismo democrático oferece um modelo
para combinar participação e execução eficiente, aplicável em contextos
contemporâneos, como algoritmos e protocolos colaborativos \cite{Colbac,
EstatutosDoPCP}.

A integração de metodologias como \gls{stride}, árvores de ataque e
práticas colaborativas demonstra que a segurança é um esforço tanto técnico
quanto social, sendo essencial adaptar abordagens a diferentes dinâmicas
organizacionais. Assim, este capítulo reafirma a importância de combinar
inovação e estratégia para fortalecer a resiliência em organizações
complexas e descentralizadas.