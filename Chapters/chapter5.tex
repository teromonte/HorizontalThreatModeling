%!TEX root = ../template.tex
%%%%%%%%%%%%%%%%%%%%%%%%%%%%%%%%%%%%%%%%%%%%%%%%%%%%%%%%%%%%%%%%%%%%
%% chapter5.tex
%% NOVA thesis document file
%%%%%%%%%%%%%%%%%%%%%%%%%%%%%%%%%%%%%%%%%%%%%%%%%%%%%%%%%%%%%%%%%%%%

\typeout{NT FILE chapter5.tex}%

\chapter{Evaluation}
\label{cha:evaluation}

\glsresetall

\section{Evaluation Strategy}
\label{sec:evaluation_strategy}

The evaluation of the proposed protocol will be conducted through an
experimental and comparative approach, involving real case studies in
organizations with different degrees of horizontality. The main objective is to
validate the effectiveness of the protocol in identifying, mitigating and
preventing threats in horizontal organizational contexts, directly comparing its
performance with established frameworks, such as \gls{stride}.

Candidate organizations will be selected to represent a variety of horizontal
structures, ensuring diversity in power distribution, size and operational
complexity. Participants from each organization will receive structured training
sessions covering both the proposed protocol and \gls{stride}, ensuring
familiarity and a level playing field for comparison.

\textbf{Parallel Threat Modeling Sessions:}
Each organization will conduct simultaneous threat modeling sessions using both
the proposed protocol and \gls{stride} in identical scenarios. These sessions
will be observed to systematically document the process and collect data,
analyzed according to the following clearly defined metrics:

\begin{itemize}
\item \textbf{Precision}: The ratio of valid threats correctly identified to the
total number of threats identified by the protocol.
\item \textbf{Coverage (Recall)}: The proportion of existing threats identified
by the protocol relative to a reference set established by experts.
\item \textbf{Operational Efficiency}: Total time (latency) required to complete
each phase of the threat modeling process, including identification, analysis,
and mitigation planning.
\item \textbf{Usability}: Participants' opinions on the ease of use, clarity,
and overall effort required to learn and effectively apply the protocol.
\end{itemize}

By systematically applying these clearly defined metrics and evaluation steps,
this strategy ensures a comprehensive comparative analysis, providing robust
empirical evidence on the effectiveness of the protocol relative to \gls{stride}
in horizontal organizations.

