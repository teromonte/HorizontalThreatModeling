%!TEX root = ../template.tex
%%%%%%%%%%%%%%%%%%%%%%%%%%%%%%%%%%%%%%%%%%%%%%%%%%%%%%%%%%%%%%%%%%%%
%% chapter5.tex
%% NOVA thesis document file
%%%%%%%%%%%%%%%%%%%%%%%%%%%%%%%%%%%%%%%%%%%%%%%%%%%%%%%%%%%%%%%%%%%%

\typeout{NT FILE chapter5.tex}%

\chapter{Evaluation}
\label{cha:evaluation}

\glsresetall

To validate the effectiveness, usability, and unique contributions of the threat
modeling protocol developed in this research (see Chapter \ref{cha:solution}), a
structured evaluation is required. The primary objective of this evaluation is
to empirically assess whether the protocol achieves its design goals,
particularly in its capacity to identify a broader range of threats relevant to
non-hierarchical organizations compared to traditional methods. Furthermore, the
evaluation seeks to measure the quality of the outcomes and of the collaborative
experience of the participants.

This chapter outlines the methodology for a comparative experiment designed to
measure the protocol's performance against an established industry standard,
STRIDE. The evaluation is structured around a real world workshop and uses a
combination of quantitative output metrics and qualitative participant feedback
to provide a holistic assessment.

\section{Experimental Design}
\label{sec:experimental_design}

The evaluation will be conducted as a two session workshop with members
of multiple horizontal organizations. To ensure a robust and fair comparison, the
experiment will alternate the order of methodologies between sessions. Participants
will be randomly divided into two groups, one for each session.

If there are not enough people to fill two groups, the same group
will use both methodologies in different sessions. In this case, the
order of methodologies will be randomized within different
organizations to avoid bias.

\section{Evaluation Metrics}
\label{sec:evaluation_metrics}

The outputs from the sessions using the proposed protocol and STRIDE will be
compared using a set of objective metrics. These metrics are designed to move
beyond a simple count of threats to assess the scope, severity, and practicality
of the results.

\subsection{Threat Volume and Diversity}
\label{subsec:threat_volume_and_diversity}

This metric assesses the breadth and scope of the threats identified by each
methodology. The goal is to determine if the proposed protocol successfully
guides participants to consider risks beyond the purely technical domain.

\begin{itemize}
\item \textbf{Measurement:} After each session, the complete list of unique
threats will be categorized. A predefined set of categories will be used for
consistency:
\begin{itemize}
\item \textbf{Technical:} Threats related to software flaws, network
vulnerabilities, data breaches, or malware.
\item \textbf{Governance/Process:} Threats related to decision making failures,
loss of critical credentials due to poor process, single points of human
failure, or flawed onboarding/offboarding.
\item \textbf{Social/Human:} Threats originating from human actors, such as
malicious insiders, social engineering, member conflict impacting operations, or
unintentional errors.
\item \textbf{Third Party:} Threats originating from the failure or compromise
of an external service provider.
\end{itemize}
\item \textbf{Analysis:} The total number of threats in each category will be
counted and compared for both methodologies. It is hypothesized that STRIDE will
produce a high volume of threats in the "Technical" category, whereas the
proposed protocol will yield a more balanced distribution across all categories,
particularly highlighting "Governance/Process" and "Social/Human" threats.
\end{itemize}

\subsection{Threat Quality and Relevance}
\label{subsec:threat_quality_and_relevance}

This dimension evaluates whether the identified threats are significant and
pertinent to the organization's core concerns, moving the focus from quantity to
severity.

\begin{itemize}
\item \textbf{Prioritized Risk Score:} For the top 10 threats identified by each
method, a risk score will be calculated based on the group's assessment of
Impact and Likelihood (as performed in Step 6 of the protocol). Using a simple
scale (e.g., Low=1, Medium=2, High=3), the score will be calculated as:
\textit{Risk = Impact x Likelihood}. The total risk score for the top threats
from each method will be compared to determine which methodology was more
effective at surfacing high priority issues.
\item \textbf{Relevance to Critical Assets:} Using the list of critical assets
defined in Step 1 of the protocol, we will count the number of these assets that
have at least one valid threat identified against them by each method. This
metric measures how well each methodology focused the discussion on what the
organization explicitly defined as most valuable.
\end{itemize}

\subsection{Actionability of Outcomes}
\label{subsec:actionability_of_outcomes}
An effective threat modeling process must lead to concrete, implementable
security improvements. This metric quantifies the practical value of the
mitigations proposed.
\begin{itemize}
\item \textbf{Number of Actionable Mitigations:} We will count the number of
proposed mitigations from each session that are specific, measurable, and
assignable. An actionable mitigation is one that can be converted directly into
a task (e.g., "Enable two factor authentication for all member accounts") rather
than a vague goal (e.g., "Improve login security"). This metric provides a
direct measure of the protocol's ability to translate analysis into action.
\end{itemize}

\section{Participant Experience and Usability}
\label{sec:participant_experience}

Given that the proposed protocol is designed 
for democratic and inclusive
participation, the subjective experience of the users is a primary criterion for
success. This will be measured using post session surveys.

\subsection{Quantitative Feedback via Surveys}
\label{subsec:quantitative_feedback}

After each threat modeling session, participants will be asked to anonymously
complete a short survey. They will rate their agreement with a series of
statements on a 5 point Likert scale (1 = Strongly Disagree, 5 = Strongly
Agree).

The survey will include the following core questions:

\begin{enumerate}
\item \textbf{Clarity:} "I understood the goals and steps of the process we just used."
\item \textbf{Coverage:} "This method helped us consider a wide range of relevant threats."
\item \textbf{Focus:} "This method helped us focus on the most critical risks to our organization."
\item \textbf{Empowerment:} "I felt comfortable contributing my ideas, regardless of my technical knowledge."
\item \textbf{Effectiveness:} "Overall, I believe this session was an effective use of our time to improve our security."
\end{enumerate}

The mean scores for each question will be calculated and compared between the
two methodologies to provide quantitative insight into usability, inclusivity,
and perceived effectiveness.

\subsection{Qualitative Feedback}
\label{subsec:qualitative_feedback}

In addition to scaled questions, the survey will include open ended questions to
gather richer, qualitative insights. These questions will ask participants what
they found most and least useful about each process and invite any other
suggestions for improvement. This feedback will be crucial for identifying
specific strengths and areas for refinement in the protocol's design and
facilitation guidance.

\section{Results}
\label{sec:results}

The proposed protocol was evaluated in practical workshops with two non-hierarchical
organizations: ComuniDária, a non-profit community association, and Frente Anti-Racista,
a political activist collective. Each organization participated in two separate sessions,
one using the proposed protocol and another using the traditional STRIDE methodology.
The following sections present the results of these workshops, analyzed according to the
evaluation metrics defined in section ~\ref{sec:evaluation_metrics}.

\subsection{Threat Volume and Diversity}
\label{subsec:threat_volumediversity}

\begin{table}[]
    \caption{Quantitative Comparison of Threat Modeling Methodologies}
    \label{tab:evaluation-results}
    \scriptsize
    \resizebox{\textwidth}{!}{
        \begin{tabular}{|p{0.45\textwidth}|p{0.25\textwidth}|p{0.25\textwidth}|}
            \toprule
            Evaluation Metric
                &STRIDE
                &Proposed Protocol\\
            \midrule
            \textbf{Threat Volume}
                & 
                & \\
            Total Threats Identified
                &11
                &16\\
            \midrule
            \textbf{Threat Diversity (\% of Total Threats)}
                & 
                & \\
            \hspace{1em} Technical
                &45\% (5 threats)
                &31.25\% (5 threats)\\
            \hspace{1em} Governance/Process
                &0\% (0 threats)
                &18\% (3 threats)\\
            \hspace{1em} Social/Human
                &36\% (4 threats)
                &37\% (6 threats)\\
            \hspace{1em} Third Party
                &18\% (2 threats)
                &12\% (2 threats)\\
            \midrule
            \textbf{Actionability of Outcomes}
                & 
                & \\
            Total Actionable Mitigations Proposed
                &2
                &12\\
            \bottomrule
        \end{tabular}%
    }
\end{table}

As hypothesized in the evaluation design, the results demonstrate a
significant divergence in the types of threats identified by each methodology.
The STRIDE sessions predominantly produced threats categorized as 'Technical'
and 'Third Party'. For instance, with Frente Anti-Racista, STRIDE identified
risks such as a lack of two-factor authentication and dependency on Meta's
platform security. While valid, these threats are external or purely
technical in nature.

In contrast, the proposed protocol yielded a more balanced distribution, with
a clear emphasis on 'Social/Human' and 'Governance/Process' threats. The Frente
Anti-Racista workshop using our protocol identified critical risks such as "a new
member joining with the intent to infiltrate" (Social/Human) and "failure to comply
with bureaucracy leading to loss of state funding" (Governance/Process). These
findings suggest that the proposed protocol is more effective at surfacing the
socio-technical risks that are particularly relevant to non-hierarchical and
politically-oriented organizations.

\subsection{Threat Quality and Relevance}
\label{subsec:threat_quality_relevance}

Beyond volume, the quality and relevance of the identified threats were analyzed.
For both ComuniDária and Frente Anti-Racista, the highest-priority threats identified
using the proposed protocol were directly linked to their core mission and organizational
vulnerabilities. For example, Frente Anti-Racista prioritized "an extremist physically
attacking a member using leaked data" (Impact: 5, Likelihood: Medium) and "an infiltrator
leaking internal communications" (Impact: 4, Likelihood: High).

The STRIDE methodology, while identifying important technical weaknesses, did not surface
these kinds of existential, mission-critical threats. Its focus remained on the security
of data flows, leading to priorities like "potential for data interception on WhatsApp,"
which, while important, does not capture the same level of organizational risk.

The proposed protocol's process begins by identifying Critical Assets, including
intangible ones like 'reputation' and 'member trust'. Consequently, the threats
identified were inherently relevant to these assets. In contrast, STRIDE's asset model is
the system's data flow diagram. This led to a disconnect, where threats were identified
against processes (e.g., 'Gerenciar Divulgação Pública') but not explicitly against the
assets the organization itself deemed most valuable.

\subsection{Actionability of Outcomes}
\label{subsec:actionability_outcomes}

An effective threat modeling process must result in actionable mitigations. The workshops
demonstrated a clear difference in the nature of the proposed solutions. The proposed
protocol guided organizations toward developing internal, process-based mitigations. For
example, in response to the "infiltrator" threat, Frente Anti-Racista proposed creating
"a screening process for new members." For the risk of data leaks, ComuniDária proposed
"registering which member has access to each page or file."

The mitigations derived from the STRIDE sessions were often technical fixes (e.g., "enable
two-factor authentication") or pointed to dependencies on external providers ("depender da
segurança da Google"). This suggests the proposed protocol is more effective at empowering
organizations to improve their security posture through changes in their own collective
governance and processes, which is a core goal for non-hierarchical structures.

\section*{}
