%!TEX root = ../template.tex
%%%%%%%%%%%%%%%%%%%%%%%%%%%%%%%%%%%%%%%%%%%%%%%%%%%%%%%%%%%%%%%%%%%%
%% chapter5.tex
%% NOVA thesis document file
%%
%% Chapter with a short latex tutorial and examples
%%%%%%%%%%%%%%%%%%%%%%%%%%%%%%%%%%%%%%%%%%%%%%%%%%%%%%%%%%%%%%%%%%%%

\typeout{NT FILE chapter5.tex}%
\   
\chapter{Work Plan}
\label{cha:work_plan}

\glsresetall

\section{Evaluation}
\label{sec:evaluation}

A avaliação será conduzida em duas etapas principais: um estudo de
caso prático e um estudo de usuários. 

Na primeira etapa, será realizada a aplicação do protocolo em organizações
reais que operam de maneira horizontal, como cooperativas de trabalhadores,
sindicatos, grupos ativistas e projetos de software de código aberto. Cada organização
participante será analisada para identificar suas necessidades específicas de segurança e
como o protocolo pode ser adaptado para atender a essas necessidades. Durante esta
fase, serão coletados dados sobre a eficácia do protocolo em identificar e mitigar
ameaças, bem como sobre a facilidade de uso e a aceitação pelos membros da organização.

A segunda etapa consistirá em um estudo de usuários, onde participantes de
diferentes organizações horizontais serão convidados a utilizar o protocolo desenvolvido.
Este estudo será comparativo, envolvendo também a aplicação de um protocolo de
modelagem de ameaças tradicional, como STRIDE, para servir como base de comparação. Os
participantes serão divididos em dois grupos: um grupo utilizará o novo protocolo, enquanto
o outro grupo utilizará o protocolo tradicional. Durante o estudo, os
participantes serão solicitados a identificar ameaças em cenários específicos fornecidos. A
eficácia de cada protocolo será medida com base na quantidade e na qualidade das
ameaças identificadas. Espera-se que o novo protocolo
demonstre uma maior eficácia na identificação de ameaças e cenários de colusão,
refletindo a necessidade de soluções de segurança adaptadas para organizações
não-hierárquicas. 

Os dados coletados durante o estudo de caso prático e o estudo de usuários
serão analisados quantitativa e qualitativamente. A análise quantitativa incluirá
métricas como o número de ameaças identificadas, a taxa de falsos positivos e
negativos, e o tempo necessário para completar a modelagem de ameaças. A análise
qualitativa envolverá feedback dos participantes sobre a usabilidade do protocolo, a
clareza das instruções e a percepção geral de segurança proporcionada pelo protocolo. 

Para fornecer uma visão abrangente da eficácia do novo protocolo, os
resultados serão comparados com os obtidos utilizando protocolos tradicionais, como
STRIDE. Esta comparação destacará as vantagens e desvantagens de cada abordagem em
contextos horizontais, fornecendo uma base sólida para futuras pesquisas e
desenvolvimentos no campo da segurança cibernética para organizações não-hierárquicas. 

A avaliação proposta permitirá uma compreensão detalhada da eficácia do
protocolo de modelagem de ameaças desenvolvido, considerando a horizontalidade como um
ativo. Os resultados fornecerão insights valiosos sobre como melhorar e adaptar o
protocolo para diferentes tipos de organizações horizontais, garantindo que as soluções
de segurança respeitem e reforcem os princípios de participação e igualdade. 

\section{Scheduling}
\label{sec:scheduling}

We also show some stuff which is not that common!