%!TEX root = ../template.tex
%%%%%%%%%%%%%%%%%%%%%%%%%%%%%%%%%%%%%%%%%%%%%%%%%%%%%%%%%%%%%%%%%%%%
%% chapter5.tex
%% NOVA thesis document file
%%%%%%%%%%%%%%%%%%%%%%%%%%%%%%%%%%%%%%%%%%%%%%%%%%%%%%%%%%%%%%%%%%%%

\typeout{NT FILE chapter5.tex}%

\chapter{Evaluation}
\label{cha:evaluation}

\glsresetall

To validate the effectiveness, usability, and unique contributions of the threat
modeling protocol developed in this research (see Chapter \ref{cha:solution}), a
structured evaluation with the objective to empirically assess whether the protocol
achieves better results in identifying a broader range of threats relevant to
non-hierarchical organizations compared to \gls{stride}.

The evaluation is structured around a real world workshop and uses a
combination of quantitative output metrics to provide a holistic assessment.

\section{Experimental Design}
\label{sec:experimental_design}

The evaluation will be conducted as a two session workshop with members
of multiple horizontal organizations. To ensure a fair comparison, the
experiment will alternate the order of methodologies between sessions. Participants
will be randomly divided into two groups, one for each session.

If there are not enough people to fill two groups, the same group
will use both methodologies in different sessions. In this case, the
order of methodologies will be randomized within different
organizations to avoid bias.

\section{Evaluation Metrics}
\label{sec:evaluation_metrics}

The outputs from the sessions using the proposed protocol and \gls{stride} will be
compared using a set of objective metrics.

\subsection{Threat Volume and Diversity}
\label{subsec:threat_volume_and_diversity}

This metric assesses the breadth and scope of the threats identified by each
methodology. The goal is to determine if the proposed protocol successfully
guides participants to consider risks beyond the purely technical domain.

\begin{itemize}
    \item \textbf{Measurement:} After each session, the complete list of unique
    threats will be categorized. A predefined set of categories will be used for
    consistency:

    \begin{itemize}
        \item \textbf{Technical:} Threats related to software flaws, network
        vulnerabilities, data breaches, or malware.

        \item \textbf{Governance/Process:} Threats related to decision making failures,
        loss of critical credentials due to poor process, single points of human
        failure, or flawed onboarding/offboarding.

        \item \textbf{Social/Human:} Threats originating from human actors, such as
        malicious insiders, social engineering, member conflict impacting operations, or
        unintentional errors.

        \item \textbf{Third Party:} Threats originating from the failure or compromise
        of an external service provider.
    \end{itemize}

    \item \textbf{Analysis:} The total number of threats in each category will be
    counted and compared for both methodologies. It is hypothesized that \gls{stride} will
    produce a high volume of threats in the "Technical" category, whereas the
    proposed protocol will show a more balanced distribution across all categories,
    particularly highlighting "Governance/Process" and "Social/Human" threats.
\end{itemize}

\subsection{Relevance to Critical Assets}
\label{subsec:relevance_to_critical_assets}

This dimension evaluates whether the identified threats are significant and
pertinent to the organization, moving the focus from quantity to
severity. Using the list of critical assets defined in Step 1 of the protocol,
we will count the number of these assets that have at least one valid
threat identified against them by each method. This metric measures how well
each methodology focused the discussion on what the organization explicitly
defined as most valuable.

\subsection{Actionability of Outcomes}
\label{subsec:actionability_of_outcomes}

An effective threat modeling process must lead to concrete, implementable
security improvements. This metric quantifies the practical value of the
mitigations proposed.

We will count the number of proposed mitigations from each session that are specific,
measurable, and assignable. An actionable mitigation is one that can be converted directly
into a task (e.g., "Enable two factor authentication for all member accounts") rather
than a vague goal (e.g., "Improve login security"). This metric provides a
direct measure of the protocol's ability to translate analysis into action.

\section{Participant Experience and Usability}
\label{sec:participant_experience}

Given that the proposed protocol is designed for democratic and inclusive participation,
the subjective experience of the users is a primary criteria for
success.

We will evaluate the experience through careful observation during the workshops. We
will watch for specific things like who is speaking? Is it only the members with technical
skills, or is everyone contributing ideas? Do people seem interested and focused? Are they
building on each other's ideas? 

At the end of each session, we will have a short, informal
conversation. We will ask what felt useful, what was difficult, and how the
process could be better. This direct feedback is more valuable than numbers
on a scale. It helps us see if the protocol truly serves the collective.

\subsection{Ethics and Protection of Organizations}
\label{subsec:ethics_protection}

A principle of this study is the protection and confidentiality
of the organizations that agree to participate. The workshops
involves the disclosure of sensitive information, including existing vulnerabilities,
internal processes, and strategic challenges, creating a significant responsibility to
safeguard this data from any potential harm.

To honor this responsibility, a strict ethical framework was followed. All
information gathered during the evaluation is treated with
confidentiality. Any specific findings, examples, or details about the
participating organizations are included in this thesis only after receiving
their explicit and informed authorization. This ensures that the collaboration
and the data shared are used responsibly and do not expose the organizations to
risk.

\section{Results}
\label{sec:results}

The proposed protocol was evaluated in practical workshops with two non-hierarchical
organizations: a non-profit immigrant association and a political activist collective.
Each organization participated in two separate sessions, one using the
proposed protocol and another using the traditional \gls{stride} methodology.
The following sections present the results of these workshops, analyzed according to the
evaluation metrics defined in section ~\ref{sec:evaluation_metrics}.

\subsection{Threat Volume and Diversity}
\label{subsec:threat_volumediversity}



\begin{table}[!htbp]
    \caption{Quantitative Comparison of Threat Modeling Methodologies}
    \label{tab:evaluation-results}
    \scriptsize
    \resizebox{\textwidth}{!}{
        \begin{tabular}{|p{0.45\textwidth}|p{0.25\textwidth}|p{0.25\textwidth}|}
            \toprule
            Evaluation Metric
                &\gls{stride}
                &Proposed Protocol\\
            \midrule
            \textbf{Threat Volume}
                & 
                & \\
            Total Threats Identified
                &11
                &16\\
            \midrule
            \textbf{Threat Diversity (\% of Total Threats)}
                & 
                & \\
            \hspace{1em} Technical
                &45\% (5 threats)
                &31.25\% (5 threats)\\
            \hspace{1em} Governance/Process
                &0\% (0 threats)
                &18\% (3 threats)\\
            \hspace{1em} Social/Human
                &36\% (4 threats)
                &37\% (6 threats)\\
            \hspace{1em} Third Party
                &18\% (2 threats)
                &12\% (2 threats)\\
            \midrule
            \textbf{Actionability of Outcomes}
                & 
                & \\
            Total Actionable Mitigations Proposed
                &2
                &12\\
            \bottomrule
        \end{tabular}%
    }
\end{table}

As we hypothesized in the evaluation design, the results show a big difference
in the types of threats identified by each methodology. The numbers themselves
tell a story. In total, the proposed protocol found more threats across the two
workshops—16 in total, compared to the 11 found by \gls{stride}.

The \gls{stride} sessions, for both ComuniDária and Frente Anti-Racista, gave us a
list of threats that were very technical or about external things. We saw things
like “Phishing”, “lack of backup”, and “mobile device security failure”.
Important, yes. But these are problems any organization could have. And most
critically, across both workshops, \gls{stride} identified exactly zero threats in the
Governance/Process category. It is a blind spot.

Then we look at our protocol. The picture is completely different. The
distribution of threats is much more balanced. For Frente Anti-Racista, it
uncovered critical governance threats like “a person pretending to be a member
to vote in an assembly” and the risk of “not complying with bureaucracy and
losing state funding.” For ComuniDária, it found the risk of “misconduct for
personal gain.” These are the deep, internal, structural risks that can really
damage a non-hierarchical group. These are threats \gls{stride} cannot see.

Also, our protocol found more Social/Human threats (6 versus 4), and they were
more specific to the context of these organizations. Not just a generic “user
error,” but the very real danger of “a new member joining just to infiltrate and
leak information.” This shows that the protocol guides the conversation to the
risks that are most real and dangerous for these specific groups, moving beyond
the purely technical view into the socio-technical reality of their work.

\subsection{Relevance to Critical Assets}
\label{subsec:threat_quality_relevance}

Beyond just the number of threats, it is the quality and relevance that showed
the biggest difference. A threat modeling process must protect what is most
important.

Our protocol's process starts by asking a simple question: What are your
critical assets? And the groups did not just list technical things. ComuniDária
identified intangibles like credibility (reputation) and member trust.
Frente Anti-Racista listed their operationality of the nucleo and the very
safety of their members.

Because of this, the threats we found were naturally tied to these assets. It is
simple logic. When Frente Anti-Racista prioritized an extremist physically
attacking a member using leaked data or an infiltrator leaking internal
communications, these are direct attacks on their most valued assets. They are
existential threats.

The \gls{stride} methodology is different. Its asset model is the system's data
flow diagram. So it finds threats against a process, like the flow of
information in a messaging app, but not explicitly against the assets the
organization itself said were most valuable. While it identified important
technical weaknesses, the priorities were different, like potential for data
interception on WhatsApp. This is important, yes, but it does not capture the
same level of organizational risk. It is a technical vulnerability, not a threat
to the soul of the organization.

This created a disconnect. \gls{stride} was looking at the machine, but our protocol
was looking at the mission. And for these groups, that is where the real risk
lives.

\subsection{Actionability of Outcomes}
\label{subsec:actionability_outcomes}

A threat modeling process must lead to action. Real change. Without this, it is
only talk. The workshops showed a very clear difference here.

The proposed protocol guided the organizations to create their own internal,
process-based mitigations. These are solutions they can build themselves. For
example, in response to the "infiltrator" threat, Frente Anti-Racista did not
propose a new software; they proposed to create "a screening process for new
members." This is not a technical fix. It is a change in their collective
governance. ComuniDária proposed to "register which member has access to each
page or file." These are actions they can take, together, to make themselves
stronger from the inside. They are empowered.

The mitigations from the \gls{stride} sessions were different. Across both
organizations, the number of actionable mitigations proposed was zero. Nothing.
But even if there were mitigations, the threats \gls{stride} found point to a
different kind of solution. The threats were technical, so the fixes would be
technical—"enable two-factor authentication"—or they would point to a dependency
on others. On a big company. "Depender da segurança da Google" (to depend on
Google's security). This suggests the proposed protocol is much more effective
at helping organizations improve their security through changes in their own
collective processes, which is a core goal for non-hierarchical structures. It
gives them the power.

\subsection{Participant Experience and Usability}
\label{subsec:participant_outcomes}

The evaluation of the participant experience was not done with surveys, but
through direct observation of the workshops.

A clear challenge emerged, and it was common to both methodologies: the initial
step of creating a visual map of the system. Whether it was the formal \gls{dfd}
for \gls{stride} or the system and trust map in our protocol, this was
the most difficult part for the participants. It is an abstract exercise, and it
requires a kind of thinking that can be a barrier before the real discussion
begins.

But it is what happened after this step that the difference became profound.
During the \gls{stride} sessions, the conversation became the property of the few
members with more technical knowledge. The vocabulary of \gls{stride}—spoofing,
tampering, repudiation—created a wall. Many participants became quiet observers.

With the proposed protocol, this wall did not exist. The change was immediate.
The room came alive with discussion. Both ComuniDária and Frente Anti-Racista
interacted much, much more. The reason is simple. Our protocol speaks a human
language. When the discussion is about creating a persona for "the malicious
insider" or protecting an asset like "reputation," everyone is an expert. It is
not a technical debate anymore; it is a collective conversation about their own
organization.

\section*{}
Chapter 5 detailed the structured evaluation conducted to validate the
proposed threat modeling protocol. Through comparative workshops with two
non-hierarchical organizations, the protocol was tested against the
standard \gls{stride} methodology. Furthermore, the evaluation confirmed
better technical results and a significantly more inclusive experience for the participants,
transforming a technical exercise into a collective, strategic conversation.
Chapter 6 will conclude the thesis by summarizing the key findings, discussing
the implications of the research, and suggesting directions for future work.