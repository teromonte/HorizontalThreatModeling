%!TEX root = ../template.tex
%%%%%%%%%%%%%%%%%%%%%%%%%%%%%%%%%%%%%%%%%%%%%%%%%%%%%%%%%%%%%%%%%%%%
%% chapter5.tex
%% NOVA thesis document file
%%%%%%%%%%%%%%%%%%%%%%%%%%%%%%%%%%%%%%%%%%%%%%%%%%%%%%%%%%%%%%%%%%%%

\typeout{NT FILE chapter5.tex}%

\chapter{Conclusão}
\label{cha:conclusion}

\glsresetall

Esta pesquisa propõe um protocolo de modelagem de ameaças adaptado às
especificidades de organizações horizontais, integrando segurança e governança
distribuída como elementos estratégicos. O trabalho parte do reconhecimento de
que estruturas não hierárquicas enfrentam desafios únicos em segurança, desde a
centralização informal de recursos digitais até a vulnerabilidade a ataques que
exploram processos participativos. Ao alinhar a horizontalidade com práticas de
segurança colaborativa, o protocolo busca transformar a descentralização em um
ativo, mitigando pontos críticos de falha sem comprometer a autonomia coletiva.

A análise crítica de metodologias tradicionais, como o \gls{stride} e as árvores
de ataque, revelou lacunas na abordagem de dinâmicas não hierárquicas. Enquanto
esses frameworks são eficazes em contextos centralizados, sua dependência de
hierarquias formais e fluxos de decisão lineares limita sua aplicabilidade em
ambientes distribuídos. Por outro lado, abordagens emergentes, como o
\gls{colbac} e o \gls{abc}, demonstraram potencial para preencher essas lacunas
ao incorporar mecanismos de consenso transparentes e análises econômicas
orientadas a incentivos. Essas soluções inspiraram a estrutura modular do
protocolo proposto, que combina criptografia colaborativa, registros imutáveis e
processos democráticos de autorização.

A principal contribuição deste trabalho reside na integração entre segurança
técnica e governança participativa. O protocolo não apenas identifica ameaças
específicas a organizações horizontais, como manipulação de quóruns, ataques
Sybil e centralização de segredos, mas também estabelece diretrizes para
mitigá-las por meio de mecanismos distribuídos. Por exemplo, a adoção de logs
auditáveis e assinaturas digitais verificáveis reforça a transparência, enquanto
sistemas de votação segura e delegação dinâmica de autoridade preservam a
agilidade decisória. Essa abordagem equilibra eficiência operacional e inclusão,
permitindo que organizações adaptem seu nível de horizontalidade conforme o
contexto. Seja em situações de crise que demandem centralização temporária ou em
operações cotidianas totalmente descentralizadas.

Os critérios de avaliação definidos, eficácia, eficiência, aceitação pelos
usuários e resiliência, fornecem um arcabouço robusto para validar o protocolo em
diferentes cenários. Estudos de caso com cooperativas de trabalhadores e redes
comunitárias permitirão testar sua aplicabilidade prática, enquanto simulações
de ataques democráticos e falhas de consenso avaliarão sua robustez. Espera-se
que os resultados demonstrem como a horizontalidade, quando estruturada de forma
coerente, pode fortalecer a segurança ao distribuir responsabilidades e reduzir
dependências críticas.