%!TEX root = ../template.tex
%%%%%%%%%%%%%%%%%%%%%%%%%%%%%%%%%%%%%%%%%%%%%%%%%%%%%%%%%%%%%%%%%%%%
%% chapter-last.tex
%% NOVA thesis document file
%%%%%%%%%%%%%%%%%%%%%%%%%%%%%%%%%%%%%%%%%%%%%%%%%%%%%%%%%%%%%%%%%%%%

\typeout{NT FILE chapter-last.tex}%

\chapter{Conclusion}
\label{cha:conclusion}

\glsresetall

\section{Summary of Findings}
\label{sec:summary-of-findings}

This research examined an important gap in cybersecurity: the
absence of threat modeling methodologies made to the distinct
socio technical challenges of non-hierarchical organizations. In response, we
proposed a participatory context driven protocol designed explicitly to
align with the core principles of democratic governance, decentralization, and
transparency. It was a new construction made from tools and practices that share
the commitment to inclusivity and collaboration.

The effectiveness of the protocol was not merely theorized but validated through
a series of comparative workshops engaging two distinct organizations. The
results confirmed a clear superiority when compared to the industry standard
STRIDE methodology. Our protocol enabled participants to identify a broader and
more relevant range of threats. Most notably those rooted in social,
governance, and political domains where traditional approaches
demonstrated a significant blindness.

Furthermore, the investigation revealed that the protocol did not just produce a
longer list of risks. It fundamentally guided organizations toward the
development of actionable, internal mitigations. These were solutions that
reinforced their horizontal principles, effectively
empowering the collectives to strengthen their security from within.

\section{Discussion and Implications}
\label{sec:discussion-and-implications}

The findings of this study have significant implications for cybersecurity
theory and its practice. Theoretically, they present a direct
challenge to the presumed universality of conventional threat modeling
frameworks. Such established methods operate on the
presupposition of a command chain that, in these contexts is simply not there,
causing them to fail when faced with the realities of decentralized collectives.
Security is not, and cannot be, a one size fits all discipline; it must be a
reflection of the organizational structure and the specific values it aims to
protect. This work makes clear that for non-hierarchical organizations,
horizontality is not a vulnerability to be mitigated but an essential design
principle for building any meaningful security.

Practically, this thesis delivers something tangible. A protocol that can be
immediately adopted by cooperatives, activist collectives, and other groups.
It answers a real need by translating the often blurred concepts of
security into a accessible process, the protocol empowers
members even those without any specialized expertise to take ownership of
their digital protection. This is not a tool to be imposed by an
outside expert, but a framework to be inhabited and shaped by the collective
itself, and this distinction is crucial for bypassing the common danger of
creating a "digital vanguard" where technical knowledge becomes a new, unelected
form of centralized power.

This enables a truly democratic security practice. It's importance
goes far beyond the technical domain. For any organization that is founded on
deep principles of equity and shared power, the act of employing an
authoritarian, top-down security model introduces a profound contradiction
between its values and its necessary practices. Our protocol offers a pathway. A
way to build resilience that is congruent with their core beliefs in democratic
participation and shared responsibility, ensuring that the very methods used to
protect the organization do not, in the end, corrupt its soul.

\section{Limitations of the Research}
\label{sec:limitations-of-the-research}

Despite the encouraging outcomes this study had several clear
limitations. First, the evaluation was conducted with only two organizations
since the disponibility of members from these organizations was limited.
While they represented different sectors from community service and activism, the
findings may not possess universal generalizability to all forms of
non-hierarchical organizations, such as large-scale worker cooperatives. The specific
contexts are different. The pressures are different.

Second, the evaluation consisted of single session workshops. These provided a
valuable snapshot in time but could not capture the long term dynamics of
security practice and efficacy of the protocol. Whether the
proposed mitigations are successfully implemented, maintained, and adapted over
time was consequently not assessed. This remains an open question.

A third limitation arises from the modest size of the workshop groups. The
protocol is designed to leverage the power of broad participation and diverse
viewpoints; it is conceivable that with a larger, more varied set of
participants, the deliberative process would have yielded even richer outcomes
and uncovered subtler, more complex threats.

Finally, as the primary researcher facilitated all workshop sessions, an element
of facilitator bias may have influenced the discussions and outcomes, despite
conscious efforts to maintain neutrality. The hand that guides the process
inevitably leaves a faint impression upon the result.

\section{Future Work}
\label{sec:future-work}

The limitations of this study do not close a door; instead, they point toward
several compelling paths for future inquiry.

The most crucial next step is to observe the protocol's effects over a much
longer stretch of time. A study that follows a group of organizations not for a
single session, but through seasons of change, for a year or more. This would be
crucial. Only then could we see if the new practices truly take root and
survive the pressures of real-world operation. That is the ultimate test of its
worth.

We must also take this protocol into new environments. Testing it across a large worker
cooperatives, would challenge its assumptions in necessary ways. It would reveal
its breaking points. Every new context is a chance to learn, to harden the
protocol, to make it better.

Furthermore, a clear opportunity exists to build a 'playbook' or a digital
toolkit. A guide to accompany the process. This would reduce
the influence of any single facilitator and make the work radically more
accessible, empowering groups to take this process into their own
hands, on their own terms.

Finally, future work could explore a hybrid construction, one that carefully integrates
the narrative, human-centered strengths of this process with the systematic technical
sharpness of older frameworks. This might be an optional 'deep dive' module. A special
addition for organizations needing a greater level of technical detail, creating
a tool that is both simple at its core and profound in its potential depth.