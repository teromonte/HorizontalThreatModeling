%!TEX root = ../template.tex
%%%%%%%%%%%%%%%%%%%%%%%%%%%%%%%%%%%%%%%%%%%%%%%%%%%%%%%%%%%%%%%%%%%%
%% chapter-last.tex
%% NOVA thesis document file
%%%%%%%%%%%%%%%%%%%%%%%%%%%%%%%%%%%%%%%%%%%%%%%%%%%%%%%%%%%%%%%%%%%%

\typeout{NT FILE chapter-last.tex}%

\chapter{Conclusion}
\label{cha:conclusion}

\glsresetall

\section{Summary of Findings}
\label{sec:summary-of-findings}

This research addressed a critical gap in cybersecurity practice: the lack of
threat modeling methodologies tailored to the unique socio-technical dynamics of
non-hierarchical organizations. We proposed a participatory, context-driven
protocol designed to align with the principles of democratic governance,
decentralization, and transparency.

The protocol's effectiveness was validated through comparative workshops with
two distinct organizations. The results confirmed that, compared to the
industry-standard STRIDE methodology, our protocol enabled participants to
identify a broader, more relevant range of threats, particularly those rooted in
social, governance, and political contexts. Furthermore, it guided organizations
toward developing actionable, internal mitigations that reinforce, rather than
undermine, their horizontal principles.

\section{Discussion and Implications}
\label{sec:discussion-and-implications}

The findings of this study have significant implications for both cybersecurity
theory and practice. Theoretically, they challenge the universality of
conventional, technically-focused threat modeling frameworks. Security is not a
one-size-fits-all discipline; it must be adapted to the organizational structure
and values of the context it aims to protect. This work demonstrates that for
non-hierarchical organizations, treating horizontality as a design principle
rather than a vulnerability is essential.

Practically, this thesis delivers a tangible protocol that can be immediately
adopted by cooperatives, activist collectives, and other decentralized groups.
By translating abstract security concepts into a collaborative and accessible
process, the protocol empowers members without specialized expertise to take
collective ownership of their digital security. It provides a pathway to
building resilience that is congruent with their core values of democratic
participation and shared responsibility.

\section{Limitations of the Research}
\label{sec:limitations-of-the-research}

Despite the positive results, this study has several limitations. First, the
evaluation was conducted with only two organizations. While they represented
different sectors (community service and activism), the findings may not be
generalizable to all forms of non-hierarchical organizations, such as
decentralized autonomous organizations (DAOs) or large-scale worker
cooperatives. Second, the evaluation consisted of single-session workshops. The
long-term efficacy of the protocol—including whether the proposed mitigations
are successfully implemented and maintained—was not assessed. Finally, as the
primary researcher facilitated all workshop sessions, an element of facilitator
bias may have influenced the discussions and outcomes.

\section{Future Work}
\label{sec:future-work}

The limitations of this study pave the way for several avenues of future
research. A longitudinal study, tracking a cohort of organizations over a year
or more, would be invaluable for assessing the long-term impact of the protocol
on their security posture and governance. Expanding the evaluation to include a
wider variety of horizontal organizations, particularly digitally-native ones
like DAOs, would further test the protocol's adaptability.

Additionally, there is an opportunity to develop a 'playbook' or digital toolkit
to accompany the protocol, making it even more accessible for self-facilitation.
Finally, future iterations could explore a hybrid model that integrates the
narrative strengths of this protocol with the systematic technical rigor of
frameworks like STRIDE, offering a 'deep dive' module for organizations with
greater technical capacity.