%!TEX root = ../template.tex
%%%%%%%%%%%%%%%%%%%%%%%%%%%%%%%%%%%%%%%%%%%%%%%%%%%%%%%%%%%%%%%%%%%%
%% chapter-last.tex
%% NOVA thesis document file
%%%%%%%%%%%%%%%%%%%%%%%%%%%%%%%%%%%%%%%%%%%%%%%%%%%%%%%%%%%%%%%%%%%%

\typeout{NT FILE chapter-last.tex}%

\chapter{Conclusion}
\label{cha:conclusion}

\glsresetall

%This research proposes a threat modeling protocol adapted to the specificities
%of horizontal organizations, integrating security and distributed governance as
%strategic elements \cite{Colbac}. The work is based on the recognition that
%non-hierarchical structures face unique challenges in security, from the
%informal centralization of digital resources to vulnerability to attacks that
%exploit participatory processes \cite{EverydayRevolutions,
%MitigationSybilAttack}. By aligning horizontality with collaborative security
%practices, the protocol seeks to transform decentralization into an asset,
%mitigating critical points of failure without compromising collective autonomy.
%
%A critical analysis of traditional methodologies, such as \gls{stride} and
%attack trees, revealed gaps in the approach to non-hierarchical dynamics
%\cite{ThreatModelingdesigningForSecurity, AttackTrees}. While these frameworks
%are effective in centralized contexts, their dependence on formal hierarchies
%and linear decision flows limits their applicability in distributed
%environments. On the other hand, emerging approaches such as \gls{colbac} and
%\gls{abc} have demonstrated potential to fill these gaps by incorporating
%transparent consensus mechanisms and incentive-driven economic analyses
%\cite{Colbac, AbcCrypto}. These solutions inspired the modular structure of the
%proposed protocol, which combines collaborative cryptography, immutable ledgers,
%and democratic authorization processes \cite{Colbac,
%ThreatModelingdesigningForSecurity}.
%
%The main contribution of this work lies in the integration of technical security
%and participatory governance. The protocol not only identifies specific threats
%to horizontal organizations, such as quorum manipulation, Sybil attacks, and
%centralization of secrets, but also establishes guidelines to mitigate them
%through distributed mechanisms \cite{Colbac, MitigationSybilAttack}. For
%example, the adoption of auditable logs and verifiable digital signatures
%reinforces transparency, while secure voting systems and dynamic delegation of
%authority preserve decision-making agility \cite{Colbac}. This approach balances
%operational efficiency and inclusiveness, allowing organizations to adapt their
%level of horizontality according to the context, whether in crisis situations
%that require temporary centralization or in fully decentralized day-to-day
%operations.
%
%The defined evaluation criteria, effectiveness, efficiency, user acceptance and
%resilience, provide a robust framework to validate the protocol in different
%scenarios \cite{RiskCentricThreatModeling}. Case studies with worker
%cooperatives and community networks will allow testing its practical
%applicability, while simulations of democratic attacks and consensus failures
%will assess its robustness \cite{EverydayRevolutions}. The results are expected
%to demonstrate how horizontality, when structured coherently, can strengthen
%security by distributing responsibilities and reducing critical dependencies
%\cite{Colbac, EverydayRevolutions}.
