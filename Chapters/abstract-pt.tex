%!TEX root = ../template.tex
%%%%%%%%%%%%%%%%%%%%%%%%%%%%%%%%%%%%%%%%%%%%%%%%%%%%%%%%%%%%%%%%%%%%
%% abstract-pt.tex
%% NOVA thesis document file
%%
%% Abstract in Portuguese
%%%%%%%%%%%%%%%%%%%%%%%%%%%%%%%%%%%%%%%%%%%%%%%%%%%%%%%%%%%%%%%%%%%%

\typeout{NT FILE abstract-pt.tex}%

Esta tese apresenta e valida um novo protocolo de modelação de ameaças
especificamente concebido para abordar os desafios de segurança das organizações
não hierárquicas. Enquanto as estruturas tradicionais pressupõem o controlo de
cima para baixo, este protocolo assenta na premissa de que a horizontalidade e a
participação democrática podem ser alavancadas como ativos estratégicos para a
construção de resiliência. A pesquisa traduz conceitos abstratos de segurança num processo
acessível e colaborativo que integra princípios de metodologias estabelecidas,
como o STRIDE e o PASTA, com as realidades da governação distribuída.

O protocolo foi testado em workshops reais com duas organizações não
hierarquizadas e a sua eficácia foi comparada com o modelo STRIDE. Os
resultados confirmam que o protocolo proposto se destaca na identificação de
ameaças sociotécnicas e de governação críticas como riscos internos, falhas de
processo e manipulação de quórum que os métodos tradicionais ignoram.
Consequentemente, produz mitigações mais relevantes e acionáveis que capacitam
as organizações para melhorar a sua postura de segurança através de mudanças nos
seus próprios processos coletivos. Este trabalho fornece uma ferramenta tangível
para os grupos descentralizados assumirem a propriedade coletiva da sua
segurança digital, eliminando efetivamente o fosso entre os princípios
democráticos e as práticas robustas de cibersegurança.

% Palavras-chave do resumo em Português
\begin{keywords}
  modelagem de ameaças, organizações horizontais, governança distribuída,
  segurança colaborativa, confiança descentralizada
\end{keywords}
  