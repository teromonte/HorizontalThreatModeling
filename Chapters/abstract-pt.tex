%!TEX root = ../template.tex
%%%%%%%%%%%%%%%%%%%%%%%%%%%%%%%%%%%%%%%%%%%%%%%%%%%%%%%%%%%%%%%%%%%%
%% abstract-pt.tex
%% NOVA thesis document file
%%
%% Abstract in Portuguese
%%%%%%%%%%%%%%%%%%%%%%%%%%%%%%%%%%%%%%%%%%%%%%%%%%%%%%%%%%%%%%%%%%%%

\typeout{NT FILE abstract-pt.tex}%

A pesquisa explora a criação de um protocolo de modelagem de ameaças projetado
especificamente para organizações não-hierárquicas. Com base em análises de
frameworks de governança distribuída e metodologias de segurança, o estudo
desenvolve uma abordagem inovadora que considera a horizontalidade como um ativo
estratégico. Utilizando ferramentas como árvores de ataque, STRIDE, PASTA e
COLBAC, o protocolo busca integrar segurança cibernética e participação
democrática, abordando desafios únicos de estruturas descentralizadas. Este
trabalho contribui para preencher lacunas na literatura ao oferecer diretrizes
práticas para identificar, mitigar e prevenir ameaças em contextos onde a
confiança distribuída e a colaboração são fundamentais.

% Palavras-chave do resumo em Português
\begin{keywords}
  modelagem de ameaças, organizações horizontais, governança distribuída,
  segurança colaborativa, confiança descentralizada
\end{keywords}
  