%!TEX root = ../template.tex
%%%%%%%%%%%%%%%%%%%%%%%%%%%%%%%%%%%%%%%%%%%%%%%%%%%%%%%%%%%%%%%%%%%%
%% abstract-en.tex
%% NOVA thesis document file
%%
%% Abstract in English([^%]*)
%%%%%%%%%%%%%%%%%%%%%%%%%%%%%%%%%%%%%%%%%%%%%%%%%%%%%%%%%%%%%%%%%%%%

\typeout{NT FILE abstract-en.tex}%

This thesis presents and validates a novel threat modeling protocol specifically
designed to address the security challenges of non-hierarchical organizations.
While traditional structures assume top-down control, this protocol is based on
the premise that horizontality and democratic participation can be leveraged as
strategic assets for building resilience. It translates abstract security
concepts into an accessible and collaborative process that integrates principles
from established methodologies, such as STRIDE and PASTA, with the realities of
distributed governance.

The protocol was tested in real-world workshops with two non-hierarchical
organizations, and its effectiveness was compared with the STRIDE framework. The
results confirm that the proposed protocol excels at identifying critical
sociotechnical and governance threats—such as insider risks, process failures,
and quorum manipulation—that traditional methods miss. Consequently, it produces
more relevant and actionable mitigations that empower organizations to improve
their security posture through changes to their own collective processes. This
work provides a tangible tool for decentralized groups to take collective
ownership of their digital security, effectively bridging the gap between
democratic principles and robust cybersecurity practices.

% Keywords of the abstract in English
\begin{keywords}
  threat modeling, horizontal organizations, distributed governance, collaborative security, decentralized trust
\end{keywords}
  