%!TEX root = ../template.tex
%%%%%%%%%%%%%%%%%%%%%%%%%%%%%%%%%%%%%%%%%%%%%%%%%%%%%%%%%%%%%%%%%%%
%% chapter1.tex
%% NOVA thesis document file
%%
%% Chapter with introduction
%%%%%%%%%%%%%%%%%%%%%%%%%%%%%%%%%%%%%%%%%%%%%%%%%%%%%%%%%%%%%%%%%%%

\typeout{NT FILE chapter1.tex}%

\chapter{Introduction}
\label{cha:introduction}

\prependtographicspath{{Chapters/Figures/Covers/}}

% epigraph configuration
% \epigraphfontsize{\small\itshape}
% \setlength\epigraphwidth{12.5cm}
% \setlength\epigraphrule{0pt}
% 
% \epigraph{
% 
% }

\section{Context}
\label{sec:context}

A segurança cibernética moderna frequentemente pressupõe a existência de estruturas hierárquicas,
o que pode não ser adequado para organizações que operam de maneira horizontal, como cooperativas
de trabalhadores, sindicatos, organizações ativistas, etc.
Essas organizações, que evitam deliberadamente estruturas hierárquicas, enfrentam desafios únicos
em termos de segurança, pois as soluções tradicionais não consideram a horizontalidade de todo.
Este projeto visa desenvolver protocolos de modelagem de ameaças que valorizem a horizontalidade,
especificamente para as organizações mencionadas.

Organizações hierárquicas são estruturadas de forma que o poder e a tomada de decisões fluem de
cima para baixo. Em uma organização hierárquica típica, há uma cadeia de comando clara, onde
cada nível da hierarquia tem autoridade sobre o nível abaixo. Por exemplo, em uma empresa
tradicional, o CEO toma decisões estratégicas que são implementadas por gerentes de nível médio
e, finalmente, executadas por funcionários de nível operacional. Este modelo facilita a tomada
de decisões rápidas e a implementação de políticas de segurança, pois há uma clara atribuição
de responsabilidades e controle \cite{Colbac}.

Em contraste, organizações não-hierárquicas, ou horizontais, distribuem o poder de forma mais
equitativa entre seus membros. Nessas organizações, a tomada de decisões é frequentemente feita
de forma coletiva, através de processos democráticos e participativos. Por exemplo, em uma
cooperativa de trabalhadores, todos os membros podem ter uma voz igual nas decisões importantes,
e não há uma cadeia de comando rígida. Isso pode levar a uma maior transparência e inclusão, mas
também pode criar desafios únicos em termos de segurança, como a dificuldade em gerenciar o acesso
a informações sensíveis sem criar uma hierarquia implícita \cite{Colbac}.

Um dos principais desafios de segurança em organizações horizontais é a gestão
de segredos, como senhas e chaves de criptografia. Em uma organização
hierárquica, esses segredos são frequentemente controlados por um pequeno grupo de
administradores que têm autoridade para gerenciar o acesso. No entanto, em uma organização
horizontal, decidir quem deve ter acesso a esses segredos pode ser mais complicado. Se
todos os membros tiverem acesso, há um risco maior de abuso ou erro humano. Por
outro lado, restringir o acesso a um pequeno grupo pode criar uma hierarquia de
fato, minando os princípios de horizontalidade \cite{Colbac}. 

Ao desenvolver protocolos de modelagem de ameaças para organizações
horizontais, é crucial considerar a horizontalidade como um ativo. Isso significa criar
sistemas de segurança que não apenas protejam contra ameaças externas, mas que também
respeitem e reforcem a estrutura participativa da organização. Por exemplo, em vez de
confiar em um único administrador para gerenciar o acesso a recursos críticos, pode-se
implementar um sistema de autorização coletiva, onde múltiplos membros devem concordar
antes que o acesso seja concedido. Isso não só distribui a responsabilidade, mas
também aumenta a resiliência contra ataques internos, pois um único indivíduo não
pode comprometer o sistema \cite{Colbac}.

Além disso, a horizontalidade pode ser usada para
melhorar a detecção e resposta a incidentes de segurança. Em uma organização
hierárquica, a detecção de ameaças pode ser centralizada, o que pode levar a pontos únicos
de falha. Em uma organização horizontal, a detecção de ameaças pode ser
distribuída entre todos os membros, aumentando a probabilidade de que atividades suspeitas
sejam rapidamente identificadas e respondidas \cite{AbcCrypto}.

Ao considerar a horizontalidade como um ativo na modelagem de ameaças, podemos
desenvolver protocolos de segurança que não apenas protejam as organizações horizontais,
mas que também reforcem seus princípios fundamentais de participação e igualdade. Este
projeto busca explorar essas possibilidades e criar um novo paradigma de segurança
cibernética que seja verdadeiramente adequado para organizações não-hierárquicas \cite{MicrosoftBible}.

\section{Your Time is Precious}
\label{sub:time_is_money}


