%!TEX root = ../template.tex
%%%%%%%%%%%%%%%%%%%%%%%%%%%%%%%%%%%%%%%%%%%%%%%%%%%%%%%%%%%%%%%%%%%
%% chapter1.tex
%% NOVA thesis document file
%%
%% Chapter with introduction
%%%%%%%%%%%%%%%%%%%%%%%%%%%%%%%%%%%%%%%%%%%%%%%%%%%%%%%%%%%%%%%%%%%

\typeout{NT FILE chapter1.tex}%

\prependtographicspath{{Chapters/Figures/Covers/}}

% epigraph configuration
% \epigraphfontsize{\small\itshape}
% \setlength\epigraphwidth{12.5cm}
% \setlength\epigraphrule{0pt}
% 
% \epigraph{
% 
% }

\chapter{Introduction}
\label{cha:Introduction}

\section{Governança Organizacional: Uma Perspectiva
Histórica} \label{sec:contexto_historico}

A governança organizacional reflete as estruturas sociais,
econômicas e tecnológicas vigentes em cada época. Desde os
primeiros agrupamentos humanos até as organizações complexas
da contemporaneidade, as formas de organizar o poder e a
tomada de decisão foram moldadas para responder a contextos
específicos. O modelo hierárquico, adotado amplamente,
emergiu como solução para demandas de controle e eficiência,
mas a história também registra experimentos que desafiaram
essa lógica.

Mesmo em sistemas participativos históricos, como a
democracia ateniense, a horizontalidade enfrentou limitações
de inclusão e praticidade \cite{AthenianDemocracyABrief}.
Durante a Revolução Industrial, a centralização hierárquica
intensificou-se para lidar com o crescimento e a
complexidade organizacional. Por outro lado, cooperativas e
movimentos sindicalistas do século XIX começaram a esboçar
alternativas à centralização absoluta, apontando novos
caminhos para a governança
\cite{WorkerCooperativesinAmerica, EverydayRevolutions}.

No século XX, os avanços tecnológicos e as novas teorias
organizacionais puseram em xeque a inevitabilidade das
hierarquias rígidas. Nesse contexto, cooperativas modernas e
redes descentralizadas demonstraram a possibilidade de
alinhar eficiência a valores igualitários
\cite{WorkerCooperativesandRevolution}. Ao mesmo tempo,
tecnologias de vigilância em massa tendem a reforçar
estruturas centralizadoras, enquanto ferramentas como o
blockchain catalisam processos de descentralização,
ampliando o alcance de modelos horizontais
\cite{DoArtifactsHavePolitics}. Essa evolução histórica
lança as bases para entender, a seguir, os desafios
contemporâneos de segurança em estruturas não-hierárquicas.

\section{A Segurança Horizontal em Tempos de Interconexão}
\label{sec:desafios_contemporaneos}

No mundo interconectado atual, organizações horizontais
enfrentam desafios de segurança específicos. A ausência de
hierarquia formal dificulta ataques centralizados, mas exige
uma redefinição da gestão da confiança. Em sistemas de
confiança distribuída, como os utilizados em organizações
descentralizadas baseadas em blockchain, a segurança depende
de mecanismos que substituem líderes formais por processos
participativos e algoritmos orientados à transparência e ao
consenso \cite{Reputation-basedDAO}.

Modelos tradicionais de segurança, como STRIDE e árvores de
ataque, fornecem bases sólidas para análise de ameaças, mas
nem sempre se adaptam à complexidade das estruturas
descentralizadas \cite{ThreatModellingSurvey}. Por outro
lado, tecnologias como a criptografia colaborativa
\cite{Colbac, AbcCrypto} permitem que múltiplos membros
contribuam para a proteção de dados sem comprometer a
horizontalidade.

Essas abordagens já demonstram um avanço significativo: a
segurança pode ser concebida de forma coerente com
princípios democráticos, preservando a horizontalidade.
Compreender tais desafios é o primeiro passo para a
proposição de um protocolo de segurança específico, que será
discutido no próximo capítulo.

\section{Protocolo de Segurança para Organizações
Não-Hierárquicas} \label{sec:objetivos_pesquisa}

Esta pesquisa propõe um protocolo de segurança que não
apenas preserva a horizontalidade, mas a converte em um
diferencial estratégico. Ao invés de considerar a ausência
de hierarquia como vulnerabilidade, buscou-se demonstrar que
ela pode fornecer resiliência diante de cenários adversos.
Tal esforço preenche uma lacuna na literatura sobre
segurança em estruturas horizontalizadas, oferecendo
diretrizes práticas para organizações comprometidas com a
governança democrática.

O protocolo fundamenta-se em valores participativos, na
análise de casos reais de cooperativas e redes comunitárias,
e em abordagens metodológicas orientadas à eficiência e à
transparência. Por exemplo, imagine uma cooperativa de
trabalhadores que gerencia recursos digitais sensíveis: como
assegurar a proteção contra fraudes internas e ataques
externos sem recorrer a estruturas autoritárias? O protocolo
aqui delineado oferece mecanismos de consenso,
transparência, criptografia colaborativa e modelagem de
ameaças adaptada, respondendo a esse tipo de questionamento.

Assim, ao alinhar eficiência técnica, participação coletiva
e acesso igualitário à informação, o protocolo se diferencia
de abordagens tradicionais. Antes de detalhar seus
elementos, é necessário delimitar o escopo da pesquisa para
garantir precisão analítica.

\section{Delimitando o Escopo da Pesquisa}
\label{sec:delimitacao_escopo}

A variedade de arranjos organizacionais horizontais é ampla.
Para uma análise mais profunda, esta pesquisa concentra-se
em estruturas plenamente horizontais que operam sob
princípios de confiança distribuída e governança
democrática, como cooperativas de trabalhadores e redes
comunitárias \cite{WorkerCooperativesinAmerica,
EverydayRevolutions}. Ao excluir organizações híbridas,
busca-se compreender o cerne da relação entre
horizontalidade e segurança, esclarecendo a eficácia do
protocolo proposto em cenários puros. Futuras investigações
poderão, então, integrar elementos híbridos, ampliando a
aplicabilidade do conhecimento gerado.

\section{Contribuições Esperadas}
\label{sec:contribuicoes_esperadas}

Esta pesquisa espera avançar a compreensão teórica da
segurança em estruturas horizontais, abordando lacunas
evidenciadas por
\cite{ThreatModelingAsABasisForSecurityRequirements} e
\cite{DemystifyingTheThreatModelingProcess}, que ressaltam a
necessidade de ferramentas específicas para tais contextos.
No plano prático, espera-se oferecer um conjunto de
recomendações que combine eficiência, participação coletiva
e transparência na proteção de dados e processos.

Ao ampliar o repertório de soluções, a segurança não se
torna um obstáculo, mas uma oportunidade de aprofundar a
democratização organizacional. Assim, o protocolo contribui
para o desenvolvimento de organizações em que a
horizontalidade não é meramente um ideal normativo, mas uma
estratégia efetiva contra ameaças complexas.

\section{Estrutura da Tese} \label{sec:estrutura_tese}

Após esta introdução, o capítulo de fundamentação teórica
definirá conceitos centrais, como modelagem de ameaças,
governança horizontal e confiança distribuída. Em seguida, o
capítulo de trabalhos relacionados analisará estudos
pré-existentes, como \cite{Colbac, AbcCrypto}, situando o
protocolo proposto no debate acadêmico. O capítulo de design
detalhará o protocolo, seus componentes e os métodos de
avaliação. Por fim, as conclusões sintetizarão os achados e
sugerirão direções para pesquisas futuras, reforçando o
papel da horizontalidade como aliada da segurança em um
mundo cada vez mais interconectado.
