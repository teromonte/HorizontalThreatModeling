%!TEX root = ../template.tex
%%%%%%%%%%%%%%%%%%%%%%%%%%%%%%%%%%%%%%%%%%%%%%%%%%%%%%%%%%%%%%%%%%%
%% chapter1.tex
%% NOVA thesis document file
%%
%% Chapter with introduction
%%%%%%%%%%%%%%%%%%%%%%%%%%%%%%%%%%%%%%%%%%%%%%%%%%%%%%%%%%%%%%%%%%%

\typeout{NT FILE chapter1.tex}%

\chapter{Introduction}
\label{cha:introduction}

\prependtographicspath{{Chapters/Figures/Covers/}}

% epigraph configuration
% \epigraphfontsize{\small\itshape}
% \setlength\epigraphwidth{12.5cm}
% \setlength\epigraphrule{0pt}
% 
% \epigraph{
% 
% }

\section{Context}
\label{sec:context}

A segurança cibernética moderna frequentemente pressupõe a existência de estruturas hierárquicas,
o que pode não ser adequado para organizações que operam de maneira horizontal, como cooperativas
de trabalhadores, sindicatos, organizações ativistas, etc. Essas organizações, que evitam
deliberadamente estruturas hierárquicas, enfrentam desafios únicos em termos de segurança, pois
as soluções tradicionais não consideram a horizontalidade de todo.

Organizações hierárquicas são estruturadas de forma que o poder e a tomada de decisões fluem de
cima para baixo. Em uma organização hierárquica típica, há uma cadeia de comando clara, onde
cada nível da hierarquia tem autoridade sobre o nível abaixo. Por exemplo, em uma empresa
tradicional, o CEO toma decisões estratégicas que são implementadas por gerentes de nível médio
e, finalmente, executadas por funcionários de nível operacional. Este modelo facilita a tomada
de decisões rápidas e a implementação de políticas de segurança, pois há uma clara atribuição
de responsabilidades e controle.

Em contraste, organizações não-hierárquicas, ou horizontais, distribuem o poder de forma mais
equitativa entre seus membros. Nessas organizações, a tomada de decisões é frequentemente feita
de forma coletiva, através de processos democráticos e participativos. Por exemplo, em uma
cooperativa de trabalhadores, todos os membros podem ter uma voz igual nas decisões importantes,
e não há uma cadeia de comando rígida. Isso pode levar a uma maior transparência e inclusão, mas
também pode criar desafios únicos em termos de segurança, como a dificuldade em gerenciar o acesso
a informações sensíveis sem criar uma hierarquia implícita.

Um dos principais desafios de segurança em organizações horizontais é a gestão
de segredos, como senhas e chaves de criptografia. Em uma organização
hierárquica, esses segredos são frequentemente controlados por um pequeno grupo de
administradores que têm autoridade para gerenciar o acesso. No entanto, em uma organização
horizontal, decidir quem deve ter acesso a esses segredos pode ser mais complicado. Se
todos os membros tiverem acesso, há um risco maior de abuso ou erro humano. Por
outro lado, restringir o acesso a um pequeno grupo pode criar uma hierarquia de
fato, minando os princípios de horizontalidade.

Ao desenvolver protocolos de modelagem de ameaças para organizações
horizontais, é crucial considerar a horizontalidade como um ativo. Isso significa criar
sistemas de segurança que não apenas protejam contra ameaças externas, mas que também
respeitem e reforcem a estrutura participativa da organização.

Ao considerar a horizontalidade como um ativo na modelagem de ameaças, podemos
desenvolver protocolos de segurança que não apenas protejam as organizações horizontais,
mas que também reforcem seus princípios fundamentais de participação e igualdade.

\section{Objective}
\label{sec:objective}

O objetivo principal desta tese é desenvolver um protocolo de modelagem de
ameaças especificamente adaptado para organizações não-hierárquicas, como cooperativas
de trabalhadores, sindicatos, grupos ativistas e projetos de software de código
aberto. Este protocolo visa abordar os desafios únicos enfrentados por essas
organizações em termos de segurança cibernética, considerando a horizontalidade como um
ativo e não como uma limitação. 

O desenvolvimento de um protocolo de modelagem de ameaças que leve em
consideração a estrutura participativa e democrática das organizações não-hierárquicas é
essencial para garantir que as soluções de segurança não comprometam os princípios de
igualdade e participação. A avaliação da eficácia do protocolo desenvolvido será
realizada com membros de diferentes organizações que exibem variados níveis de
horizontalidade, analisando como o protocolo se adapta a diferentes contextos e necessidades
específicas de cada tipo de organização. 

Além disso, será feita uma comparação entre o novo protocolo e os modelos de
ameaças tradicionais, como STRIDE e DREAD, destacando as vantagens e desvantagens de
cada abordagem em contextos horizontais. A identificação dos principais desafios de
segurança enfrentados por organizações não-hierárquicas e a proposição de soluções
que respeitem e reforcem a estrutura participativa dessas organizações são passos
cruciais para o sucesso do protocolo.

Por fim, a documentação de casos de uso reais onde o protocolo foi
implementado fornecerá exemplos práticos de como ele pode ser aplicado e os resultados
obtidos.


\section{Contributions}
\label{sec:contributions}

\section{Structure}
\label{sec:structure}

O primeiro capítulo, Introdução, apresenta o contexto e a motivação do estudo,
destacando a necessidade de um protocolo específico para organizações
horizontais e delineando os objetivos da pesquisa. O segundo
capítulo, Background and Related Work, revisa a literatura existente
sobre modelagem de ameaças e as características das organizações não-hierárquicas,
fornecendo uma base teórica para o desenvolvimento do protocolo. O terceiro capítulo,
Research Design, descreve a metodologia utilizada na pesquisa, incluindo os métodos de
coleta e análise de dados, bem como os critérios para a avaliação do protocolo. O
quarto capítulo, Conclusion, sintetiza os principais achados da pesquisa, discutindo
as implicações dos resultados e sugerindo direções para trabalhos futuros.
Finalmente, o capítulo Work Plan detalha o cronograma e as etapas previstas para a
implementação e validação do protocolo, garantindo uma abordagem estruturada e sistemática
ao longo do desenvolvimento da tese
