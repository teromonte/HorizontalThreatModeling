%!TEX root = ../template.tex
%%%%%%%%%%%%%%%%%%%%%%%%%%%%%%%%%%%%%%%%%%%%%%%%%%%%%%%%%%%%%%%%%%%
%% chapter1.tex
%% NOVA thesis document file
%%%%%%%%%%%%%%%%%%%%%%%%%%%%%%%%%%%%%%%%%%%%%%%%%%%%%%%%%%%%%%%%%%%

%\epigraphfontsize{\small\itshape}
%\setlength\epigraphwidth{12.5cm}
%\setlength\epigraphrule{0pt}
%
%\epigraph{
%    "The very soldiers who had no courage in the White army yesterday are
%    very brave in the Red Army today; such is the effect of democracy."
%} {Mao Tsé-Tung}

\typeout{NT FILE chapter1.tex}%

\chapter{Introdução}
\label{cha:Introduction}

\section{Modelagem de Ameaças: Relevância e Desafios}
\label{sec:modelagem_ameacas}

A modelagem de ameaças é uma disciplina essencial no âmbito da segurança da
informação, cuja principal função é identificar, classificar e mitigar
vulnerabilidades em sistemas tecnológicos antes que estas possam ser exploradas
por adversários. Em um contexto onde os sistemas se tornam cada vez mais
complexos e integrados, a modelagem de ameaças se destaca como uma ferramenta
crítica para antecipar riscos e estabelecer medidas de segurança eficazes
\cite{ThreatModellingSurvey}.

As organizações hierárquicas, devido à centralização do controle e de recursos,
geralmente aplicam metodologias de modelagem de ameaças que dependem de fluxos
de decisão lineares e centralizados, como é o caso da metodologia STRIDE,
desenvolvida pela Microsoft \cite{ThreatModelingASummaryOfAvailableMethods}.
Esse modelo centralizado reflete uma estrutura organizacional que facilita a
implementação de políticas de segurança homogêneas, mas carece de flexibilidade
para lidar com as dinâmicas horizontais
\cite{EvaluationofCompetingThreatModeling}.

Em contrapartida, estruturas organizacionais horizontais, caracterizadas por uma
distribuição equitativa de poder e responsabilidades, apresentam desafios
específicos para a modelagem de ameaças \cite{Colbac}. A ausência de uma
hierarquia clara muitas vezes resulta em incompatibilidades estruturais, como
sistemas de controle de acesso que forçam a hierarquização, dificultando a
implementação de políticas democraticamente acordadas \cite{Colbac}.
Adicionalmente, a centralização de segredos organizacionais, como senhas, pode
gerar conflitos durante transições de liderança, o que é exemplificado no
fenômeno conhecido como pasword wars \cite{FromCounterpublicstoContentious}.

A utilização de ferramentas digitais por movimentos sociais também revela um
fenômeno conhecido como vanguard digital. Isso ocorre quando indivíduos que
controlam plataformas e contas digitais assumem posições de poder que podem
contrariar os princípios de igualdade estrutural
\cite{SocialMediaTeamsAsDigitalVanguards}. Além disso, ataques a processos
democráticos, como falsificação de identidades (Sybil attacks) e manipulação de
quorum, destacam vulnerabilidades exclusivas de sistemas horizontais devido à
dependência em processos participativos \cite{MitigationSybilAttack,
TheSybilAttack}. Finalmente, a falta de ferramentas tecnológicas que acomodem
transições entre centralizações temporárias e estruturas horizontais limita a
eficiência em cenários que demandam respostas rápidas ou especialização
\cite{Colbac}.

Esses desafios evidenciam a necessidade de adaptações nos métodos de segurança e
na modelagem de ameaças para que possam atender às especificidades das
organizações horizontais. A modelagem de ameaças para esses contextos deve
equilibrar aspectos técnicos e dinâmicas organizacionais, promovendo a
resiliência e a segurança em ambientes onde o poder e a responsabilidade são
compartilhados \cite{ThreatModelingdesigningForSecurity}.

\section{A Segurança Horizontal em Tempos de Interconexão}
\label{sec:desafios_contemporaneos}

No mundo interconectado atual, as organizações horizontais desafiam o
pressuposto de que a segurança depende de uma cadeia clara de comando. A
ausência de hierarquia formal pode se transformar em um ativo estratégico
ao dificultar ataques centralizados e ao permitir uma reconfiguração da
gestão da confiança, promovendo a resiliência organizacional
\cite{EverydayRevolutions}. Em sistemas de confiança distribuída,
como os utilizados em organizações baseadas em blockchain,
a segurança é promovida por mecanismos colaborativos que
substituem líderes formais por processos participativos e soluções
orientadas à transparência e consenso \cite{Reputation-basedDAO}.

Metodologias tradicionais de análise de ameaças, tais como \gls{stride} e árvores
de ataque, fornecem fundamentos valiosos, mas enfrentam limitações em
ambientes descentralizados, destacando a necessidade de abordagens mais
adequadas às especificidades de estruturas horizontais
\cite{ThreatModellingSurvey}. Contextos menos hierárquicos requerem
abordagens adaptadas que compreendam a complexidade da confiança horizontal
e dos potenciais riscos associados.

Nesse sentido, tecnologias como a criptografia colaborativa \cite{Colbac,
AbcCrypto} e abordagens de modelagem de ameaças que adotam a perspectiva global
da organização podem promover um entendimento mais realista da segurança em
estruturas descentralizadas. A horizontalidade, frequentemente vista como
um desafio, deve ser explorada como um ativo estratégico capaz de diluir
pontos únicos de falha e fortalecer a resiliência organizacional.

\section{Governança Organizacional: Uma Perspectiva Histórica}
\label{sec:contexto_historico}

A governança organizacional reflete as estruturas sociais, econômicas e
tecnológicas de cada época. Desde os primeiros agrupamentos humanos até as
organizações complexas da contemporaneidade, as formas de organizar o poder
e a tomada de decisão foram moldadas para responder a contextos
específicos. O modelo hierárquico, amplamente adotado, emergiu como solução
para demandas de controle e eficiência. Contudo, a história também registra
experimentos que desafiaram essa lógica, sugerindo a possibilidade de novas
abordagens na gestão e coordenação de atividades.

Mesmo em sistemas considerados pioneiros na horizontalidade, como a
democracia ateniense, a governança enfrentou limitações significativas
relacionadas à inclusão e à aplicabilidade prática, evidenciando
fragilidades na operacionalização da participação igualitária
\cite{AthenianDemocracyABrief}. Com o avanço da Revolução Industrial, a
centralização hierárquica intensificou-se para lidar com o crescimento e a
complexidade organizacional \cite{WorkerCooperativesandRevolution}.
Adicionalmente, experiências como as cooperativas e os movimentos
sindicalistas do século XIX delinearam alternativas à centralização absoluta,
enquanto tecnologias modernas, oferecendo estruturas descentralizadas que desafiam paradigmas
tradicionais de controle \cite{WorkerCooperativesinAmerica, EverydayRevolutions}.

Enquanto tecnologias de vigilância em massa reforçam estruturas
centralizadoras, bloqueando a adoção plena de governança horizontal,
inovações como o blockchain abrem novas possibilidades de descentralização,
ainda que enfrentem desafios na distribuição equitativa de poder e
recursos, como evidenciado na concentração de mineradores em redes públicas
\cite{DoArtifactsHavePolitics}.

Essas evoluções históricas e tecnológicas não apenas moldam as estruturas
de governança, mas também introduzem desafios únicos na modelagem de
ameaças. A análise crítica dessas tentativas permite identificar
vulnerabilidades e forças que fundamentam a construção de protocolos de
segurança em organizações horizontais.

\section{Protocolo de Segurança para Organizações Não-Hierárquicas}
\label{sec:objetivos_pesquisa}

Esta pesquisa propõe um protocolo de segurança que integra a horizontalidade
como elemento estratégico, indo além da simples adaptação de metodologias
tradicionais. O objetivo central é demonstrar como a descentralização, quando
estruturada de forma coerente com os princípios organizacionais, pode reforçar a
resiliência frente a ameaças complexas, mitigando pontos únicos de falha e
distribuindo responsabilidades de maneira equitativa. O protocolo visa
equilibrar eficiência operacional e participação democrática, garantindo que
medidas de segurança não comprometam a agilidade decisória nem a inclusão de
membros em processos críticos. Para isso, baseia-se em abordagens colaborativas,
como a criptografia adaptada a contextos horizontais \cite{Colbac}, e em
metodologias de modelagem de ameaças que consideram dinâmicas participativas
\cite{ThreatModelingdesigningForSecurity}.

A integração entre segurança e governança é abordada por meio de diretrizes que
harmonizam requisitos técnicos com princípios organizacionais. O protocolo prevê
a aplicação de mecanismos de consenso transparentes e auditáveis, inspirados em
modelos de reputação distribuída \cite{Reputation-basedDAO}, para validar
políticas de acesso e mitigar riscos como ataques Sybil
\cite{MitigationSybilAttack}. Além disso, incorpora estruturas modulares que
permitem adaptação a diferentes níveis de horizontalidade, desde redes
totalmente descentralizadas até organizações com estruturas mais hierarquizadas
\cite{Colbac}. Essa flexibilidade é essencial para responder a ameaças
dinâmicas sem comprometer a autonomia dos membros e abranger o maior número de
organizações.

Para fortalecer a resiliência, o protocolo combina camadas técnicas e sociais:
técnicas criptográficas protegem contra ameaças externas, enquanto
estruturas de transparência radical e revisões periódicas por comitês
rotativos previnem fraudes internas. A rastreabilidade de decisões via
registros imutáveis assegura que vulnerabilidades sejam
identificadas e corrigidas de forma colaborativa, alinhando-se a estudos sobre
falhas em sistemas distribuídos \cite{Reputation-basedDAO}. Ao incorporar lições
de casos históricos e inovações metodológicas, o protocolo oferece um arcabouço
prático para organizações que almejam segurança sem abrir mão de sua identidade
horizontal, preparando o terreno para análises detalhadas nos capítulos
subsequentes.


\section{Delimitando o Escopo da Pesquisa}
\label{sec:delimitacao_escopo}

A diversidade de organizações horizontais abrange desde coletivos informais até
redes digitais complexas, cada uma com dinâmicas particulares. Para garantir
foco analítico, este estudo limita-se a estruturas que operam sob princípios
estritos de horizontalidade, caracterizadas por: (1) ausência de hierarquias
formais ou centralização permanente de poder; (2) processos decisórios baseados
em consenso ou participação ampla; e (3) mecanismos explícitos de distribuição
de responsabilidades e recursos \cite{Colbac}. Essa delimitação exclui modelos
híbridos ou parcialmente descentralizados, onde a coexistência de estruturas
hierárquicas e horizontais introduz variáveis adicionais que dificultam a
avaliação isolada do protocolo proposto.

A opção por analisar organizações como cooperativas de trabalhadores e redes
comunitárias justifica-se por sua relevância empírica: esses modelos possuem
documentação robusta sobre desafios operacionais
\cite{WorkerCooperativesinAmerica}, além de adotarem explicitamente princípios
de autogestão e transparência radical \cite{EverydayRevolutions}. Tais
características permitem testar o protocolo em contextos onde a segurança
depende diretamente da coordenação coletiva, sem intermediários ou autoridades
centrais.

Embora plataformas digitais e redes sociais descentralizadas
\cite{CreatingTheCollectiveSocialMedia} representem casos igualmente relevantes,
sua natureza dinâmica e dependência de infraestruturas técnicas heterogêneas
exigiriam adaptações metodológicas além do escopo atual. Estudos futuros poderão
explorar essas variações, utilizando o protocolo aqui desenvolvido como base
para análises comparativas em ambientes menos controlados.


\section{Contribuições Esperadas} 
\label{sec:contribuicoes_esperadas}

A pesquisa entregará um protocolo de modelagem de ameaças específico para
organizações horizontais, integrando os princípios de governança distribuída,
participação coletiva e transparência. Esse protocolo será projetado para
identificar, avaliar e mitigar ameaças em estruturas descentralizadas,
fornecendo diretrizes práticas adaptadas às particularidades dessas
organizações. Além disso, será acompanhado por um método de avaliação para
validar sua eficácia e sua aplicação em casos reais.

Espera-se que o protocolo demonstre como a horizontalidade pode ser um ativo
estratégico, reforçando a segurança e a resiliência organizacional frente a
ameaças complexas. Além de contribuir para o avanço teórico sobre segurança em
estruturas descentralizadas, a pesquisa busca oferecer uma solução prática que
fortaleça a autonomia e promova a integração entre segurança e governança
democrática.

\section{Estrutura da Tese} 
\label{sec:estrutura_tese}

Após a introdução, o capítulo de Background apresenta os fundamentos
da modelagem de ameaças e segurança em estruturas horizontais. O
capítulo de Related Work analisa estudos prévios, identificando
lacunas e oportunidades relevantes.

O capítulo de Design esboça o protocolo proposto, suas metodologias e
critérios de avaliação. O capítulo de Work Plan descreve as etapas e
cronograma da pesquisa, assegurando a execução estruturada e viável do
projeto. Por fim, o capítulo de Conclusão sintetiza os resultados,
discute limitações e propõe direções futuras, destacando a
horizontalidade como ativo estratégico para a segurança
organizacional.