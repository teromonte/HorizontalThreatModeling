%!TEX root = ../template.tex
%%%%%%%%%%%%%%%%%%%%%%%%%%%%%%%%%%%%%%%%%%%%%%%%%%%%%%%%%%%%%%%%%%%
%% chapter1.tex
%% NOVA thesis document file
%%
%% Chapter with introduction
%%%%%%%%%%%%%%%%%%%%%%%%%%%%%%%%%%%%%%%%%%%%%%%%%%%%%%%%%%%%%%%%%%%

\typeout{NT FILE chapter1.tex}%

\prependtographicspath{{Chapters/Figures/Covers/}}

% epigraph configuration
% \epigraphfontsize{\small\itshape}
% \setlength\epigraphwidth{12.5cm}
% \setlength\epigraphrule{0pt}
% 
% \epigraph{
% 
% }

\chapter{Introduction}
\label{cha:introduction}

\section{Contexto Histórico da Governança Organizacional}
\label{sec:contexto-historico}

A governança organizacional tem sido moldada ao longo da
história por diferentes formas de organização social,
econômica e política. Desde as democracias atenienses,
descritas como um exemplo inicial de decisão coletiva e
distribuição equitativa de poder, até modelos hierárquicos
mais complexos que surgiram com a industrialização,
observa-se uma evolução significativa na maneira como as
pessoas administram recursos e tomam decisões coletivas. A
experiência ateniense, destacada no documento
\cite{AthenianDemocracyABrief}, oferece uma perspectiva
única de como a inclusão cidadã e a participação direta
podem funcionar como alternativa ao controle centralizado,
embora com limitações estruturais e sociais evidentes à
época.

No entanto, modelos como o centralismo democrático, que
evoluíram em regimes socialistas no século XX, apresentam
uma tensão entre a hierarquia e a participação coletiva.
Conforme discutido em \cite{StillaCenturyoftheChineseModel},
esses sistemas tentaram conciliar o controle centralizado
com mecanismos de consulta popular, mas muitas vezes
sacrificaram a horizontalidade em prol da eficiência.

Hoje, é essencial compreender as origens desses sistemas não
apenas para criticar, mas para extrair lições que possam
informar estruturas descentralizadas e horizontais, como as
exploradas neste estudo.

\section{Desafios Contemporâneos na Segurança Horizontal}
\label{sec:desafios-contemporaneos}

Com o advento das tecnologias digitais e das organizações em
rede, surgiram novos desafios para a segurança em ambientes
horizontais. Diferentemente de estruturas hierárquicas, em
que o controle é concentrado, as organizações horizontais
enfrentam dificuldades para coordenar respostas às ameaças
sem comprometer sua estrutura descentralizada. De acordo com
\cite{ThreatModelingASummaryOfAvailableMethods} e
\cite{ParticipatoryThreatModelling}, os métodos tradicionais
de modelagem de ameaças frequentemente subestimam os riscos
associados à falta de um comando centralizado, como
comunicação ineficaz e desconfiança entre membros.

Em paralelo, o crescimento de tecnologias de vigilância e
controle exacerba essas vulnerabilidades.
\cite{DoArtifactsHavePolitics} argumenta que as ferramentas
tecnológicas não são neutras, muitas vezes refletindo as
dinâmicas de poder de seus criadores. Essa perspectiva
destaca a necessidade de soluções tecnológicas alinhadas aos
princípios de horizontalidade, garantindo que a segurança
seja distribuída e responsiva.

\section{Objetivos da Pesquisa}
\label{sec:objetivos-pesquisa}

O principal objetivo desta pesquisa é desenvolver um
protocolo de modelagem de ameaças que leve em consideração
as características únicas de organizações horizontais.
Inspirando-se nos conceitos de criptografia colaborativa
apresentados em \cite{AbcCrypto} e \cite{Colbac}, o estudo
busca propor uma abordagem que valorize a participação
coletiva sem comprometer a segurança e a eficiência.

Ademais, espera-se identificar ferramentas e métodos que
possam ser integrados em estruturas horizontais,
considerando os desafios específicos levantados por
\cite{Non-HierarchicalForms} e
\cite{WorkerCooperativesandRevolution}. O foco está em
equilibrar a distribuição do poder com a capacidade de
responder de maneira eficaz a ameaças externas e internas.

\section{Delimitação do Escopo}
\label{sec:delimitacao-escopo}

Esta pesquisa concentra-se em organizações não hierárquicas,
com ênfase em cooperativas de trabalhadores, movimentos
sociais e redes descentralizadas. Embora o escopo teórico
inclua uma ampla variedade de modelos, os estudos de caso se
limitarão a organizações que operam com princípios de
horizontalidade clara, como as descritas em
\cite{WorkerCooperativesinAmerica}.

Exclusões importantes incluem organizações que utilizam
apenas elementos horizontais de maneira parcial, ou aquelas
que possuem uma estrutura claramente hierárquica disfarçada
de horizontalidade. Essa delimitação visa garantir que as
conclusões sejam relevantes para o contexto específico das
organizações horizontais autênticas.

\section{Contribuições Esperadas}
\label{sec:contribuicoes-esperadas}

Espera-se que esta pesquisa forneça um protocolo prático e
acessível para organizações horizontais, ajudando-as a
identificar e mitigar ameaças de maneira eficaz. O trabalho
também contribuirá para a literatura acadêmica, preenchendo
lacunas identificadas em \cite{ParticipatoryThreatModelling}
e \cite{ThreatModelingASummaryOfAvailableMethods}, além de
oferecer soluções inspiradas em casos práticos de
cooperativas e redes descentralizadas.

Por fim, a pesquisa espera promover um diálogo entre
disciplinas, combinando teorias de ciência política,
tecnologia e administração para abordar os desafios das
organizações do futuro.

\section{Estrutura da Tese}
\label{sec:estrutura-tese}

A tese está organizada em seis capítulos principais. O
primeiro capítulo introduz o contexto histórico e
contemporâneo da pesquisa, detalhando os objetivos, a
delimitação do escopo e as contribuições esperadas. O
segundo capítulo oferece uma revisão detalhada do background
teórico, explorando modelos de governança horizontal e
ferramentas de modelagem de ameaças. Os capítulos seguintes
apresentam o design metodológico, os estudos de caso e as
análises empíricas, culminando em uma discussão crítica das
descobertas.

A conclusão resume os principais achados, oferecendo
recomendações práticas para organizações horizontais e
identificando direções futuras para pesquisa. A organização
da tese busca equilibrar rigor acadêmico com relevância
prática, assegurando que o conhecimento produzido seja útil
para acadêmicos e praticantes.

