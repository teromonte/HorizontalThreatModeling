%!TEX root = ../template.tex
%%%%%%%%%%%%%%%%%%%%%%%%%%%%%%%%%%%%%%%%%%%%%%%%%%%%%%%%%%%%%%%%%%%
%% chapter1.tex
%% NOVA thesis document file
%%%%%%%%%%%%%%%%%%%%%%%%%%%%%%%%%%%%%%%%%%%%%%%%%%%%%%%%%%%%%%%%%%%

%\epigraphfontsize{\small\itshape}
%\setlength\epigraphwidth{12.5cm}
%\setlength\epigraphrule{0pt}
%
%\epigraph{
%    "The very soldiers who had no courage in the White army yesterday are
%    very brave in the Red Army today; such is the effect of democracy."
%} {Mao Tsé-Tung}

\typeout{NT FILE chapter1.tex}%

\chapter{Introduction}
\label{cha:Introduction}

\prependtographicspath{{Chapters/Figures/Covers/}}

\section{Governança Organizacional: Uma Perspectiva Histórica}
\label{sec:contexto_historico}

A governança organizacional reflete as estruturas sociais, econômicas e
tecnológicas de cada época. Desde os primeiros agrupamentos humanos até as
organizações complexas da contemporaneidade, as formas de organizar o poder
e a tomada de decisão foram moldadas para responder a contextos
específicos. O modelo hierárquico, amplamente adotado, emergiu como solução
para demandas de controle e eficiência. Contudo, a história também registra
experimentos que desafiaram essa lógica, sugerindo a possibilidade de novas
abordagens na gestão e coordenação de atividades.

Mesmo em sistemas considerados pioneiros na horizontalidade, como a
democracia ateniense, a governança enfrentou limitações significativas
relacionadas à inclusão e à aplicabilidade prática, evidenciando
fragilidades na operacionalização da participação igualitária
\cite{AthenianDemocracyABrief}. Com o avanço da Revolução Industrial, a
centralização hierárquica intensificou-se para lidar com o crescimento e a
complexidade organizacional \cite{WorkerCooperativesandRevolution}.
Adicionalmente, experiências como as cooperativas e os movimentos
sindicalistas do século XIX delinearam alternativas à centralização absoluta,
enquanto tecnologias modernas, oferecendo estruturas descentralizadas que desafiam paradigmas
tradicionais de controle \cite{WorkerCooperativesinAmerica, EverydayRevolutions}.

Enquanto tecnologias de vigilância em massa reforçam estruturas
centralizadoras, bloqueando a adoção plena de governança horizontal,
inovações como o blockchain abrem novas possibilidades de descentralização,
ainda que enfrentem desafios na distribuição equitativa de poder e
recursos, como evidenciado na concentração de mineradores em redes públicas
\cite{DoArtifactsHavePolitics}.

Essas evoluções históricas e tecnológicas não apenas moldam as estruturas
de governança, mas também introduzem desafios únicos na modelagem de
ameaças. A análise crítica dessas tentativas permite identificar
vulnerabilidades e forças que fundamentam a construção de protocolos de
segurança em organizações horizontais.

\section{A Segurança Horizontal em Tempos de Interconexão}
\label{sec:desafios_contemporaneos}

No mundo interconectado atual, as organizações horizontais desafiam o
pressuposto de que a segurança depende de uma cadeia clara de comando. A
ausência de hierarquia formal pode se transformar em um ativo estratégico
ao dificultar ataques centralizados e ao permitir uma reconfiguração da
gestão da confiança, promovendo a resiliência organizacional
\cite{EverydayRevolutions}. Em sistemas de confiança distribuída,
como os utilizados em organizações baseadas em blockchain,
a segurança é promovida por mecanismos colaborativos que
substituem líderes formais por processos participativos e soluções
orientadas à transparência e consenso \cite{Reputation-basedDAO}.

Metodologias tradicionais de análise de ameaças, tais como \gls{stride} e árvores
de ataque, fornecem fundamentos valiosos, mas enfrentam limitações em
ambientes descentralizados, destacando a necessidade de abordagens mais
adequadas às especificidades de estruturas horizontais
\cite{ThreatModellingSurvey}. Contextos menos hierárquicos requerem
abordagens adaptadas que compreendam a complexidade da confiança horizontal
e dos potenciais riscos associados.

Nesse sentido, tecnologias como a criptografia colaborativa \cite{Colbac,
AbcCrypto} e abordagens de modelagem de ameaças que adotam a perspectiva global
da organização podem promover um entendimento mais realista da segurança em
estruturas descentralizadas. A horizontalidade, frequentemente vista como
um desafio, deve ser explorada como um ativo estratégico capaz de diluir
pontos únicos de falha e fortalecer a resiliência organizacional.

\section{Protocolo de Segurança para Organizações Não-Hierárquicas}
\label{sec:objetivos_pesquisa}

Esta pesquisa propõe um protocolo de segurança que integra a
horizontalidade como elemento estratégico, indo além da simples adaptação
de metodologias tradicionais. Em vez disso, busca demonstrar como a
descentralização, quando bem estruturada, reforça a resiliência frente a
ameaças complexas, mitigando pontos únicos de falha.

A proposta preenche uma lacuna na literatura sobre segurança em estruturas
horizontalizadas, abordando desafios apontados em estudos como o uso
limitado de metodologias tradicionais em contextos descentralizados
\cite{ThreatModellingSurvey} e a necessidade de criptografia adaptada
\cite{Colbac}. Considere organizações descentralizadas que gerenciam ativos
digitais sensíveis. A pergunta central é como garantir proteção contra
fraudes internas e ataques externos, preservando a governança horizontal.

O protocolo proposto oferecerá mecanismos de consenso, transparência e uma
modelagem de ameaças adaptada, como a inclusão de abordagens colaborativas
e participativas descritas em \cite{Colbac} e \cite{AbcCrypto}, fornecendo
soluções pragmáticas para tais desafios.


\section{Delimitando o Escopo da Pesquisa}
\label{sec:delimitacao_escopo}

A variedade de arranjos horizontais é ampla, e analisar todos em um único
estudo seria pouco produtivo. Para permitir uma análise detalhada e
alinhada com os objetivos da pesquisa, este trabalho concentra-se em
estruturas plenamente horizontais que exemplifiquem confiança distribuída,
governança democrática e mecanismos colaborativos para a tomada de decisão
\cite{Colbac}. Ambientes híbridos ou parcialmente horizontalizados, onde a
governança é compartilhada entre níveis hierárquicos e horizontais, ficam
fora do escopo, permitindo avaliar com maior precisão a eficácia do
protocolo em um cenário idealizado e mais controlado.

Futuras investigações poderão expandir este protocolo, adaptando suas
diretrizes a contextos organizacionais híbridos ou altamente dinâmicos,
como redes sociais e plataformas cooperativas digitais
\cite{CreatingTheCollectiveSocialMedia, Non-HierarchicalForms}. A escolha
por cooperativas de trabalhadores e redes comunitárias também reflete a
relevância prática dessas estruturas em demonstrar a viabilidade de
governança descentralizada e segurança distribuída
\cite{WorkerCooperativesinAmerica, EverydayRevolutions}.


\section{Contribuições Esperadas} 
\label{sec:contribuicoes_esperadas}

Esta pesquisa busca avançar a compreensão teórica da segurança em
estruturas horizontais, abordando lacunas relacionadas à aplicabilidade de
metodologias tradicionais em contextos descentralizados, como a falta de
adaptação às dinâmicas de governança horizontal identificadas em
\cite{ThreatModelingAsABasisForSecurityRequirements} e
\cite{DemystifyingTheThreatModelingProcess}. O objetivo é superar
adaptações limitadas de metodologias existentes, desenvolvendo um protocolo
que não apenas respeite os valores de participação coletiva, transparência
e confiança distribuída, mas também aproveitem a horizontalidade como ativo
estratégico, conforme sugerido em \cite{Colbac}.

Do ponto de vista prático, espera-se oferecer diretrizes que demonstrem
como a segurança pode ser integrada à governança democrática, promovendo
decisões participativas e protegendo ativos organizacionais de forma
descentralizada, conforme discutido em \cite{Non-HierarchicalForms}. Ao
fazê-lo, o protocolo busca demonstrar que a horizontalidade pode ser uma
vantagem estratégica, transformando a segurança em um catalisador para
autonomia e resiliência organizacional frente a ameaças complexas, como
enfatizado em \cite{Reputation-basedDAO} e \cite{AbcCrypto}.



\section{Estrutura da Tese} 
\label{sec:estrutura_tese}

Após a introdução, o capítulo de Background apresenta os fundamentos
da modelagem de ameaças e segurança em estruturas horizontais. O
capítulo de Related Work analisa estudos prévios, identificando
lacunas e oportunidades relevantes.

O capítulo de Design detalha o protocolo proposto, suas metodologias e
critérios de avaliação. O capítulo de Work Plan descreve as etapas e
cronograma da pesquisa, assegurando a execução estruturada e viável do
projeto. Por fim, o capítulo de Conclusão sintetiza os resultados,
discute limitações e propõe direções futuras, destacando a
horizontalidade como ativo estratégico para a segurança
organizacional.