%!TEX root = ../template.tex
%%%%%%%%%%%%%%%%%%%%%%%%%%%%%%%%%%%%%%%%%%%%%%%%%%%%%%%%%%%%%%%%%%%
%% chapter1.tex
%% NOVA thesis document file
%%%%%%%%%%%%%%%%%%%%%%%%%%%%%%%%%%%%%%%%%%%%%%%%%%%%%%%%%%%%%%%%%%%

%\epigraphfontsize{\small\itshape}
%\setlength\epigraphwidth{12.5cm}
%\setlength\epigraphrule{0pt}
%
%\epigraph{
%    "The very soldiers who had no courage in the White army yesterday are
%    very brave in the Red Army today; such is the effect of democracy."
%} {Mao Tsé-Tung}

\typeout{NT FILE chapter1.tex}%

\chapter{Introdução}
\label{cha:Introduction}

\prependtographicspath{{Chapters/Figures/Covers/}}

\section{Modelagem de Ameaças: Relevância e Desafios}
\label{sec:modelagem_ameacas}

A modelagem de ameaças é uma disciplina essencial no âmbito da segurança da
informação, cuja principal função é identificar, classificar e mitigar
vulnerabilidades em sistemas tecnológicos antes que estas possam ser exploradas
por adversários. Em um contexto onde os sistemas se tornam cada vez mais
complexos e integrados, a modelagem de ameaças se destaca como uma ferramenta
crítica para antecipar riscos e estabelecer medidas de segurança eficazes
\cite{ThreatModellingSurvey}.

As organizações hierárquicas, devido à centralização do controle e de recursos,
geralmente aplicam metodologias de modelagem de ameaças que dependem de fluxos
de decisão lineares e centralizados, como é o caso da metodologia STRIDE,
desenvolvida pela Microsoft \cite{ThreatModelingASummaryOfAvailableMethods}.
Esse modelo centralizado reflete uma estrutura organizacional que facilita a
implementação de políticas de segurança homogêneas, mas carece de flexibilidade
para lidar com as dinâmicas horizontais
\cite{EvaluationofCompetingThreatModeling}.

Em contrapartida, estruturas organizacionais horizontais, caracterizadas por uma
distribuição equitativa de poder e responsabilidades, apresentam desafios
específicos para a modelagem de ameaças \cite{Colbac}. A ausência de uma
hierarquia clara muitas vezes resulta em incompatibilidades estruturais, como
sistemas de controle de acesso que forçam a hierarquização, dificultando a
implementação de políticas democraticamente acordadas \cite{Colbac}.
Adicionalmente, a centralização de segredos organizacionais, como senhas, pode
gerar conflitos durante transições de liderança, o que é exemplificado no
fenômeno conhecido como pasword wars \cite{FromCounterpublicstoContentious}.

A utilização de ferramentas digitais por movimentos sociais também revela um
fenômeno conhecido como vanguard digital. Isso ocorre quando indivíduos que
controlam plataformas e contas digitais assumem posições de poder que podem
contrariar os princípios de igualdade estrutural
\cite{SocialMediaTeamsAsDigitalVanguards}. Além disso, ataques a processos
democráticos, como falsificação de identidades (Sybil attacks) e manipulação de
quorum, destacam vulnerabilidades exclusivas de sistemas horizontais devido à
dependência em processos participativos \cite{MitigationSybilAttack,
TheSybilAttack}. Finalmente, a falta de ferramentas tecnológicas que acomodem
transições entre centralizações temporárias e estruturas horizontais limita a
eficiência em cenários que demandam respostas rápidas ou especialização
\cite{Colbac}.

Esses desafios evidenciam a necessidade de adaptações nos métodos de segurança e
na modelagem de ameaças para que possam atender às especificidades das
organizações horizontais. A modelagem de ameaças para esses contextos deve
equilibrar aspectos técnicos e dinâmicas organizacionais, promovendo a
resiliência e a segurança em ambientes onde o poder e a responsabilidade são
compartilhados \cite{ThreatModelingdesigningForSecurity}.

\section{A Segurança Horizontal em Tempos de Interconexão}
\label{sec:desafios_contemporaneos}

No mundo interconectado atual, as organizações horizontais desafiam o
pressuposto de que a segurança depende de uma cadeia clara de comando. A
ausência de hierarquia formal pode se transformar em um ativo estratégico
ao dificultar ataques centralizados e ao permitir uma reconfiguração da
gestão da confiança, promovendo a resiliência organizacional
\cite{EverydayRevolutions}. Em sistemas de confiança distribuída,
como os utilizados em organizações baseadas em blockchain,
a segurança é promovida por mecanismos colaborativos que
substituem líderes formais por processos participativos e soluções
orientadas à transparência e consenso \cite{Reputation-basedDAO}.

Metodologias tradicionais de análise de ameaças, tais como \gls{stride} e árvores
de ataque, fornecem fundamentos valiosos, mas enfrentam limitações em
ambientes descentralizados, destacando a necessidade de abordagens mais
adequadas às especificidades de estruturas horizontais
\cite{ThreatModellingSurvey}. Contextos menos hierárquicos requerem
abordagens adaptadas que compreendam a complexidade da confiança horizontal
e dos potenciais riscos associados.

Nesse sentido, tecnologias como a criptografia colaborativa \cite{Colbac,
AbcCrypto} e abordagens de modelagem de ameaças que adotam a perspectiva global
da organização podem promover um entendimento mais realista da segurança em
estruturas descentralizadas. A horizontalidade, frequentemente vista como
um desafio, deve ser explorada como um ativo estratégico capaz de diluir
pontos únicos de falha e fortalecer a resiliência organizacional.

\section{Governança Organizacional: Uma Perspectiva Histórica}
\label{sec:contexto_historico}

A governança organizacional reflete as estruturas sociais, econômicas e
tecnológicas de cada época. Desde os primeiros agrupamentos humanos até as
organizações complexas da contemporaneidade, as formas de organizar o poder
e a tomada de decisão foram moldadas para responder a contextos
específicos. O modelo hierárquico, amplamente adotado, emergiu como solução
para demandas de controle e eficiência. Contudo, a história também registra
experimentos que desafiaram essa lógica, sugerindo a possibilidade de novas
abordagens na gestão e coordenação de atividades.

Mesmo em sistemas considerados pioneiros na horizontalidade, como a
democracia ateniense, a governança enfrentou limitações significativas
relacionadas à inclusão e à aplicabilidade prática, evidenciando
fragilidades na operacionalização da participação igualitária
\cite{AthenianDemocracyABrief}. Com o avanço da Revolução Industrial, a
centralização hierárquica intensificou-se para lidar com o crescimento e a
complexidade organizacional \cite{WorkerCooperativesandRevolution}.
Adicionalmente, experiências como as cooperativas e os movimentos
sindicalistas do século XIX delinearam alternativas à centralização absoluta,
enquanto tecnologias modernas, oferecendo estruturas descentralizadas que desafiam paradigmas
tradicionais de controle \cite{WorkerCooperativesinAmerica, EverydayRevolutions}.

Enquanto tecnologias de vigilância em massa reforçam estruturas
centralizadoras, bloqueando a adoção plena de governança horizontal,
inovações como o blockchain abrem novas possibilidades de descentralização,
ainda que enfrentem desafios na distribuição equitativa de poder e
recursos, como evidenciado na concentração de mineradores em redes públicas
\cite{DoArtifactsHavePolitics}.

Essas evoluções históricas e tecnológicas não apenas moldam as estruturas
de governança, mas também introduzem desafios únicos na modelagem de
ameaças. A análise crítica dessas tentativas permite identificar
vulnerabilidades e forças que fundamentam a construção de protocolos de
segurança em organizações horizontais.

\section{Protocolo de Segurança para Organizações Não-Hierárquicas}
\label{sec:objetivos_pesquisa}

Esta pesquisa propõe um protocolo de segurança que integra a
horizontalidade como elemento estratégico, indo além da simples adaptação
de metodologias tradicionais. Em vez disso, busca demonstrar como a
descentralização, quando bem estruturada, reforça a resiliência frente a
ameaças complexas, mitigando pontos únicos de falha.

A proposta preenche uma lacuna na literatura sobre segurança em estruturas
horizontalizadas, abordando desafios apontados em estudos como o uso
limitado de metodologias tradicionais em contextos descentralizados
\cite{ThreatModellingSurvey} e a necessidade de criptografia adaptada
\cite{Colbac}. Considere organizações descentralizadas que gerenciam ativos
digitais sensíveis. A pergunta central é como garantir proteção contra
fraudes internas e ataques externos, preservando a governança horizontal.

O protocolo proposto oferecerá mecanismos de consenso, transparência e uma
modelagem de ameaças adaptada, como a inclusão de abordagens colaborativas
e participativas descritas em \cite{Colbac} e \cite{AbcCrypto}, fornecendo
soluções pragmáticas para tais desafios.


\section{Delimitando o Escopo da Pesquisa}
\label{sec:delimitacao_escopo}

A variedade de arranjos horizontais é ampla, e analisar todos em um único
estudo seria pouco produtivo. Para permitir uma análise detalhada e
alinhada com os objetivos da pesquisa, este trabalho concentra-se em
estruturas plenamente horizontais que exemplifiquem confiança distribuída,
governança democrática e mecanismos colaborativos para a tomada de decisão
\cite{Colbac}. Ambientes híbridos ou parcialmente horizontalizados, onde a
governança é compartilhada entre níveis hierárquicos e horizontais, ficam
fora do escopo, permitindo avaliar com maior precisão a eficácia do
protocolo em um cenário idealizado e mais controlado.

Futuras investigações poderão expandir este protocolo, adaptando suas
diretrizes a contextos organizacionais híbridos ou altamente dinâmicos,
como redes sociais e plataformas cooperativas digitais
\cite{CreatingTheCollectiveSocialMedia, Non-HierarchicalForms}. A escolha
por cooperativas de trabalhadores e redes comunitárias também reflete a
relevância prática dessas estruturas em demonstrar a viabilidade de
governança descentralizada e segurança distribuída
\cite{WorkerCooperativesinAmerica, EverydayRevolutions}.


\section{Contribuições Esperadas} 
\label{sec:contribuicoes_esperadas}

A pesquisa entregará um protocolo de modelagem de ameaças específico para
organizações horizontais, integrando os princípios de governança distribuída,
participação coletiva e transparência. Esse protocolo será projetado para
identificar, avaliar e mitigar ameaças em estruturas descentralizadas,
fornecendo diretrizes práticas adaptadas às particularidades dessas
organizações. Além disso, será acompanhado por um método de avaliação para
validar sua eficácia e sua aplicação em casos reais ou simulados.

Espera-se que o protocolo demonstre como a horizontalidade pode ser um ativo
estratégico, reforçando a segurança e a resiliência organizacional frente a
ameaças complexas. Além de contribuir para o avanço teórico sobre segurança em
estruturas descentralizadas, a pesquisa busca oferecer uma solução prática que
fortaleça a autonomia e promova a integração entre segurança e governança
democrática.

\section{Estrutura da Tese} 
\label{sec:estrutura_tese}

Após a introdução, o capítulo de Background apresenta os fundamentos
da modelagem de ameaças e segurança em estruturas horizontais. O
capítulo de Related Work analisa estudos prévios, identificando
lacunas e oportunidades relevantes.

O capítulo de Design detalha o protocolo proposto, suas metodologias e
critérios de avaliação. O capítulo de Work Plan descreve as etapas e
cronograma da pesquisa, assegurando a execução estruturada e viável do
projeto. Por fim, o capítulo de Conclusão sintetiza os resultados,
discute limitações e propõe direções futuras, destacando a
horizontalidade como ativo estratégico para a segurança
organizacional.