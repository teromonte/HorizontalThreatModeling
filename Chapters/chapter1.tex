%!TEX root = ../template.tex
%%%%%%%%%%%%%%%%%%%%%%%%%%%%%%%%%%%%%%%%%%%%%%%%%%%%%%%%%%%%%%%%%%%
%% chapter1.tex
%% NOVA thesis document file
%%%%%%%%%%%%%%%%%%%%%%%%%%%%%%%%%%%%%%%%%%%%%%%%%%%%%%%%%%%%%%%%%%%

%\epigraphfontsize{\small\itshape}
%\setlength\epigraphwidth{12.5cm}
%\setlength\epigraphrule{0pt}
%
%\epigraph{
%    "The very soldiers who had no courage in the White army yesterday are
%    very brave in the Red Army today; such is the effect of democracy."
%} {Mao Tsé-Tung}

\typeout{NT FILE chapter1.tex}%

\chapter{Introduction}
\label{cha:introduction}

\section{Threat Modeling: Relevance and Challenges}
\label{sec:threat_modeling_relevance_challenges}

Threat modeling is an essential discipline in information security, whose main
function is to identify, classify and mitigate vulnerabilities in technological
systems before they can be exploited by adversaries
\cite{ThreatModelingdesigningForSecurity,
ThreatModelingASystematicLiteratureReview}. In a context where systems become
increasingly complex and integrated, threat modeling stands out as a critical
tool for anticipating risks and establishing effective security measures
\cite{DemystifyingTheThreatModelingProcess,
ThreatModelingASummaryOfAvailableMethods}.

Traditional models such as \gls{stride}, attack trees, and iterative
methodologies such as \gls{pasta} have been widely applied in hierarchical contexts
\cite{MicrosoftThreatModelingTechnique, AttackTrees, RiskCentricThreatModeling}.
These approaches focus on linear, hierarchical data flows, but face
significant challenges when applied to horizontal organizations, where the
distribution of power and responsibilities fundamentally alters risk dynamics
\cite{EvaluationofCompetingThreatModeling, Colbac}.

Horizontal organizational structures, characterized by the absence of a formal
hierarchy, face particular challenges in threat modeling \cite{Colbac}. The lack of
centralization can make it difficult to implement \gls{rbac} and other
systems that rely on hierarchical structures \cite{Colbac}. Furthermore, the
temporary centralization of organizational secrets, such as passwords or
encryption keys, often leads to conflicts known as "password wars" during
leadership transitions \cite{FromCounterpublicstoContentious}.

Additionally, digital tools often promote implicit centralization of power,
creating the phenomenon of the "digital vanguard", where individuals control
critical resources such as communication platforms \cite{SocialMediaTeamsAsDigitalVanguards}.
This is exacerbated by attacks specific to horizontal systems, such as identity spoofing (Sybil
attacks) and quorum manipulation, which exploit the dependency on participatory
processes \cite{MitigationSybilAttack, TheSybilAttack}.

The challenges highlighted indicate the need for adaptations in threat modeling
methods for horizontal contexts \cite{Colbac}. In addition to technical
limitations, such as the difficulty of integrating collaborative cryptography
\cite{AbcCrypto}, the need for participatory tools that respect democratic
dynamics and promote resilience is also highlighted \cite{SecurityCardsToolkit}.

\section{Horizontal Security in Times of Interconnection}
\label{sec:horizontal_security_interconnection}

In today's interconnected world, horizontal organizations challenge the
assumption that security depends on a clear chain of command
\cite{Non-HierarchicalForms, EverydayRevolutions}.
The absence of a formal hierarchy can become a strategic asset by preventing
centralized attacks and enabling a reconfiguration of trust management,
promoting organizational resilience \cite{EverydayRevolutions, Colbac}. In
distributed trust systems, such as those used in blockchain based organizations,
security is promoted by collaborative mechanisms that replace formal leaders
with participatory processes and solutions oriented towards transparency and
consensus \cite{Reputation-basedDAO, AbcCrypto}.

Traditional threat analysis methodologies such as \gls{stride} and \gls{pasta}
provide valuable insights but face limitations in decentralized environments,
highlighting the need for approaches better suited to the specificities of
horizontal structures \cite{ThreatModellingSurvey,
ThreatModelingASummaryOfAvailableMethods}. Less hierarchical contexts require
adapted approaches that understand the complexity of horizontal trust and the
potential associated risks \cite{Colbac}.

In this sense, technologies such as collaborative cryptography \cite{Colbac,
AbcCrypto} and threat modeling approaches that adopt the global perspective of
the organization can promote a more realistic understanding of security in
decentralized structures. Horizontality, often seen as a challenge, should be
explored as a strategic asset capable of diluting single points of failure and
strengthening organizational resilience \cite{EverydayRevolutions}.

\section{Organizational Governance: A Historical Perspective}
\label{sec:organizational_governance_historical}

Organizational governance reflects the social, economic, and technological
structures of each era. From the earliest human groups to the complex
organizations of contemporary times, the ways of organizing power and
decision making have been shaped to respond to specific contexts
\cite{Non-HierarchicalForms}. The hierarchical model, widely adopted, emerged as
a solution to demands for control and efficiency. However, history also records
experiments that challenged this logic, suggesting the possibility of new
approaches to the management and coordination of activities
\cite{WorkerCooperativesinAmerica, WorkerCooperativesandRevolution}.

Even in systems considered pioneers in horizontality, such as Athenian
democracy, governance faced significant limitations related to inclusion and
practical applicability, highlighting weaknesses in the operationalization of
equal participation \cite{AthenianDemocracyABrief}. As the Industrial Revolution
progressed, hierarchical centralization intensified to cope with organizational
growth and complexity \cite{WorkerCooperativesandRevolution}. Additionally,
experiences such as cooperatives and the 19th century labor movement outlined
alternatives to absolute centralization \cite{WorkerCooperativesinAmerica, EverydayRevolutions}.

Innovations such as the internet open up new possibilities for decentralization,
yet face challenges in equitable distribution of power and resources, as
evidenced by the concentration of miners on public networks
\cite{Energytheftdetectionissues}.

These historical and technological evolutions not only shape governance
structures, but also introduce unique challenges in threat modeling
\cite{DoArtifactsHavePolitics, Democraciaeoscodigosinvisiveis}. Critical analysis of
these attempts allows us to identify vulnerabilities and strengths that underpin
the construction of security protocols in horizontal organizations
\cite{Colbac}.

\section{Security Protocol for Non-Hierarchical Organizations}
\label{sec:security_protocol_nonhierarchical}

This research proposes a security protocol that integrates horizontality as a
strategic element, going beyond the simple adaptation of traditional
methodologies. The main objective is to demonstrate how
decentralization, when structured in a way that is coherent with organizational
principles, can strengthen resilience against complex threats, mitigating single
points of failure and distributing responsibilities equitably. The protocol aims to balance
operational efficiency and democratic participation, ensuring that security
measures do not compromise decision making agility or the inclusion of members
in critical processes. To this end, it relies on
collaborative approaches, such as cryptography adapted to horizontal contexts
\cite{Colbac}, and on threat modeling methodologies that consider participatory
dynamics \cite{ParticipatoryThreatModelling}.

The integration between security and governance is addressed through guidelines
that harmonize technical requirements with organizational principles
\cite{ParticipatoryThreatModelling}. A good protocol should incorporate
modular structures that allow adaptation to different levels of
horizontality, from fully decentralized networks to organizations with more
hierarchical structures \cite{Colbac}. This flexibility is essential to respond
to dynamic threats without compromising the autonomy of members and to cover the
largest number of organizations \cite{Colbac}.

The protocol should strengthen resilience by combining technical and social layers:
cryptographic techniques protect against external threats, while radical
transparency structures and periodic reviews by rotating committees prevent
internal fraud \cite{EverydayRevolutions}. The traceability of decisions via
immutable records ensures that vulnerabilities are identified and corrected
collaboratively, in line with studies on failures in distributed systems
\cite{Reputation-basedDAO}.

\section{Defining the Scope of Research}
\label{sec:defining_research_scope}

The diversity of horizontal organizations ranges from informal collectives to
complex digital networks, each with its own dynamics \cite{EverydayRevolutions}.
To ensure analytical focus, this study is limited to structures that operate
under strict principles of horizontality, characterized by: (1) absence of
formal hierarchies or permanent centralization of power; (2) decision making
processes based on consensus or broad participation; and (3) explicit mechanisms
for distributing responsibilities and resources.

The choice to analyze organizations such as worker cooperatives and community
networks is justified by their empirical relevance: these models have robust
documentation on operational challenges \cite{WorkerCooperativesinAmerica}, in
addition to explicitly adopting principles of self management and radical
transparency \cite{EverydayRevolutions}. These characteristics allow testing the
protocol in contexts where security depends directly on collective coordination,
without intermediaries or central authorities
\cite{ThreatModelingdesigningForSecurity}.

Although digital platforms and decentralized social networks
\cite{CreatingTheCollectiveSocialMedia} represent equally relevant cases, their
dynamic nature and dependence on heterogeneous technical infrastructures would
require methodological adaptations beyond the current scope. Future studies
could explore these variations, using the protocol developed here as a basis for
comparative analyses in less controlled environments.


\section{Expected Contributions}
\label{sec:expected_contributions}

The research will deliver a threat modeling protocol specific to horizontal
organizations, integrating the principles of distributed governance, collective
participation, and transparency. This protocol will be designed to
identify, understand and mitigate threats in decentralized structures, providing
practical guidelines adapted to the particularities of these organizations. In
addition, it will be accompanied by an evaluation method to validate its
effectiveness and its application in real cases.

The objective is to demonstrate how horizontality can be a strategic
asset, reinforcing organizational security and resilience in the face of complex
threats. In addition the research seeks to offer a practical solution that
strengthens autonomy and promotes the integration between security and
democratic governance.

\section{Structure of the Thesis}
\label{sec:structure_thesis}

After the introduction, the Background chapter presents the fundamentals of
threat modeling and security in horizontal structures. The Related Work chapter
analyzes previous studies, identifying relevant gaps and opportunities. After
having established the theoretical foundation, we advance to the Solution
chapter that details the proposed protocol. In the end the Evaluation chapter
presents the results of the application of the protocol in real cases.

\section*{} 
The first chapter presented the context that support this research, identifying
challenges faced by non-hierarchical organizations in the area of digital
security and threat modeling. The structural particularities of these
organizations were described, highlighting the need for specific methods for
risk analysis and mitigation. Having defined the objectives, intended
contributions and structure of this research, Chapter 2 will address the
essential concepts that underpin the study, including threat modeling, the
conceptual differences between horizontal organizations and organizations
without explicit leadership, and the role of democratic centralism as an
organizational principle.
