%!TEX root = ../template.tex
%%%%%%%%%%%%%%%%%%%%%%%%%%%%%%%%%%%%%%%%%%%%%%%%%%%%%%%%%%%%%%%%%%%
%% chapter1.tex
%% NOVA thesis document file
%%
%% Chapter with introduction
%%%%%%%%%%%%%%%%%%%%%%%%%%%%%%%%%%%%%%%%%%%%%%%%%%%%%%%%%%%%%%%%%%%

\typeout{NT FILE chapter1.tex}%

\prependtographicspath{{Chapters/Figures/Covers/}}

% epigraph configuration
% \epigraphfontsize{\small\itshape}
% \setlength\epigraphwidth{12.5cm}
% \setlength\epigraphrule{0pt}
% 
% \epigraph{
% 
% }

%\chapter{Introduction}
%\label{cha:introduction}
%
%\section{Context}
%\label{sec:context}
%
%A segurança cibernética moderna frequentemente pressupõe a existência
%de estruturas hierárquicas, o que pode não ser adequado para
%organizações que operam de maneira horizontal, como cooperativas de
%trabalhadores, sindicatos e grupos ativistas. Essas organizações
%enfrentam desafios específicos em termos de segurança devido à
%inadequação das soluções tradicionais, que geralmente ignoram a
%horizontalidade em sua concepção \cite{Colbac,
%AHybridThreatModelingMethod, EvaluationofCompetingThreatModeling,
%ThreatModelingASystematicLiteratureReview,
%ParticipatoryThreatModelling,
%ThreatModelingASummaryOfAvailableMethods}.
%
%Organizações hierárquicas normalmente facilitam a implementação de
%políticas de segurança devido à clara atribuição de responsabilidades
%e controle, como destacado em práticas tradicionais de modelagem de
%ameaças, incluindo STRIDE e DREAD
%\cite{ThreatModelingASystematicLiteratureReview,
%ExperiencesThreatModelingAtMicrosoft, ThreatModellingSurvey}. Por
%outro lado, organizações horizontais distribuem o poder de forma
%equitativa, baseando decisões em processos coletivos e participativos.
%Essa estrutura pode ser mais inclusiva e transparente, mas apresenta
%desafios únicos, como a gestão de segredos e o controle de acesso
%\cite{Colbac, CoReTM, ParticipatoryThreatModelling}.
%
%Soluções tradicionais, como as fornecidas pelo STRIDE, muitas vezes
%dependem de diagramas de fluxo de dados (DFDs) e de processos
%hierárquicos de decisão
%\cite{ThreatModelingASystematicLiteratureReview,
%AHybridThreatModelingMethod, ThreatModellingSurvey}. Essas abordagens
%frequentemente falham em abordar as complexidades organizacionais
%horizontais, onde é necessário um equilíbrio cuidadoso entre
%participação coletiva e a segurança de informações sensíveis
%\cite{EvaluationofCompetingThreatModeling,
%ParticipatoryThreatModelling}. 
%
%A horizontalidade, porém, pode ser vista como um ativo. Protocolos de
%segurança que respeitem essa estrutura podem reforçar os princípios
%fundamentais de participação e igualdade das organizações. Isso exige
%o desenvolvimento de novas abordagens que considerem tanto as ameaças
%externas quanto os requisitos internos de governança horizontal
%\cite{Colbac, CoReTM, DemystifyingTheThreatModelingProcess,
%ThreatModelingASummaryOfAvailableMethods}.
%
%Além disso, a modelagem de ameaças em organizações horizontais deve
%considerar a complexidade adicional de gerenciar segredos e acessos de
%forma distribuída. Em organizações hierárquicas, segredos como senhas
%e chaves de acesso são frequentemente centralizados e controlados por
%administradores. Em contraste, em organizações horizontais, esses
%segredos precisam ser gerenciados de maneira que todos os membros
%tenham acesso equitativo, sem comprometer a segurança \cite{Colbac}.
%
%\section{Objective}
%\label{sec:objective}
%
%O objetivo desta pesquisa é desenvolver um protocolo de modelagem de
%ameaças adaptado para organizações não-hierárquicas, como
%cooperativas, sindicatos e projetos de software de código aberto. Este
%protocolo busca resolver os desafios de segurança enfrentados por
%essas organizações, valorizando a horizontalidade como um diferencial
%positivo \cite{ThreatModelingAsABasisForSecurityRequirements,
%AHybridThreatModelingMethod, ParticipatoryThreatModelling}. A
%abordagem proposta considera não apenas as ameaças externas, mas
%também os processos democráticos internos, garantindo soluções que
%respeitem a igualdade e a participação \cite{Colbac}.
%
%Para validar o protocolo, serão conduzidas avaliações em diferentes
%organizações horizontais, comparando seus resultados com abordagens
%tradicionais, como STRIDE e DREAD, destacando vantagens e desvantagens
%em contextos não-hierárquicos
%\cite{ThreatModelingASystematicLiteratureReview,
%EvaluationofCompetingThreatModeling,
%ExperiencesThreatModelingAtMicrosoft,
%ThreatModelingASummaryOfAvailableMethods}.
%
%\section{Contributions}
%\label{sec:contributions}
%
%As contribuições desta pesquisa incluem o desenvolvimento de um
%protocolo de modelagem de ameaças inovador que considera a
%horizontalidade como um ativo estratégico. Este protocolo será
%projetado para abordar os desafios específicos enfrentados por
%cooperativas, sindicatos e projetos de código aberto
%\cite{ThreatModelingASystematicLiteratureReview, CoReTM,
%ParticipatoryThreatModelling}. Além disso, será realizada uma
%avaliação comparativa detalhada do protocolo com métodos tradicionais,
%como STRIDE, para ilustrar a aplicabilidade prática e eficácia em
%diferentes contextos \cite{AHybridThreatModelingMethod,
%EvaluationofCompetingThreatModeling,
%ExperiencesThreatModelingAtMicrosoft,
%ThreatModelingASummaryOfAvailableMethods}.
%
%Por fim, espera-se que este trabalho contribua para o avanço das
%práticas de segurança cibernética, promovendo soluções inclusivas e
%respeitando os princípios de participação e igualdade \cite{Colbac,
%AbcCrypto}.





\chapter{Introduction}
\label{cha:introduction}

\section{Context}
\label{sec:context}


A segurança cibernética moderna frequentemente pressupõe a existência de
estruturas hierárquicas, o que pode não ser adequado para organizações que
operam de maneira horizontal, como cooperativas de trabalhadores, sindicatos e
grupos ativistas. Essas organizações enfrentam desafios específicos em termos de
segurança devido à inadequação das soluções tradicionais, que geralmente ignoram
a horizontalidade em sua concepção. Por exemplo, métodos como STRIDE, amplamente
utilizados em modelagem de ameaças, dependem fortemente de diagramas de fluxo de
dados (DFDs) e processos hierárquicos de decisão, mas enfrentam limitações
significativas ao abordar organizações horizontais, conforme explorado por
Shostak (2014) e outros autores \cite{ThreatModelingASystematicLiteratureReview,
AHybridThreatModelingMethod, ThreatModellingSurvey}.

Organizações hierárquicas normalmente facilitam a implementação de políticas de
segurança devido à clara atribuição de responsabilidades e controle. Essa
abordagem é evidente em práticas tradicionais como STRIDE e DREAD, que
estruturam as ameaças em categorias bem definidas, tornando o processo mais
direto, ainda que por vezes oneroso \cite{ExperiencesThreatModelingAtMicrosoft,
ThreatModellingSurvey, DemystifyingTheThreatModelingProcess}.
Em contraste, organizações horizontais distribuem o poder de forma
equitativa, baseando decisões em processos coletivos e participativos.
Essa estrutura, enquanto mais inclusiva e transparente, apresenta
desafios únicos, como a gestão de segredos e o controle de acesso,
temas amplamente debatidos na literatura sobre segurança para estruturas
horizontais \cite{Colbac, ParticipatoryThreatModelling, CoReTM}.

Além disso, autores como Uzunov e Fernandez (2014) destacam que a modelagem de
ameaças frequentemente ignora as complexidades de governança horizontal,
requerendo ferramentas que incorporem múltiplas perspectivas
\cite{ThreatModelingASystematicLiteratureReview}. Por outro lado, metodologias
participativas, como a descrita por CoReTM, demonstram a eficácia da colaboração
e da diversidade de expertise na modelagem de ameaças, mas ainda enfrentam
desafios de implementação \cite{CoReTM}.

A horizontalidade, porém, pode ser vista como um ativo estratégico. Protocolos
de segurança que respeitem essa estrutura podem reforçar os princípios
fundamentais de participação e igualdade das organizações. Esse conceito é
destacado em trabalhos como COLBAC, que explora tecnologias para suporte a
organizações democráticas e horizontais, enfatizando a importância de integrar
políticas de autorização baseadas na coletividade \cite{Colbac}. No entanto,
como aponta o estudo da COLBAC, soluções eficazes para segurança horizontal
exigem não apenas tecnologias, mas também adaptações sociotécnicas e um
alinhamento entre estruturas organizacionais e suas ferramentas tecnológicas.

Finalmente, para organizações horizontais, a gestão de segredos, como senhas e
chaves de acesso, precisa ser distribuída de forma justa e segura, evitando
concentração de poder. Isso contrasta diretamente com práticas hierárquicas
tradicionais, onde essas responsabilidades são centralizadas \cite{Colbac}.

\section{Objective}
\label{sec:objective}

O objetivo desta pesquisa é desenvolver um protocolo de modelagem de ameaças
adaptado para organizações não hierárquicas, como cooperativas, sindicatos e
projetos de software de código aberto. Esse protocolo busca resolver os desafios
específicos enfrentados por essas organizações, valorizando a horizontalidade
como um diferencial positivo. A abordagem aqui proposta considera tanto as
ameaças externas quanto os processos democráticos internos, garantindo soluções
que respeitem a igualdade e a participação
\cite{ThreatModelingAsABasisForSecurityRequirements,
AHybridThreatModelingMethod, ParticipatoryThreatModelling}.

Para validar o protocolo, serão conduzidas avaliações em diferentes organizações
horizontais, comparando seus resultados com abordagens tradicionais, como STRIDE
e DREAD. Esses métodos tradicionais, ainda que robustos, frequentemente dependem
de diagramas de fluxo de dados e abordagens sistemáticas, como checklist, que
não capturam nuances participativas
\cite{ThreatModelingASystematicLiteratureReview,
ExperiencesThreatModelingAtMicrosoft}. A comparação irá destacar vantagens e
desvantagens de cada abordagem no contexto de organizações horizontais.

\section{Contributions}
\label{sec:contributions}

As contribuições desta pesquisa incluem o desenvolvimento de um
protocolo de modelagem de ameaças inovador que considera a
horizontalidade como um ativo estratégico. Este protocolo será
projetado para abordar os desafios específicos enfrentados por
cooperativas, sindicatos e projetos de código aberto
\cite{ThreatModelingASystematicLiteratureReview, CoReTM,
ParticipatoryThreatModelling}. Além disso, será realizada uma
avaliação comparativa detalhada do protocolo com métodos tradicionais,
como STRIDE, para ilustrar a aplicabilidade prática e eficácia em
diferentes contextos \cite{AHybridThreatModelingMethod,
EvaluationofCompetingThreatModeling,
ExperiencesThreatModelingAtMicrosoft,
ThreatModelingASummaryOfAvailableMethods}.

Por fim, espera-se que este trabalho contribua para o avanço das
práticas de segurança cibernética, promovendo soluções inclusivas e
respeitando os princípios de participação e igualdade \cite{Colbac,
AbcCrypto}.

\section{Structure}
\label{sec:structure}

O primeiro capítulo apresenta o contexto e os objetivos do estudo. No
segundo capítulo, são revisados trabalhos relacionados e fundamentos
teóricos, com ênfase em abordagens participativas e tradicionais de
modelagem de ameaças. O terceiro capítulo detalha o protocolo
proposto. O quarto capítulo sintetiza os resultados e propõe direções
futuras. Finalmente, o capítulo Work Plan apresenta o cronograma e as
etapas de validação.
