%!TEX root = ../template.tex
%%%%%%%%%%%%%%%%%%%%%%%%%%%%%%%%%%%%%%%%%%%%%%%%%%%%%%%%%%%%%%%%%%%
%% chapter1.tex
%% NOVA thesis document file
%%
%% Chapter with introduction
%%%%%%%%%%%%%%%%%%%%%%%%%%%%%%%%%%%%%%%%%%%%%%%%%%%%%%%%%%%%%%%%%%%

\typeout{NT FILE chapter1.tex}%

\chapter{Introduction}
\label{cha:introduction}

\prependtographicspath{{Chapters/Figures/Covers/}}

% epigraph configuration
\epigraphfontsize{\small\itshape}
\setlength\epigraphwidth{12.5cm}
\setlength\epigraphrule{0pt}

\includegraphics[width=0.1\linewidth]{NOVAthesisFiles/Images/novathesis-insignia}\hfill
\includegraphics[width=0.875\linewidth]{NOVAthesisFiles/Images/novathesis-text}

\noindent This is the \gls{novathesis} \LaTeX\ template \ntindex[Template!]{Version} \novathesisversion\ from   {Template!date}\novathesisdate.

\epigraph{
  This work is licensed under the \href{https://www.latex-project.org/lppl/lppl-1-3c/}{\LaTeX\ Project Public License v1.3c}.
  To view a copy of this \ntindex[Template!]{license}, visit the \href{https://www.latex-project.org/lppl/}{LaTeX project public license}.
}

\section{Welcome to}
\label{sec:if_you_use_this_template}

This first Chapter introduces the \gls{novathesis} template and how it is organized. In Chapter~\ref{cha:users_manual} you can find some specific instructions on how to use this template.  Chapter~\ref{cha:a_short_latex_tutorial_with_examples} shows some examples and give some hints on how to write your text. Please read these next Chapters carefully.

\subsection{Your Time is Precious}
\label{sub:time_is_money}

Did you learn how to drive by sitting by the wheel and throwing your car into the road?  Most probably you did take your time \emph{learning the rules} and \emph{practicing} first! Likewise, it is not wise to throw yourself at the task of writing a thesis/dissertation in \LaTeX\ without seriously considering the following \ntindex{recommendation}!

