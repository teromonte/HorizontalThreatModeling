%!TEX root = ../template.tex
%%%%%%%%%%%%%%%%%%%%%%%%%%%%%%%%%%%%%%%%%%%%%%%%%%%%%%%%%%%%%%%%%%%%
%% chapter2.tex
%% NOVA thesis document file
%%
%% Chapter with the template manual
%%%%%%%%%%%%%%%%%%%%%%%%%%%%%%%%%%%%%%%%%%%%%%%%%%%%%%%%%%%%%%%%%%%%

\typeout{NT FILE chapter2.tex}%

\chapter{Background and Related Work}
\label{cha:background}

\glsresetall

\section{Modelagem de Ameaças: Conceitos e Estratégias}
\label{sec:threat_modeling}

A modelagem de ameaças é um processo estruturado que permite identificar,
analisar e mitigar vulnerabilidades em sistemas, compondo um elemento essencial
para a segurança cibernética. Este processo também serve como base para a
definição de requisitos de segurança e é amplamente utilizado no desenvolvimento
de sistemas mais resilientes. Segundo Torr
\cite{DemystifyingTheThreatModelingProcess}, a modelagem de ameaças deve ser
integrada aos processos de design e especificação desde o início, como uma
prática preventiva essencial.

\subsection{Definições e Objetivos da Modelagem de Ameaças}

Diversos autores oferecem definições que destacam aspectos fundamentais da
modelagem de ameaças. Uzunov e Fernandez
\cite{ThreatModelingASystematicLiteratureReview} descrevem-na como um processo
para analisar ataques potenciais, utilizando bibliotecas de ameaças ou
taxonomias. Em um contexto mais específico, Shull et al.
\cite{EvaluationofCompetingThreatModeling} destacam que a modelagem de ameaças
cria uma abstração do sistema de software para identificar habilidades e metas
dos atacantes, gerando catálogos de possíveis ameaças que precisam ser
mitigadas.

As metodologias de modelagem de ameaças são ferramentas valiosas para responder
a perguntas como: "Quais são os ativos mais críticos?", "Quem são os
adversários?" e "Quais vulnerabilidades podem ser exploradas?"
\cite{ThreatModelingAsABasisForSecurityRequirements}.

\subsection{Ferramentas e Representações Visuais}

As ferramentas desempenham um papel crucial na eficácia da modelagem de ameaças.
\textit{Data Flow Diagrams} (DFDs), por exemplo, são amplamente usados para
mapear fluxos de dados e identificar fronteiras de confiança
\cite{ThreatModelingASummaryOfAvailableMethods}. As árvores de ataque são outro
exemplo comum, oferecendo uma maneira hierárquica de decompor ameaças
\cite{ThreatModellingSurvey}.

\subsection{Desafios e Limitações}

Embora eficaz, a modelagem de ameaças enfrenta desafios, como a dependência de
DFDs precisos e a complexidade crescente em sistemas maiores
\cite{ThreatModelingASummaryOfAvailableMethods}. Almashaqbeh et al.
\cite{AbcCrypto} destacam que modelos específicos para domínios emergentes, como
criptomoedas, ainda estão em estágios iniciais de desenvolvimento.

\subsection{Futuras Direções}

Para melhorar a modelagem de ameaças, é crucial integrar abordagens
colaborativas e adaptativas \cite{CoReTM}. Modelos que considerem dinâmicas
horizontais, como o COLBAC, e a participação ativa dos usuários, como o
Participatory Threat Modeling, mostram promessas significativas
\cite{Colbac,ParticipatoryThreatModelling}.

Com uma evolução constante nas metodologias e ferramentas, a modelagem de
ameaças continuará sendo uma área essencial para avanços em segurança
cibernética e resiliência organizacional.



\subsection{STRIDE}
\label{sec:STRIDE}

A metodologia STRIDE é uma das mais maduras e amplamente utilizadas para
modelagem de ameaças. Desenvolvida por Loren Kohnfelder e Praerit Garg em 1999 e adotada
pela Microsoft em 2002, STRIDE evoluiu ao longo do tempo para incluir novas tabelas
específicas de ameaças e variantes como STRIDE-per-Element e STRIDE-per-Interaction.
A metodologia STRIDE é baseada na criação de Diagramas de Fluxo de Dados
(DFDs) para identificar entidades do sistema, eventos e limites do sistema. A
precisão dos DFDs é crucial para o sucesso da aplicação do STRIDE, embora seu uso
exclusivo possa ser limitante, pois não representa decisões arquitetônicas relacionadas
à segurança. 

O acrônimo STRIDE representa seis categorias de ameaças: Spoofing
(falsificação de identidade), Tampering (manipulação de dados), Repudiation (repúdio),
Information Disclosure (divulgação de informações), Denial of Service (negação de serviço)
e Elevation of Privilege (elevação de privilégio). Cada uma dessas categorias
corresponde a uma propriedade de segurança violada, como autenticação, integridade,
não-repúdio, confidencialidade, disponibilidade e autorização, respectivamente.
A metodologia STRIDE é utilizada para identificar ameaças conhecidas com base
nessas categorias, auxiliando na navegação pelo modelo do sistema criado na fase
inicial. 

Apesar de a Microsoft não manter mais o STRIDE, ele ainda é implementado como
parte do Ciclo de Vida de Desenvolvimento Seguro da Microsoft (SDL) com a Ferramenta
de Modelagem de Ameaças, que continua disponível. A metodologia
STRIDE é fácil de adotar, mas pode ser demorada, especialmente à medida que a
complexidade do sistema aumenta. Estudos descritivos da técnica de modelagem de ameaças da
Microsoft mostram que o STRIDE tem uma taxa moderadamente baixa de falsos positivos e
uma taxa moderadamente alta de falsos negativos. 

A aplicação do STRIDE não se limita a sistemas cibernéticos, mas também a
sistemas ciber-físicos, demonstrando sua versatilidade. Além disso, a metodologia
STRIDE pode ser combinada com outras abordagens de modelagem de ameaças para criar
uma visão mais robusta e abrangente das potenciais ameaças. A
escolha da metodologia de modelagem de ameaças deve considerar áreas específicas a
serem abordadas, o tempo disponível para a modelagem, a experiência com modelagem de
ameaças e o nível de envolvimento dos stakeholders. 

Em resumo, a metodologia STRIDE é uma ferramenta poderosa para identificar e
mitigar ameaças em sistemas complexos. Sua aplicação em organizações não hierárquicas
pode ser particularmente valiosa, pois permite uma abordagem sistemática para a
segurança, considerando a horizontalidade como um ativo. A adoção de STRIDE, juntamente
com outras metodologias de modelagem de ameaças, pode proporcionar uma defesa mais
focada e eficaz contra ameaças cibernéticas. 

\subsection{Attack trees}
\label{sec:attack_trees}

As árvores de ataque são uma das técnicas mais antigas e amplamente aplicadas
para modelagem de ameaças em sistemas cibernéticos, ciber-físicos e físicos.
Desenvolvidas por Bruce Schneier em 1999, inicialmente foram aplicadas como um método
independente e, desde então, têm sido combinadas com outros métodos e frameworks.
As árvores de ataque são essencialmente diagramas que representam ataques a um
sistema em forma de árvore. A raiz da árvore é o objetivo do ataque, e as folhas são
as maneiras de alcançar esse objetivo. Cada objetivo é representado como uma
árvore separada, resultando em um conjunto de árvores de ataque para a análise de
ameaças do sistema. 

A construção de uma árvore de ataque geralmente requer algumas iterações de
decomposição do objetivo. Uma vez identificados todos os nós folha, podem ser atribuídos
marcadores de possibilidade, que devem ser definidos após uma pesquisa relevante sobre
cada etapa. Durante o exame de diferentes métodos para alcançar o objetivo, pode-se
perceber que isso pode ser realizado de várias maneiras. Para incorporar essas
diferentes opções na árvore, devem ser usados nós AND e OR. Nós AND indicam que ambos os
nós devem ser realizados para avançar para a próxima etapa, enquanto nós OR
representam alternativas. Em sistemas complexos, árvores de ataque
podem ser construídas para cada componente, em vez de para o sistema como um todo. 

As árvores de ataque são fáceis de entender e adotar, mas são úteis apenas
quando o sistema e as preocupações de segurança são bem compreendidos. O método
assume que os analistas possuem alta expertise em cibersegurança e, portanto, não
fornece diretrizes para avaliar sub-objetivos, ataques ou riscos.
Nos últimos anos, essa técnica tem sido frequentemente usada em combinação com
outras técnicas e dentro de frameworks como STRIDE, CVSS e PASTA.
A aplicação de árvores de ataque pode ajudar a tomar decisões de segurança,
verificar se os sistemas são vulneráveis a um ataque e avaliar um tipo específico de
ataque. 

Um objetivo adicional do método é gerar portas de ataque para componentes
individuais. Essas portas de ataque, que são efetivamente nós raiz para as árvores de
ataque dos componentes, ilustram atividades que podem passar risco para os
componentes conectados. A pontuação auxilia no processo de realização de uma avaliação de
risco do sistema. Se uma porta de ataque depende de um nó raiz de componente com uma
alta pontuação de risco, essa porta de ataque também terá uma alta pontuação de
risco e uma alta probabilidade de ser executada. O oposto também é verdadeiro.
Este método foi utilizado em um estudo de caso para uma rede de comunicações
ferroviárias, demonstrando sua aplicabilidade prática. 

Em resumo, as árvores de ataque são uma ferramenta poderosa para identificar e
mitigar ameaças em sistemas complexos. Sua aplicação em organizações não hierárquicas
pode ser particularmente valiosa, pois permite uma abordagem sistemática para a
segurança, considerando a horizontalidade como um ativo. A adoção de árvores de ataque,
juntamente com outras metodologias de modelagem de ameaças, pode proporcionar uma defesa
mais focada e eficaz contra ameaças cibernéticas. 

\subsection{Security Cards}
\label{sec:Security Cards}

\subsection{Personna non Grata}
\label{sec:Personna}

\subsection{PASTA}
\label{sec:PASTA}



\section{Trabalhos Relacionados}
\label{sec:related}

A modelagem de ameaças é um campo dinâmico e em constante evolução, com várias
abordagens e frameworks desenvolvidos ao longo dos anos para atender a diferentes
necessidades e contextos. Este capítulo revisa alguns dos trabalhos mais relevantes na área
de modelagem de ameaças, com foco em métodos que podem ser adaptados ou servir de
inspiração para o desenvolvimento de um protocolo específico para organizações
não-hierárquicas. 

No artigo "ABC: A Cryptocurrency-Focused Threat Modeling Framework", os autores
Ghada Almashaqbeh, Allison Bishop e Justin Cappos propõem um modelo de ameaças
específico para criptomoedas, destacando a necessidade de frameworks especializados
para lidar com as particularidades desses sistemas. 

O framework ABC introduz a matriz de colusão como uma inovação chave,
permitindo cobrir um amplo espectro de casos de ameaças sem tornar o processo
excessivamente complexo. Este modelo é particularmente eficaz em identificar riscos
financeiros, como demonstrado em estudos de caso reais e em um estudo de usuários, onde
71\% dos participantes que utilizaram o ABC conseguiram identificar ameaças
financeiras, em comparação com apenas 13\% dos que usaram o STRIDE. Esta abordagem é
particularmente importante em criptomoedas permissionless, onde qualquer pessoa pode
participar e onde a colusão entre atacantes é uma preocupação significativa. 

No contexto de organizações não-hierárquicas, a modelagem de ameaças deve considerar
a horizontalidade como um ativo. Nós visamos desenvolver protocolos de modelagem de ameaças que
valorizem a horizontalidade, avaliando esses novos protocolos com membros de grupos
com diferentes níveis de horizontalidade. 

Assim como o framework ABC aborda as especificidades das criptomoedas, o
trabalho busca adaptar a modelagem de ameaças para contextos onde a
ausência de hierarquia é uma característica fundamental. Ambos os trabalhos
destacam a importância de frameworks especializados que considerem as
particularidades dos sistemas e organizações que visam proteger.

\section{Princípios de Organizações Horizontais}
\label{sec:introduction}

Organizações horizontais buscam minimizar ou eliminarminimizam
hierarquias tradicionais, promovendo participação igualitária em processos de tomada
de decisão. Este modelo de organização reflete práticas que priorizam
odecisões. Este modelo prioriza trabalho coletivo e a inclusão das bases, elementos
característicos de organizações horizontaisinclusão das bases. Esta análise explora o
funcionamento e a estruturação dessas organizações com base em princípios geraisseu
funcionamento e estruturação. 

Organizações horizontais são frequentemente constituídas por núcleos ou
células que representam sua organização de base. Esses núcleos podem ser formados
em locais de trabalho, residências ou áreas de atividade específicas,
destacando a descentralização operativa. Essa formação permite que cada grupo atue
em conformidade com a realidade local, promovendo a ligação direta entre a
organização e suas bases. Ao contrário de organizações hierárquicas, onde as decisões
são impostas de cima para baixo, os núcleos em organizações horizontais têm
autonomia relativa para definir suas atividades, desde queSão organizadas em
núcleos ou células de base, formadas em locais de trabalho, residências ou áreas
específicas, destacando descentralização. Esses grupos atuam conforme a realidade
local e prestam contas a organismos coordenadores, equilibrando autonomia e
coesão. Diferente de organizações hierárquicas, onde as decisões são impostas, os
núcleos têm liberdade para definir atividades alinhadas aos princípios gerais do
coletivo. Essa autonomia é equilibrada pela prestação de contas a organismos
coordenadores, assegurando coesão sem comprometer a descentralização. 

Um dos pilares das organizações horizontais é a eleição dos organismos
coordenadores em todos os níveis, da base ao topo. Isso contrasta fortemente com
organizações hierárquicas, onde lideranças são muitas vezes designadas ou
hereditárias. Além disso, os membros têm o direito de expressar opiniões livremente,
criticar decisões e propor alternativas em todos os níveis da estrutura. Essas
organizações também promovem a prática da crítica e autocrítica como ferramenta para
aperfeiçoar o trabalho coletivo. Essa prática assegura que todos os membros possam
contribuir ativamente para a reflexão e elaboração das diretrizes doA eleição dos
organismos coordenadores é um pilar fundamental. Contrapondo-se a lideranças
designadas em sistemas hierárquicos, membros expressam opiniões, criticam decisões e
propõem alternativas. Crítica e autocrítica são ferramentas essenciais para
aperfeiçoar o trabalho coletivo. 

A direção coletiva é uma característica fundamental dessas organizações,
baseada no princípio de que as decisões devem refletir o consenso ou a vontade
majoritária dos membros. Embora exista uma estrutura de responsabilidade
coordenadora, essa é orientada a estimular o trabalho conjunto e evitar tendências
autoritárias ou centralistas. Para garantir a coesão, todas as decisões tomadas por
consenso ou maioria são vinculativas, o que reforça o compromisso coletivo. Esse
formato evita a divisão em facções, frequentemente observada em organizações
hierárquicas, onde disputas internas podem enfraquecer a unidadereflete o consenso ou a
maioria, estimulando trabalho conjunto e evitando centralismo. Decisões são
vinculativas, fortalecendo o compromisso coletivo e prevenindo divisões internas,
comuns em organizações hierárquicas. 

A disciplina é voluntária e baseada na aceitação coletiva das diretrizes.
Em contraste com imposições verticais em hierarquias, medidas disciplinares
horizontais são transparentes, com direito de apelação, promovendo confiança interna
e comprometimento. 

Organizações horizontais enfatizam a disciplina consciente e voluntária,
baseada na aceitação coletiva de suas diretrizes e princípios. Essa disciplina é
vista como um fator essencial para a unidade prevenindo a fragmentação interna.
Em contraste, organizações hierárquicas muitas vezes dependem de imposições
verticais para manter a ordem, o que pode gerar resistência e descontentamento. As
medidas disciplinares em organizações horizontais tendem a ser aplicadas de forma
transparente e incluem direitos de apelação, assegurando justiça processual. Essa
prática promove a confiança interna e fortalece o comprometimento dos membros. 

\section{Lacunas na Literatura}
\label{sec:introduction}

Embora a modelagem de ameaças tenha sido amplamente estudada e aplicada em
diversos contextos, ainda existem lacunas significativas na literatura, especialmente
no que diz respeito à segurança em organizações não-hierárquicas. A maioria das
ferramentas e técnicas de segurança cibernética foi desenvolvida com base em pressupostos
hierárquicos, refletindo as necessidades de entidades militares ou corporativas, onde há uma
clara cadeia de comando e responsabilidades bem definidas. No entanto, essas
tecnologias não são adequadas para setores horizontais e participativos, como cooperativas
de trabalhadores e grupos ativistas, que operam com base em processos
democráticos e coletivos. 

Uma das principais lacunas identificadas é a falta de exploração na área de
segurança horizontal, ou seja, técnicas e tecnologias de segurança que utilizam a
participação democrática para a tomada de decisões de segurança. O trabalho COLBAC:
Shifting Cybersecurity from Hierarchical to Horizontal Designs destaca a necessidade
de desenvolver tecnologias que beneficiem a comunidade limitando os privilégios
dos poderosos dentro de uma organização por meio da participação democrática. No
entanto, o estudo também aponta que a implementação de tais sistemas requer mais opções
de configuração e interações não mediadas, o que pode ser um desafio
significativo. 

A pesquisa etnográfica também pode fornecer insights valiosos sobre como essas
organizações trabalham com sistemas centralizados e hierárquicos, e como essas práticas
estabelecidas podem ser usadas para gerar métodos de design de interfaces entre COLBAC e
sistemas mais centralizados. A observação de como a introdução de técnicas de segurança
horizontais afeta a organização das comunidades pode ajudar a refletir essas mudanças em
novas tecnologias. 

Por fim, é crucial garantir que as soluções desenvolvidas sejam utilizáveis. A
criação de um sistema de segurança horizontal, ou qualquer sistema, deve considerar a
usabilidade para garantir que os membros da organização possam efetivamente utilizar e
manter o sistema sem comprometer os princípios de participação e igualdade.

Essas lacunas na literatura destacam a necessidade de mais pesquisa e
desenvolvimento na área de segurança cibernética para organizações não-hierárquicas.
Abordar esses desafios permitirá a criação de sistemas de segurança que não apenas
protejam contra ameaças externas, mas também respeitem e reforcem a estrutura participativa
dessas organizações. 
