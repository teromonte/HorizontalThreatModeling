%!TEX root = ../template.tex
%%%%%%%%%%%%%%%%%%%%%%%%%%%%%%%%%%%%%%%%%%%%%%%%%%%%%%%%%%%%%%%%%%%%
%% chapter2.tex
%% NOVA thesis document file
%%%%%%%%%%%%%%%%%%%%%%%%%%%%%%%%%%%%%%%%%%%%%%%%%%%%%%%%%%%%%%%%%%%%

\typeout{NT FILE chapter2.tex}%

\chapter{Background}
\label{cha:background}

\glsresetall

\section{Foundations of Threat Modeling}
\label{sec:foundations_threat_modeling}

Threat modeling is a central component of cybersecurity as it allows the
identification of valuable assets, analysis of potential attack vectors and
establishment of controls capable of mitigating risks
\cite{ThreatModelingASummaryOfAvailableMethods}. This practice goes beyond
technical factors incorporating organizational and human elements that shape
security in diverse contexts, especially in horizontal structures, where
internal processes and trust relationships become even more critical due to the
absence of formal hierarchies \cite{Colbac}. In a scenario of rapid
technological evolution and constant diversification of threats, a broad and
flexible approach gains relevance, meeting the particularities of changing
contexts \cite{ThreatModelingdesigningForSecurity}.

Studies such as \cite{ThreatModelingAsABasisForSecurityRequirements},
\cite{AdvancedThreatModeling}, and \cite{DemystifyingTheThreatModelingProcess}
demonstrate the need for structured and adaptable methods to keep up with
changing environments. Adopting the adversary perspective is crucial to
anticipate complex scenarios and strengthen resilience in decentralized
environments \cite{AHybridThreatModelingMethod}.

In addition, Microsoft's experience, documented in
\cite{ExperiencesThreatModelingAtMicrosoft}, emphasizes the importance of
involving diverse stakeholders and applying collaborative tools. These elements
become crucial when decision making is democratic or decentralized, since
identifying and mitigating risks requires collective engagement and strategic
flexibility, connecting the threat modeling exercise to organizational dynamics
\cite{ParticipatoryThreatModelling, ThreatModelingASummaryOfAvailableMethods}.

\subsection{Conceptual Definitions}
\label{subsec:conceptual_definitions}

Threat modeling can be understood as a systematic protection effort that
considers both technical vulnerabilities and social and organizational factors.
By adopting a holistic view, as suggested by
\cite{ThreatModelingAsABasisForSecurityRequirements} and
\cite{AdvancedThreatModeling}, security analysis is not restricted to
infrastructure, but incorporates internal practices, information flows, and the
organization's culture.

In turn, \cite{DemystifyingTheThreatModelingProcess} emphasizes that there is no
single solution for threat modeling. In this sense, threat modeling becomes an
iterative process, adapting to structural changes and incorporating innovations
such as participatory decision making tools and collaborative security practices
\cite{ParticipatoryThreatModelling}.

\subsection{Main Methodologies}
\label{subsec:main_methodologies}

Widely discussed methodologies such as \gls{stride} and
scenario based frameworks like attack trees \cite{EvaluationofCompetingThreatModeling}
provide a tested starting point, but they often fail to capture the complexity of
horizontal structures.

The documentation \cite{ThreatModelingASummaryOfAvailableMethods} provides an
overview of existing methods, cautioning that the effectiveness of each approach
depends on the context. For example, \gls{stride} and attack trees are useful
for identifying straightforward attack vectors, but there is no support
to explore complex scenarios such as insider
threats coupled with external attacks, as well as flaws in distributed
authentication, consensus, and governance mechanisms, as suggested in
\cite{STRIDEthreatmodelingforcyberphysical}.

In contrast to these approaches, \gls{pasta} methodology offers a 
risk-centric perspective \cite{RiskCentricThreatModeling}. Its seven-stage process is
structured to align security activities with business objectives,
thus shifting the focus from a simple enumeration of generic threats to a
detailed analysis of attacks that would realistically target valuable assets.
Its application is particularly justified for systems where governance and distributed trust are
not only technical features but are fundamental to the business proposition \cite{RiskCentricThreatModeling}.

\section{Taxonomy of Organizational Structures}
\label{sec:taxonomy_organizational_structures}

Understanding the relationship between organizational form and security is
essential to adjust threat modeling to the reality of each institution
\cite{Non-HierarchicalForms}. While hierarchical structures rely on central
decision points for control and coordination, these same points can become
critical vulnerabilities \cite{ThreatModelingdesigningForSecurity}. Horizontal
organizations can disperse vulnerabilities and increase resilience
through decentralized governance, although they can also create
multiple entry points that require collaborative control \cite{Colbac}. Analysis
of this taxonomy, as \cite{WorkerCooperativesinAmerica,
RealNotNominalGlobalDemocracy}, provides a basis for identifying how the
distribution of power in different organizational forms affects the
effectiveness of security measures, including the ability to respond to internal
and external threats.

\subsection{Traditional Hierarchical Structures}
\label{subsec:traditional_hierarchical_structures}

Hierarchical organizations have clear lines of authority, which facilitates
control but can concentrate vulnerabilities in critical areas
\cite{MicrosoftThreatModelingTechnique}. These organizations are characterized
by a well defined chain of command, where decisions flow from the top down.
Classic examples include large multinational corporations, where the board of
directors establishes policies that are implemented by layers of managers,
supervisors, and employees. On the other hand, in small businesses, such as
family offices, the hierarchy may be less formal but still based on a clear,
centralized command structure \cite{WorkerCooperativesinAmerica}.

In large organizations, such as banks or automotive manufacturers, hierarchy
allows for efficient allocation of resources and tight control over operations.
For example, IT departments in these environments often use security frameworks
such as \gls{stride} for threat modeling, focusing on protecting critical assets
and centralized access management \cite{MicrosoftThreatModelingTechnique,
ThreatModelingASystematicLiteratureReview}. Centralization facilitates rapid
incident response, but it also creates single points of failure, such as
vulnerabilities in key servers or administrative credentials
\cite{DoArtifactsHavePolitics, BigTech}.

In contrast, small businesses face different challenges. In these contexts, lack
of resources may lead to fewer hierarchical layers, but decisions are still
centered on a single owner or manager \cite{WorkerCooperativesandRevolution}.
This reduces organizational complexity but increases reliance on specific
individuals, making them prime targets for attacks
\cite{WorkerCooperativesinAmerica}. Furthermore, the lack of dedicated security
teams can limit the ability to implement sophisticated frameworks such as
\gls{stride}, requiring more streamlined solutions.

The difference between large and small organizations also impacts threat
modeling \cite{WorkerCooperativesinAmerica,
ThreatModelingASummaryOfAvailableMethods}. In larger organizations, hierarchical
structures allow for detailed segmentations to identify and mitigate risks at
specific levels of the organization. However, this segmentation can lead to
communication gaps between departments, making it difficult to implement
integrated solutions \cite{ThreatModelingASystematicLiteratureReview}. On the
other hand, smaller organizations have greater flexibility to quickly adapt
their security strategies, although they often lack the resources to implement
robust solutions \cite{WorkerCooperativesandRevolution}.

Therefore, while hierarchical organizations offer advantages in terms of control
and clarity, they also introduce specific challenges for threat modeling. These
challenges vary significantly with the size and complexity of the organization,
requiring adaptations to traditional frameworks to meet the specific needs of
each type of hierarchy.

\subsection{Horizontal Organizations}
\label{subsec:horizontal_organizations}

Horizontal organizations are distinguished by their rejection of traditional
hierarchies, prioritizing distributed decision making processes and equitable
participation of all members  \cite{Non-HierarchicalForms}.
This model contrasts directly with hierarchical
structures, which centralize power at higher levels, perpetuating inequalities
in access to information and organizational control \cite{Non-HierarchicalForms}.

Horizontality is both a tool and an objective in itself \cite{Colbac}. In
Argentine social movements, as analyzed by Marina Sitrin, horizontality has
emerged as an essential mechanism for establishing relationships based on trust
and consensus, overcoming traditional forms of organization \cite{EverydayRevolutions}.
Neighborhood assemblies and collectives of unemployed workers exemplify how horizontality can
be applied to self management and collective planning
\cite{EverydayRevolutions}.

In addition, historical examples, such as Athenian democracy, illustrate that
horizontal structures can be complemented by temporary centralization mechanisms
in times of crisis, ensuring flexibility and efficiency without compromising the
basic principles of distributed governance \cite{AthenianDemocracyABrief}.

Despite the challenges, such as the risk of domination by more influential
voices or the management of conflicts in collective spaces, horizontal
organizations demonstrate that, with adequate mechanisms, it is possible to
promote autonomy, inclusive participation and efficiency in decentralized
structures \cite{SocialMediaTeamsAsDigitalVanguards}.

\subsection{Leaderless Organizational Models}
\label{subsec:leaderless_organizational_models}

Leaderless organizations hides an additional complexity, where
the absence of a formal hierarchy does not necessarily imply genuine
horizontality \cite{SocialMediaTeamsAsDigitalVanguards}. Critical studies
highlight how these organizations often replicate veiled power dynamics and
informal centralizations \cite{SocialMediaTeamsAsDigitalVanguards,
EverydayRevolutions}.

Marina Sitrin, in her analysis of horizontal movements in Argentina, points out
that although horizontality is declared as an objective, many movements face
significant challenges in sustaining truly participatory practices. The lack of
formal hierarchy often leads to informal power structures, where dominant voices
assume leadership roles without oversight or clear collective responsibility
\cite{EverydayRevolutions}.

In the digital context, movements such as Occupy Wall Street demonstrate that
the absence of recognizable leadership does not eliminate internal conflicts
\cite{SocialMediaTeamsAsDigitalVanguards}. Studies of social media teams in
these movements reveal that account management, as on Twitter, was often marked
by struggles for control, illustrating how power and influence can consolidate
even in supposedly horizontal structures
\cite{SocialMediaTeamsAsDigitalVanguards}.

Furthermore, research on worker cooperatives in the United States indicates that
these organizations, although often seen as non-hierarchical alternatives, tend
to develop informal leaders who influence decisions in significant ways,
challenging the narrative of absolute horizontality
\cite{WorkerCooperativesandRevolution}.

Technologies used by these organizations also carry political implications
\cite{DoArtifactsHavePolitics, Democraciaeoscodigosinvisiveis}. Langdon Winner
argues that technical artifacts can perpetuate existing power structures, even
when employed in decentralized contexts \cite{DoArtifactsHavePolitics}. For
example, digital platforms, often designed for individual use, create challenges
in building effective collective governance, exacerbating latent inequalities
\cite{DoArtifactsHavePolitics, BigTech}.

These examples highlight that while the idea of formal leadershiplessness is
appealing, its practical execution often results in informal forms of hierarchy
\cite{EverydayRevolutions, SocialMediaTeamsAsDigitalVanguards}. Thus, the
success of these organizations depends on the ability to identify and mitigate
hidden power dynamics, promoting clear mechanisms of collective governance and
mutual accountability that truly sustain the desired horizontality
\cite{Colbac}.

\section{Democratic Centralism}
\label{sec:democratic_centralism}

In the middle of increasingly challenging scenarios in terms of organizational
coordination, whether in the management of large or smaller corporations, the need arises to
reconcile efficiency in decision making with the active participation of all
those involved \cite{DoArtifactsHavePolitics}. Democratic centralism, formulated
to meet the demands of revolutionary movements, maintains its relevance by
proposing a dynamic balance between collective deliberation and centralized
execution \cite{EstatutosDoPCP}. This approach has proven to be relevant both in
political and social contexts and in contemporary technological scenarios
\cite{Colbac}.

\subsection{Fundamental Principles and Theoretical Origins}
\label{subsec:fundamental_principles_origins}

Democratic centralism is based on the idea that opinion gathering and
deliberation (democracy) must be harmonized with the ability to implement
decisions in a unified manner (centralism)
\cite{ACenturyofDemocraticCentralism}. This system was originally conceived as a
response to the need for organization in contexts of high structural complexity
\cite{StillaCenturyoftheChineseModel}.

Its initial formulation, associated with the Communist Party of the Soviet
Union, inspired the adoption of the model by different groups around the world
\cite{StillaCenturyoftheChineseModel}. In China, for example, the practices of democratic
centralism demonstrate remarkable resilience, being reconfigured as political
and social situations change \cite{ACenturyofDemocraticCentralism}.

\subsection{Contemporary Models of Application}
\label{subsec:contemporary_models_application}

The transposition of the precepts of democratic centralism to modern contexts
has been observed in different organizational and technological structures
\cite{DoArtifactsHavePolitics, Colbac}. One example is the \gls{colbac}
protocol, which adapts the bases of democratic centralism to a collaborative
access control environment, enabling participatory decisions combined with
cohesive implementation \cite{Colbac}.

Trade unions and social organizations have also integrated principles of
democratic centralism \cite{CGTPStatutes}.

\subsection{Implications and Potentials for Governance}
\label{subsec:implications_potentials_governance}

The influence of democratic centralism transcends the limits of political
parties and can be extended to diverse scenarios that require both broad
participation and efficient execution \cite{ACenturyofDemocraticCentralism,
TheCostsofConnection}. Horizontal organizations can use this
model to reconcile the voice of their members with the need to make centralized
decisions at critical moments
\cite{StillaCenturyoftheChineseModel,ACenturyofDemocraticCentralism}.

In this sense, the application of elements of democratic centralism in digital
platforms highlights how historical concepts can be reappropriated to meet
contemporary governance demands \cite{Colbac}.

\section*{} 
Chapter 2 presented the theoretical foundations related to threat modeling,
addressing structured and flexible methods needed to deal with decentralized
organizational contexts. Approaches that incorporate adversarial perspectives
were discussed and emphasized the importance of considering social and
organizational factors, in addition to technical vulnerabilities, highlighting
especially the relevance of collaboration between different actors in the
process. Building on these fundamental concepts, Chapter 3 will analyze related
works, describing in detail traditional threat modeling approaches, such as
STRIDE, and exploring their applications and limitations in decentralized and
collaborative environments.
