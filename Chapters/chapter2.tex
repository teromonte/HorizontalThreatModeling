%!TEX root = ../template.tex
%%%%%%%%%%%%%%%%%%%%%%%%%%%%%%%%%%%%%%%%%%%%%%%%%%%%%%%%%%%%%%%%%%%%
%% chapter2.tex
%% NOVA thesis document file
%%
%% Chapter with the template manual
%%%%%%%%%%%%%%%%%%%%%%%%%%%%%%%%%%%%%%%%%%%%%%%%%%%%%%%%%%%%%%%%%%%%

\typeout{NT FILE chapter2.tex}%

\chapter{Background}
\label{cha:background}

\glsresetall

\section{Fundamentos da Modelagem de Ameaças}
\label{sec:fundamentos_da_modelagem_de_ameacas}

A modelagem de ameaças é um componente central da
cibersegurança, pois permite identificar ativos valiosos,
analisar potenciais vetores de ataque e estabelecer
controles capazes de mitigar riscos. Esse processo não se
restringe a fatores técnicos: ele integra também elementos
organizacionais e humanos, tornando a avaliação de segurança
mais ampla e coerente com a realidade das organizações. A
relevância desse tema torna-se evidente no cenário atual,
marcado pela rápida evolução tecnológica e pela
diversificação constante das ameaças.

Estudos como
\cite{ThreatModelingAsABasisForSecurityRequirements},
\cite{AdvancedThreatModeling} e
\cite{DemystifyingTheThreatModelingProcess} demonstram que
uma abordagem estruturada e flexível é essencial para
acompanhar ambientes em constante mudança. Ao adotar a
perspectiva do adversário, conforme sugerido em
\cite{AHybridThreatModelingMethod}, antecipam-se cenários
complexos, fortalecendo a resiliência dos sistemas e
proporcionando uma visão mais realista das vulnerabilidades
a serem tratadas. Esse ponto é especialmente relevante em
organizações que buscam maior autonomia decisória e
flexibilidade estrutural.

Adicionalmente, a experiência da Microsoft, documentada em
\cite{ExperiencesThreatModelingAtMicrosoft}, destaca a
importância de envolver uma ampla gama de stakeholders e a
utilização de ferramentas colaborativas para identificar
riscos emergentes. Esses elementos são cruciais para
endereçar não apenas vulnerabilidades técnicas, mas também
questões organizacionais e humanas que impactam a segurança.

\subsection{Definições Conceituais}
\label{sec:definicoes_conceituais}

A modelagem de ameaças pode ser entendida como um esforço
sistemático de proteção, no qual se mapeiam pontos fracos e
se antecipam possíveis ataques a partir da análise do
comportamento adversário. Dessa forma, além do contexto
puramente técnico, aspectos organizacionais e humanos são
contemplados, conferindo uma perspectiva holística que
auxilia na identificação e mitigação de riscos emergentes.
Trabalhos como
\cite{ThreatModelingAsABasisForSecurityRequirements} e
\cite{AdvancedThreatModeling} reforçam a importância desse
olhar abrangente, enquanto
\cite{DemystifyingTheThreatModelingProcess} destaca a
necessidade de metodologias flexíveis, capazes de acompanhar
a evolução tecnológica.

\subsection{Principais Metodologias}
\label{sec:principais_metodologias}

Entre as metodologias amplamente discutidas estão o
\gls{stride}, as árvores de ataque e os frameworks baseados
em cenários \cite{EvaluationofCompetingThreatModeling}. A
participação ativa de stakeholders no processo de modelagem,
sugerida em \cite{ParticipatoryThreatModelling}, mostra-se
especialmente importante em ambientes menos hierarquizados,
incentivando maior engajamento e responsabilidade coletiva.

A documentação
\cite{ThreatModelingASummaryOfAvailableMethods} fornece uma
visão geral dos métodos
existentes, destacando a necessidade de adaptação das
ferramentas de acordo com o contexto da organização. Em
particular, os frameworks baseados em cenários permitem
identificar padrões de ataque menos óbvios, incluindo
ameaças internas combinadas com ataques externos.

Para ilustrar seu uso, considere duas situações práticas. Na
primeira, uma startup de tecnologia baseada em microserviços
emprega o \gls{stride} para identificar vetores de ataque,
descobrindo que elevação de privilégios e negação de serviço
são riscos-chave. Como resposta, implementam-se controles de
acesso mais robustos e mecanismos de limitação de tráfego.

Na segunda, uma empresa do setor financeiro recorre às
árvores de ataque para analisar ações potenciais de um
invasor interessado em comprometer sistemas de autenticação
de múltiplos fatores. Ao decompor objetivos e passos
necessários ao ataque, a organização consegue estabelecer
barreiras específicas em cada etapa, aumentando assim a
segurança do sistema.

Da mesma forma, frameworks baseados em cenários permitem
simular situações complexas, como ataques internos e
externos simultâneos, fornecendo insights sobre a
resiliência e a prontidão da empresa diante de eventos
adversos.

A integração de diferentes metodologias e o uso de
criptografia colaborativa \cite{AbcCrypto} ou abordagens
híbridas \cite{CoReTM} potencializam a eficácia da modelagem
de ameaças. Para facilitar a consulta, uma tabela resumo
poderia ser incluída, comparando o \gls{stride}, as árvores
de ataque e os frameworks de cenários em relação a escopo,
complexidade e tipos de ameaças endereçadas.

\section{Taxonomia de Estruturas Organizacionais}
\label{sec:taxonomia_de_estruturas_organizacionais}

Compreender como diferentes estruturas organizacionais
influenciam as práticas de segurança é essencial para
adaptar estratégias de modelagem de ameaças a contextos
específicos. Nesta seção, analisamos tipologias de
organizações e suas implicações para a mitigação de riscos.
Ao entender as peculiaridades de estruturas hierárquicas,
horizontais ou sem liderança, é possível direcionar
abordagens de segurança mais ajustadas ao perfil da
organização (ver \cite{WorkerCooperativesinAmerica} e
\cite{RealNotNominalGlobalDemocracy}).

\subsection{Estruturas Tradicionais Hierárquicas}
\label{sec:estruturas_tradicionais_hierarquicas}

Organizações com estruturas hierárquicas bem definidas
contam com linhas claras de autoridade e comunicação. Essa
abordagem facilita a delegação e o controle, mas pode, ao
mesmo tempo, concentrar vulnerabilidades. Por exemplo, o
comprometimento de um ponto de decisão-chave pode ter
impacto desproporcional sobre todo o sistema. Estudos como
\cite{ThreatModelingASystematicLiteratureReview} e
\cite{DoArtifactsHavePolitics} mostram que hierarquias
rígidas podem reforçar desigualdades de poder e expor áreas
críticas a ameaças internas.

\subsection{Organizações Horizontais}
\label{sec:organizacoes_horizontais}

Organizações horizontais, caracterizadas pela autonomia
individual e pela ausência de hierarquias rígidas
\cite{Reputation-basedDAO,
SocialMediaTeamsAsDigitalVanguards}, favorecem a inovação e
a flexibilidade. Entretanto, a coordenação de recursos e a
tomada de decisão coletiva representam desafios. Nesse
contexto, tecnologias descentralizadas, como o blockchain,
podem fornecer mecanismos de verificação independentes,
aumentando a segurança e a transparência.

Considere uma cooperativa de produtores de software
open-source. Nesse ambiente, cada membro contribui com
código e revisões, e a segurança é garantida por uma
verificação contínua da comunidade. A ausência de um líder
formal não significa falta de governança: regras de
contribuição e protocolos de segurança são acordados
coletivamente, e a confiança é reforçada por auditorias
públicas do código e sistemas de reputação. Esses arranjos
assemelham-se em muitos aspectos a uma \gls{dao}, pois
dependem de mecanismos descentralizados para coordenação e
controle. Estudos \cite{AbcCrypto} sugerem que criptografia
colaborativa e abordagens participativas fortalecem a
resiliência em tais contextos, reduzindo a dependência de
pontos únicos de falha.

Além disso, a experiência de iniciativas educacionais, como
as analisadas em \cite{EscoladaPonte}, indica que
estruturas horizontais podem fomentar maior engajamento e
autonomia individual, contribuindo também para a
sustentabilidade das próprias dinâmicas organizacionais.

\subsection{Modelos Organizacionais Sem Liderança}
\label{sec:modelos_sem_lideranca}

Em modelos sem liderança formal, a governança distribuída
apoia-se em algoritmos de reputação, sistemas colaborativos
e processos decisórios transparentes. Conforme analisado em
\cite{FromCounterpublicstoContentious, EverydayRevolutions},
esses arranjos podem mitigar ameaças complexas ao distribuir
responsabilidades de modo equilibrado. Ao invés de depender
de uma figura central, a segurança emerge do próprio tecido
organizacional, que integra continuamente a perspectiva
adversária em sua dinâmica interna, aumentando a resiliência
de forma natural e adaptativa.

Experiências relatadas em \cite{Non-HierarchicalForms}
mostram
que modelos sem liderança podem implementar práticas
inovadoras de segurança, como redes descentralizadas de
resposta a ameaças, que se ajustam dinamicamente às
necessidades do ambiente, promovendo robustez frente a
ataques inesperados.

\section{Centralismo Democrático}
\label{sec:centralismo_democratico}

Ao analisar as diferentes estruturas organizacionais e suas
implicações para a segurança, é útil considerar teorias que
equilibram participação interna e eficiência operacional. O
centralismo democrático, abordado em \cite{EstatutosDoPCP},
\cite{CGTPStatutes} e \cite{StillaCenturyoftheChineseModel},
propõe um modelo de governança que busca harmonizar a tomada
de decisão coletiva com mecanismos de coordenação bem
definidos.

\subsection{Origens Teóricas}
\label{sec:origens_teoricas}

A teoria do centralismo democrático, descrita em
\cite{EverydayRevolutions}, procura alinhar a eficiência da
centralização com a legitimidade da participação. Nesse
sentido, o processo decisório inclui o debate interno e a
participação ativa dos membros, seguido pela implementação
disciplinada das decisões, visando ao bem-estar coletivo.
Por exemplo, o \gls{pcp} adota essa abordagem, como indicado
em \cite{EstatutosDoPCP}.

\subsection{Aplicações Contemporâneas}
\label{sec:aplicacoes_contemporaneas}

Em contextos contemporâneos, o centralismo democrático pode
ser adaptado a formatos digitais e distribuídos. Por
exemplo, \cite{CreatingTheCollectiveSocialMedia} explora a
coordenação de equipes dispersas por meio de ferramentas
on-line, enquanto \cite{Colbac} sugere o uso de tecnologias
emergentes para reforçar a disciplina interna sem inibir a
inovação. Dessa maneira, o modelo preserva sua capacidade de
resposta, mesmo quando aplicado a redes complexas. Em âmbito
sindical, a \gls{cgtp} também oferece um exemplo de como
princípios similares podem estruturar a tomada de decisão
coletiva \cite{CGTPStatutes}.

\subsection{Implicações para Governança}
\label{sec:implicacoes_para_governanca}

As implicações do centralismo democrático para a segurança
incluem a necessidade de processos decisórios transparentes,
confiança distribuída e mecanismos reputacionais para aferir
a confiabilidade dos participantes
\cite{Reputation-basedDAO}. Tecnologias mencionadas em
\cite{AbcCrypto} podem fornecer ferramentas criptográficas
que reforçam a segurança, garantindo a resiliência dos
sistemas em face de ameaças externas e internas.

\section*{Considerações Finais}

Ao longo deste capítulo, foram apresentados conceitos,
metodologias e estruturas organizacionais que se relacionam
com a modelagem de ameaças. Vimos que não existe uma
abordagem única capaz de atender a todos os contextos: a
escolha das metodologias e ferramentas depende das
características específicas da organização, da natureza das
ameaças enfrentadas e do grau de centralização ou
descentralização do poder decisório. Modelos híbridos, que
combinam flexibilidade técnica com participação distribuída,
emergem como alternativas promissoras, especialmente em
ambientes onde a colaboração e a resiliência são ativos
estratégicos. Dessa forma, as organizações podem adaptar
suas estratégias de segurança de maneira eficiente e
alinhada às demandas contemporâneas.