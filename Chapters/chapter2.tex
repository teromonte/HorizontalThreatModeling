%!TEX root = ../template.tex
%%%%%%%%%%%%%%%%%%%%%%%%%%%%%%%%%%%%%%%%%%%%%%%%%%%%%%%%%%%%%%%%%%%%
%% chapter2.tex
%% NOVA thesis document file
%%%%%%%%%%%%%%%%%%%%%%%%%%%%%%%%%%%%%%%%%%%%%%%%%%%%%%%%%%%%%%%%%%%%

\typeout{NT FILE chapter2.tex}%

\chapter{Background}
\label{cha:background}

\glsresetall

\section{Fundamentos da Modelagem de Ameaças}
\label{sec:fundamentos_da_modelagem_de_ameacas}

A modelagem de ameaças é um componente central da cibersegurança, pois
permite identificar ativos valiosos, analisar potenciais vetores de
ataque e estabelecer controles capazes de mitigar riscos. Essa prática
vai além de fatores técnicos, incorporando elementos organizacionais e
humanos que moldam a segurança em contextos diversos, especialmente em
estruturas horizontais, onde processos internos e relações de confiança
se tornam ainda mais críticos devido à ausência de hierarquias formais \cite{Colbac}.
Em um cenário de rápida evolução tecnológica e diversificação constante
das ameaças, uma abordagem ampla e flexível ganha relevância, atendendo
às particularidades de contextos em transformação \cite{ThreatModelingdesigningForSecurity}.

Estudos como \cite{ThreatModelingAsABasisForSecurityRequirements},
\cite{AdvancedThreatModeling} e
\cite{DemystifyingTheThreatModelingProcess} demonstram a necessidade
de métodos estruturados, porém adaptáveis, para acompanhar ambientes
em mutação. A adoção da perspectiva do adversário \cite{AHybridThreatModelingMethod}
é crucial para antecipar cenários complexos e fortalecer a resiliência
em ambientes descentralizados, onde a multiplicidade de atores e a
distribuição de poder requerem uma análise holística das ameaças.
Nesse tipo de contexto, a modelagem de ameaças precisa refletir
a distribuição de poder e a multiplicidade de atores, incluindo as
possíveis ameaças internas, externas e híbridas \cite{ThreatModelingASummaryOfAvailableMethods}.

Além disso, a experiência da Microsoft, documentada em
\cite{ExperiencesThreatModelingAtMicrosoft}, enfatiza a importância de
envolver stakeholders diversos e de aplicar ferramentas colaborativas.
Esses elementos tornam-se cruciais quando a tomada de decisão é
democrática ou descentralizada, pois a identificação e mitigação de
riscos demandam engajamento coletivo e flexibilidade estratégica,
conectando o exercício de modelagem de ameaças às dinâmicas
organizacionais \cite{ParticipatoryThreatModelling}.

\subsection{Definições Conceituais} 
\label{sec:definicoes_conceituais}

A modelagem de ameaças pode ser entendida como um esforço sistemático de
proteção que considera tanto vulnerabilidades técnicas quanto fatores
sociais e organizacionais. Ao adotar uma visão holística, conforme sugerem
\cite{ThreatModelingAsABasisForSecurityRequirements} e
\cite{AdvancedThreatModeling}, a análise de segurança não fica restrita à
infraestrutura, mas incorpora práticas internas, fluxos de informação e a
cultura da organização. Em contextos não-hierárquicos, a ausência de linhas
claras de autoridade e o caráter participativo tornam essencial uma análise
que considere a distribuição de responsabilidades e a dependência de
mecanismos coletivos de mitigação \cite{Colbac}.

Por sua vez, \cite{DemystifyingTheThreatModelingProcess} enfatiza que não
existe uma solução única para modelagem de ameaças. Nesse sentido,a
modelagem de ameaças torna-se um processo iterativo, adaptando-se a
alterações estruturais e incorporando inovações como ferramentas de
tomada de decisão participativa e práticas de segurança colaborativa
\cite{SecurityCardsToolkit, ParticipatoryThreatModelling}.

\subsection{Principais Metodologias}
\label{sec:principais_metodologias}

Metodologias amplamente discutidas, como o \gls{stride}, as árvores de
ataque e frameworks baseados em cenários
\cite{EvaluationofCompetingThreatModeling}, fornecem um ponto de partida
testado, mas frequentemente não capturam a complexidade de estruturas
horizontais.

A documentação \cite{ThreatModelingASummaryOfAvailableMethods} apresenta um
panorama de métodos existentes, alertando que a eficácia de cada abordagem
depende do contexto. Por exemplo, \gls{stride} e árvores de ataque são
úteis para identificar vetores de ataque diretos, mas a ausência de
hierarquias formais intensifica a necessidade de explorar cenários
complexos, como ameaças internas associadas a ataques externos, bem como
falhas em mecanismos distribuídos de autenticação, consenso e governança,
como sugerido em \cite{Colbac, AttackTrees, STRIDEthreatmodelingforcyberphysical}.

A integração de métodos diversos, como práticas de criptografia
colaborativa \cite{AbcCrypto} e abordagens híbridas \cite{CoReTM}, permite
que organizações horizontais identifiquem padrões de risco menos óbvios e
fortaleçam sua resiliência coletiva. Essa integração se torna crucial em
estruturas distribuídas, onde a ausência de um 'centro' transforma a
segurança em um esforço coletivo, e a resiliência emerge da interação
contínua entre membros, sistemas e mecanismos de governança
descentralizada \cite{Reputation-basedDAO}.

\section{Taxonomia de Estruturas Organizacionais}
\label{sec:taxonomia_de_estruturas_organizacionais}

Entender a relação entre forma organizacional e segurança é essencial para
ajustar a modelagem de ameaças à realidade de cada instituição \cite{Non-HierarchicalForms}.
Enquanto estruturas hierárquicas confiam em pontos centrais de decisão para controle
e coordenação, esses mesmos pontos podem se tornar vulnerabilidades
críticas \cite{ThreatModelingdesigningForSecurity}. Organizações horizontais ou cooperativas podem
dispersar vulnerabilidades e aumentar a resiliência por meio da governança
descentralizada, embora também possam criar múltiplos pontos de entrada que
exigem controle colaborativo \cite{Colbac}. A análise desta taxonomia, conforme
\cite{WorkerCooperativesinAmerica, RealNotNominalGlobalDemocracy}, oferece
uma base para identificar como a distribuição de poder em diferentes formas
organizacionais afeta a eficácia de medidas de segurança, incluindo a
capacidade de resposta a ameaças internas e externas.

\subsection{Estruturas Tradicionais Hierárquicas}
\label{sec:estruturas_tradicionais_hierarquicas}

Organizações hierárquicas apresentam linhas claras de autoridade, o que
facilita o controle, mas pode concentrar vulnerabilidades em pontos
críticos \cite{MicrosoftThreatModelingTechnique}. Essas organizações são caracterizadas por uma cadeia de comando
bem definida, onde as decisões fluem do topo para a base. Exemplos
clássicos incluem grandes corporações multinacionais, onde a diretoria
estabelece políticas que são implementadas por camadas de gerentes,
supervisores e funcionários. Por outro lado, em pequenas empresas, como
escritórios familiares, a hierarquia pode ser menos formal, mas ainda assim
baseada em uma estrutura de comando clara e centralizada \cite{WorkerCooperativesinAmerica}.

Em grandes organizações, como bancos ou indústrias automotivas, a
hierarquia permite uma alocação eficiente de recursos e um controle
rigoroso sobre as operações. Por exemplo, as divisões de TI desses
ambientes frequentemente utilizam frameworks de segurança como o
\gls{stride} para modelagem de ameaças, focando na proteção de ativos
críticos e no gerenciamento de acessos centralizados
\cite{MicrosoftThreatModelingTechnique, ThreatModelingASystematicLiteratureReview}.
A centralização facilita a resposta rápida a incidentes, mas também cria pontos únicos de falha,
como vulnerabilidades em servidores principais ou credenciais
administrativas \cite{DoArtifactsHavePolitics, BigTech}.

Em contrapartida, pequenas empresas enfrentam desafios diferentes. Nesses
contextos, a falta de recursos pode levar a menos camadas hierárquicas, mas
as decisões ainda se concentram em um único proprietário ou gerente \cite{WorkerCooperativesandRevolution}.
Isso reduz a complexidade organizacional, mas aumenta a dependência de
indivíduos específicos, tornando-os alvos prioritários em ataques
\cite{WorkerCooperativesinAmerica}. Ademais, a ausência de equipes
dedicadas de segurança pode limitar a capacidade de implementar frameworks
sofisticados, como \gls{stride}, exigindo soluções mais simplificadas.

A diferença entre organizações grandes e pequenas também reflete no impacto
sobre a modelagem de ameaças \cite{WorkerCooperativesinAmerica, ThreatModelingASummaryOfAvailableMethods}.
Em empresas maiores, as estruturas hierárquicas permitem segmentações detalhadas para identificar e mitigar
riscos em níveis específicos da organização. No entanto, essa segmentação
pode levar a lacunas de comunicação entre departamentos, dificultando a
implementação de soluções integradas
\cite{ThreatModelingASystematicLiteratureReview}. Por outro lado,
organizações menores têm maior flexibilidade para adaptar rapidamente suas
estratégias de segurança, embora frequentemente careçam de recursos para
implementar soluções robustas \cite{WorkerCooperativesandRevolution}.

Portanto, enquanto organizações hierárquicas oferecem vantagens em termos
de controle e clareza, elas também introduzem desafios específicos para a
modelagem de ameaças. Esses desafios variam significativamente com o
tamanho e a complexidade da organização, exigindo adaptações nos frameworks
tradicionais para atender às necessidades específicas de cada tipo de
hierarquia.

\subsection{Organizações Horizontais}
\label{sec:organizacoes_horizontais}

Organizações horizontais se distinguem pela rejeição de hierarquias
tradicionais, priorizando processos decisórios distribuídos e participação
equitativa de todos os membros. Este modelo contrasta diretamente com
estruturas hierárquicas, que centralizam o poder em níveis superiores,
perpetuando desigualdades no acesso à informação e controle organizacional
\cite{Non-HierarchicalForms, EstatutosDoPCP}.

A horizontalidade é tanto uma ferramenta quanto um objetivo em si \cite{Colbac}. Nos
movimentos sociais argentinos, como analisado por Marina Sitrin, a
horizontalidade emergiu como um mecanismo essencial para estabelecer
relações baseadas na confiança e no consenso, superando formas tradicionais
de organização. Assembleias de bairro e coletivos de trabalhadores
desempregados exemplificam como a horizontalidade pode ser aplicada para
autogestão e planejamento coletivo \cite{EverydayRevolutions}.

No campo da cibernética, o protocolo COLBAC demonstra a relevância da
horizontalidade em sistemas de segurança digital, promovendo um controle de
acesso colaborativo que reduz a centralização de poder. Este modelo evita
as vulnerabilidades criadas pela dependência de proprietários únicos de
senhas ou permissões, reforçando a coerência entre práticas organizacionais
e ferramentas tecnológicas \cite{Colbac}.

Adicionalmente, exemplos históricos, como a democracia ateniense, ilustram
que estruturas horizontais podem ser complementadas por mecanismos
temporários de centralização em momentos de crise, garantindo flexibilidade
e eficiência sem comprometer os princípios básicos da governança
distribuída \cite{AthenianDemocracyABrief}.

Apesar dos desafios, como o risco de dominação por vozes mais influentes ou
a gestão de conflitos em espaços coletivos, as organizações horizontais
demonstram que, com mecanismos adequados, é possível promover autonomia,
participação inclusiva e eficiência em estruturas descentralizadas
\cite{SocialMediaTeamsAsDigitalVanguards}.

\subsection{Modelos Organizacionais Sem Liderança}
\label{sec:modelos_sem_lideranca}

O discurso de organizações sem liderança esconde uma complexidade
adicional, onde a ausência de uma hierarquia formal não implica
necessariamente uma horizontalidade genuína \cite{SocialMediaTeamsAsDigitalVanguards}.
Estudos críticos destacam como essas organizações frequentemente
replicam dinâmicas de poder veladas e centralizações informais
\cite{SocialMediaTeamsAsDigitalVanguards, EverydayRevolutions}.

Marina Sitrin, em sua análise sobre movimentos horizontais na Argentina,
aponta que, embora a horizontalidade seja declarada como objetivo, muitos
movimentos enfrentam desafios significativos para sustentar práticas
realmente participativas. A falta de hierarquia formal frequentemente leva
a estruturas de poder informais, onde vozes dominantes assumem papéis de
liderança sem supervisão ou responsabilidade coletiva clara
\cite{EverydayRevolutions}.

No contexto digital, movimentos como Occupy Wall Street demonstram que a
ausência de uma liderança reconhecível não elimina conflitos internos
\cite{SocialMediaTeamsAsDigitalVanguards}.
Estudos sobre as equipes de mídia social desses movimentos revelam que a
administração de contas, como no Twitter, foi frequentemente marcada por
disputas de controle, ilustrando como poder e influência podem se
consolidar mesmo em estruturas supostamente horizontais
\cite{SocialMediaTeamsAsDigitalVanguards}.

Além disso, pesquisas sobre cooperativas de trabalhadores nos Estados
Unidos indicam que essas organizações, embora frequentemente vistas como
alternativas não hierárquicas, tendem a desenvolver líderes informais que
influenciam decisões de maneira significativa, questionando a narrativa de
horizontalidade absoluta \cite{WorkerCooperativesandRevolution}.

Tecnologias utilizadas por essas organizações também carregam implicações
políticas \cite{DoArtifactsHavePolitics, Democraciaeoscodigosinvisiveis}.
Langdon Winner argumenta que artefatos técnicos podem perpetuar estruturas de poder
existentes, mesmo quando empregados em contextos
descentralizados \cite{DoArtifactsHavePolitics}. Por exemplo, plataformas digitais,
muitas vezes projetadas para usos individuais, criam desafios na construção de
governança coletiva efetiva, exacerbando desigualdades latentes
\cite{DoArtifactsHavePolitics, BigTech}.

Esses exemplos destacam que, embora a ideia de ausência de liderança formal
seja atraente, sua execução prática frequentemente resulta em formas
informais de hierarquia \cite{EverydayRevolutions, SocialMediaTeamsAsDigitalVanguards}.
Assim, o sucesso dessas organizações depende da
capacidade de identificar e mitigar as dinâmicas de poder ocultas,
promovendo mecanismos claros de governança coletiva e responsabilidade
mútua que realmente sustentem a horizontalidade desejada \cite{Colbac}.

\section{Centralismo Democrático}
\label{sec:centralismo_democratico}

Em meio a cenários cada vez mais desafiadores em termos de coordenação
organizacional, seja na gestão de grandes corporações ou na manutenção
de redes descentralizadas como a blockchain, surge a necessidade de se
conciliar a eficiência na tomada de decisões com a participação ativa
de todos os envolvidos \cite{DoArtifactsHavePolitics}.
O centralismo democrático, formulado para
atender às exigências de movimentos revolucionários, mantém sua
relevância ao propor um equilíbrio dinâmico entre deliberação coletiva
e execução centralizada \cite{EstatutosDoPCP}. Essa abordagem tem se mostrado pertinente
tanto em contextos políticos e sociais quanto em cenários tecnológicos
contemporâneos \cite{Colbac}.

\subsection{Princípios Fundamentais e Origens Teóricas}
\label{sec:principios_origens_teoricas}

O centralismo democrático fundamenta-se na ideia de que a coleta de
opiniões e a deliberação (democracia) devem ser harmonizadas com a
capacidade de implementar decisões de forma unificada (centralismo)
\cite{ACenturyofDemocraticCentralism}.
Esse sistema foi originalmente concebido como resposta à necessidade
de organização em contextos de alta complexidade estrutural 
\cite{StillaCenturyoftheChineseModel,EstatutosDoPCP}.

Sua formulação inicial, associada ao Partido Comunista da União
Soviética, inspirou a adoção do modelo por diferentes grupos ao redor
do mundo \cite{EstatutosDoPCP}. Na China, por exemplo, as práticas de centralismo
democrático demonstram uma resiliência notável, sendo reconfiguradas à
medida que variam as conjunturas políticas e sociais
\cite{ACenturyofDemocraticCentralism}.

\subsection{Modelos Contemporâneos de Aplicação}
\label{sec:modelos_contemporaneos}

A transposição dos preceitos do centralismo democrático para contextos
modernos tem sido observada em diferentes estruturas organizacionais e
tecnológicas \cite{DoArtifactsHavePolitics, Colbac}.
Um exemplo é o protocolo \gls{colbac}, que adapta as bases do
centralismo democrático a um ambiente de controle de acesso
colaborativo, possibilitando decisões participativas aliadas a uma
implementação coesa \cite{Colbac}.

Movimentos sindicais e organizações sociais também
têm integrado princípios de centralismo democrático \cite{CGTPStatutes}. A
Confederação Geral dos Trabalhadores Portugueses, por exemplo,
combina deliberação coletiva e períodos de centralização estratégica
para dar respostas eficazes a desafios organizacionais \cite{CGTPStatutes}.

\subsection{Implicações e Potenciais para Governança}
\label{sec:implicacoes_potenciais}

A influência do centralismo democrático transcende os limites dos
partidos políticos, podendo ser estendida a cenários diversos que
exijam tanto participação ampla quanto execução eficiente
\cite{ACenturyofDemocraticCentralism, TheCostsofConnection}.
Organizações horizontais ou distribuídas podem recorrer a esse modelo
para conciliar a voz de seus integrantes com a necessidade de tomar
decisões de forma centralizada em momentos críticos
\cite{StillaCenturyoftheChineseModel,ACenturyofDemocraticCentralism}.

Nesse sentido, a aplicação de elementos do centralismo democrático em
plataformas digitais evidencia como conceitos históricos podem ser
reapropriados para atender às demandas de governança contemporâneas \cite{Colbac}.