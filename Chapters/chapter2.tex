%!TEX root = ../template.tex
%%%%%%%%%%%%%%%%%%%%%%%%%%%%%%%%%%%%%%%%%%%%%%%%%%%%%%%%%%%%%%%%%%%%
%% chapter2.tex
%% NOVA thesis document file
%%
%% Chapter with the template manual
%%%%%%%%%%%%%%%%%%%%%%%%%%%%%%%%%%%%%%%%%%%%%%%%%%%%%%%%%%%%%%%%%%%%

\typeout{NT FILE chapter2.tex}%

\chapter{Background and Related Work}
\label{cha:background}

\glsresetall

\section{Modelagem de Ameaças: Conceitos}
\label{sec:introduction}

A modelagem de ameaças é um processo estruturado que visa identificar,
analisar e mitigar riscos de segurança em sistemas e aplicações. Na sua base, um
modelo de ameaças genérico oferece uma abordagem sistemática para entender as
vulnerabilidades potenciais de um sistema e as possíveis ações que os adversários podem
tomar para explorá-las. Ele serve tanto como uma ferramenta de design quanto um
quadro de avaliação, apoiando o desenvolvimento seguro de sistemas e a mitigação
proativa de riscos.  

O processo geralmente começa com a caracterização do sistema em análise.
Isso envolve definir sua arquitetura, identificar ativos críticos e mapear o
fluxo de dados através de seus componentes. Um entendimento completo das
fronteiras de confiança — pontos onde os dados transitam entre áreas com diferentes
níveis de segurança — é essencial. Essas fronteiras frequentemente destacam áreas
de vulnerabilidade aumentada, onde as ações adversárias são mais prováveis de
ocorrer. 

Uma vez que o sistema é mapeado, o foco se desloca para a identificação de
ameaças potenciais. Técnicas como STRIDE fornecem uma estrutura para categorizar
sistematicamente as ameaças. Alternativamente, representações gráficas como árvores de
ataque mapeiam visualmente caminhos de ataque potenciais, oferecendo uma maneira
intuitiva de avaliar vulnerabilidades. 

Após as ameaças serem identificadas, seu impacto potencial é analisado.
Esta análise envolve avaliar a probabilidade de cada ameaça e a gravidade de
seu impacto na funcionalidade e segurança do sistema. Ferramentas que variam
de métodos qualitativos, como revisões de especialistas, a modelos formais,
como redes de Petri, apoiam esta fase. O objetivo é priorizar riscos,
garantindo que os esforços de mitigação se concentrem nas vulnerabilidades mais
críticas. 

Estratégias de mitigação são então desenvolvidas para abordar os riscos
identificados. Essas estratégias podem incluir controles técnicos, como criptografia ou
gestão de acesso, bem como medidas organizacionais como políticas e treinamentos.
Um modelo de ameaças genérico também inclui loops de feedback, garantindo
que, à medida que o sistema evolui, o modelo de ameaças seja atualizado para
refletir novas vulnerabilidades e ameaças. 

Embora os métodos de modelagem de ameaças variem amplamente, eles
compartilham um foco comum na análise metódica e defesa proativa. Seja aplicado
manualmente ou com ferramentas automatizadas, o processo fornece uma visão abrangente
dos riscos potenciais, permitindo que desenvolvedores e profissionais de
segurança abordem proativamente as vulnerabilidades antes que possam ser exploradas.

\subsection{STRIDE}
\label{sec:STRIDE}

A metodologia STRIDE é uma das mais maduras e amplamente utilizadas para
modelagem de ameaças. Desenvolvida por Loren Kohnfelder e Praerit Garg em 1999 e adotada
pela Microsoft em 2002, STRIDE evoluiu ao longo do tempo para incluir novas tabelas
específicas de ameaças e variantes como STRIDE-per-Element e STRIDE-per-Interaction.
A metodologia STRIDE é baseada na criação de Diagramas de Fluxo de Dados
(DFDs) para identificar entidades do sistema, eventos e limites do sistema. A
precisão dos DFDs é crucial para o sucesso da aplicação do STRIDE, embora seu uso
exclusivo possa ser limitante, pois não representa decisões arquitetônicas relacionadas
à segurança. 

O acrônimo STRIDE representa seis categorias de ameaças: Spoofing
(falsificação de identidade), Tampering (manipulação de dados), Repudiation (repúdio),
Information Disclosure (divulgação de informações), Denial of Service (negação de serviço)
e Elevation of Privilege (elevação de privilégio). Cada uma dessas categorias
corresponde a uma propriedade de segurança violada, como autenticação, integridade,
não-repúdio, confidencialidade, disponibilidade e autorização, respectivamente.
A metodologia STRIDE é utilizada para identificar ameaças conhecidas com base
nessas categorias, auxiliando na navegação pelo modelo do sistema criado na fase
inicial. 

Apesar de a Microsoft não manter mais o STRIDE, ele ainda é implementado como
parte do Ciclo de Vida de Desenvolvimento Seguro da Microsoft (SDL) com a Ferramenta
de Modelagem de Ameaças, que continua disponível. A metodologia
STRIDE é fácil de adotar, mas pode ser demorada, especialmente à medida que a
complexidade do sistema aumenta. Estudos descritivos da técnica de modelagem de ameaças da
Microsoft mostram que o STRIDE tem uma taxa moderadamente baixa de falsos positivos e
uma taxa moderadamente alta de falsos negativos. 

A aplicação do STRIDE não se limita a sistemas cibernéticos, mas também a
sistemas ciber-físicos, demonstrando sua versatilidade. Além disso, a metodologia
STRIDE pode ser combinada com outras abordagens de modelagem de ameaças para criar
uma visão mais robusta e abrangente das potenciais ameaças. A
escolha da metodologia de modelagem de ameaças deve considerar áreas específicas a
serem abordadas, o tempo disponível para a modelagem, a experiência com modelagem de
ameaças e o nível de envolvimento dos stakeholders. 

Em resumo, a metodologia STRIDE é uma ferramenta poderosa para identificar e
mitigar ameaças em sistemas complexos. Sua aplicação em organizações não hierárquicas
pode ser particularmente valiosa, pois permite uma abordagem sistemática para a
segurança, considerando a horizontalidade como um ativo. A adoção de STRIDE, juntamente
com outras metodologias de modelagem de ameaças, pode proporcionar uma defesa mais
focada e eficaz contra ameaças cibernéticas. 

\subsection{Attack trees}
\label{sec:attack_trees}

As árvores de ataque são uma das técnicas mais antigas e amplamente aplicadas
para modelagem de ameaças em sistemas cibernéticos, ciber-físicos e físicos.
Desenvolvidas por Bruce Schneier em 1999, inicialmente foram aplicadas como um método
independente e, desde então, têm sido combinadas com outros métodos e frameworks.
As árvores de ataque são essencialmente diagramas que representam ataques a um
sistema em forma de árvore. A raiz da árvore é o objetivo do ataque, e as folhas são
as maneiras de alcançar esse objetivo. Cada objetivo é representado como uma
árvore separada, resultando em um conjunto de árvores de ataque para a análise de
ameaças do sistema. 

A construção de uma árvore de ataque geralmente requer algumas iterações de
decomposição do objetivo. Uma vez identificados todos os nós folha, podem ser atribuídos
marcadores de possibilidade, que devem ser definidos após uma pesquisa relevante sobre
cada etapa. Durante o exame de diferentes métodos para alcançar o objetivo, pode-se
perceber que isso pode ser realizado de várias maneiras. Para incorporar essas
diferentes opções na árvore, devem ser usados nós AND e OR. Nós AND indicam que ambos os
nós devem ser realizados para avançar para a próxima etapa, enquanto nós OR
representam alternativas. Em sistemas complexos, árvores de ataque
podem ser construídas para cada componente, em vez de para o sistema como um todo. 

As árvores de ataque são fáceis de entender e adotar, mas são úteis apenas
quando o sistema e as preocupações de segurança são bem compreendidos. O método
assume que os analistas possuem alta expertise em cibersegurança e, portanto, não
fornece diretrizes para avaliar sub-objetivos, ataques ou riscos.
Nos últimos anos, essa técnica tem sido frequentemente usada em combinação com
outras técnicas e dentro de frameworks como STRIDE, CVSS e PASTA.
A aplicação de árvores de ataque pode ajudar a tomar decisões de segurança,
verificar se os sistemas são vulneráveis a um ataque e avaliar um tipo específico de
ataque. 

Um objetivo adicional do método é gerar portas de ataque para componentes
individuais. Essas portas de ataque, que são efetivamente nós raiz para as árvores de
ataque dos componentes, ilustram atividades que podem passar risco para os
componentes conectados. A pontuação auxilia no processo de realização de uma avaliação de
risco do sistema. Se uma porta de ataque depende de um nó raiz de componente com uma
alta pontuação de risco, essa porta de ataque também terá uma alta pontuação de
risco e uma alta probabilidade de ser executada. O oposto também é verdadeiro.
Este método foi utilizado em um estudo de caso para uma rede de comunicações
ferroviárias, demonstrando sua aplicabilidade prática. 

Em resumo, as árvores de ataque são uma ferramenta poderosa para identificar e
mitigar ameaças em sistemas complexos. Sua aplicação em organizações não hierárquicas
pode ser particularmente valiosa, pois permite uma abordagem sistemática para a
segurança, considerando a horizontalidade como um ativo. A adoção de árvores de ataque,
juntamente com outras metodologias de modelagem de ameaças, pode proporcionar uma defesa
mais focada e eficaz contra ameaças cibernéticas. 

\subsection{Security Cards}
\label{sec:Security Cards}

\subsection{Personna non Grata}
\label{sec:Personna}

\subsection{PASTA}
\label{sec:PASTA}



\section{Trabalhos Relacionados}
\label{sec:related}

A modelagem de ameaças é um campo dinâmico e em constante evolução, com várias
abordagens e frameworks desenvolvidos ao longo dos anos para atender a diferentes
necessidades e contextos. Este capítulo revisa alguns dos trabalhos mais relevantes na área
de modelagem de ameaças, com foco em métodos que podem ser adaptados ou servir de
inspiração para o desenvolvimento de um protocolo específico para organizações
não-hierárquicas. 

No artigo "ABC: A Cryptocurrency-Focused Threat Modeling Framework", os autores
Ghada Almashaqbeh, Allison Bishop e Justin Cappos propõem um modelo de ameaças
específico para criptomoedas, destacando a necessidade de frameworks especializados
para lidar com as particularidades desses sistemas. 

O framework ABC introduz a matriz de colusão como uma inovação chave,
permitindo cobrir um amplo espectro de casos de ameaças sem tornar o processo
excessivamente complexo. Este modelo é particularmente eficaz em identificar riscos
financeiros, como demonstrado em estudos de caso reais e em um estudo de usuários, onde
71\% dos participantes que utilizaram o ABC conseguiram identificar ameaças
financeiras, em comparação com apenas 13\% dos que usaram o STRIDE. Esta abordagem é
particularmente importante em criptomoedas permissionless, onde qualquer pessoa pode
participar e onde a colusão entre atacantes é uma preocupação significativa. 

No contexto de organizações não-hierárquicas, a modelagem de ameaças deve considerar
a horizontalidade como um ativo. Nós visamos desenvolver protocolos de modelagem de ameaças que
valorizem a horizontalidade, avaliando esses novos protocolos com membros de grupos
com diferentes níveis de horizontalidade. 

Assim como o framework ABC aborda as especificidades das criptomoedas, o
trabalho busca adaptar a modelagem de ameaças para contextos onde a
ausência de hierarquia é uma característica fundamental. Ambos os trabalhos
destacam a importância de frameworks especializados que considerem as
particularidades dos sistemas e organizações que visam proteger.

\section{Princípios de Organizações Horizontais}
\label{sec:introduction}

Aqui está o texto traduzido e revisado para melhorar a coesão:

---

As organizações horizontais, como cooperativas de trabalhadores, sindicatos, organizações ativistas e projetos de software de código aberto, operam com base em princípios que promovem a igualdade, a participação coletiva e a ausência de hierarquia rígida. Esses princípios são fundamentais para assegurar que todos os membros tenham voz ativa e que as decisões sejam tomadas de maneira democrática e inclusiva.

Um dos princípios centrais dessas organizações é a eleição dos organismos dirigentes, da base ao topo, com o direito de destituição de qualquer eleito pelo coletivo que o escolheu. Isso garante que os líderes sejam responsáveis perante os membros e que possam ser substituídos caso não desempenhem suas funções de maneira satisfatória.

Outro princípio relevante é a obrigatoriedade de os organismos dirigentes prestarem contas regularmente de suas atividades às respectivas organizações. Esse processo inclui considerar atentamente as opiniões e críticas dos membros, valorizando-as como contribuições para a reflexão e as decisões coletivas, de modo a aprimorar o funcionamento organizacional.

A livre expressão de opiniões e o debate cuidadoso sobre elas também são características essenciais. As organizações horizontais buscam assegurar que o maior número possível de membros participe do trabalho, da reflexão, da tomada de decisões e da ação coletiva, incorporando contribuições individuais. Esse modelo promove um ambiente colaborativo e inovador, onde todas as vozes são ouvidas e valorizadas.

O trabalho coletivo e a direção compartilhada são pilares dessas organizações. A tomada de decisões por consenso ou maioria, junto com a iniciativa mais ampla possível de todas as organizações dentro de sua esfera de atuação, é incentivada, sempre em conformidade com os princípios estatutários e as resoluções dos organismos de responsabilidade superior.

Além disso, o respeito pelas decisões coletivas e opiniões divergentes é fundamental. As organizações horizontais estimulam e valorizam o estudo, a reflexão, a intervenção e as contribuições de cada membro, combatendo o individualismo e a imposição de opiniões ou decisões pessoais. Isso cria um ambiente de unidade de pensamento e ação, onde a disciplina consciente e voluntária é promovida.

Esses princípios são indispensáveis para o funcionamento eficaz de organizações horizontais, assegurando que a horizontalidade seja mantida como um ativo valioso. No contexto da modelagem de ameaças, é crucial desenvolver protocolos que respeitem esses princípios, garantindo que as soluções de segurança sejam compatíveis com a estrutura e os valores dessas organizações.

Com base nos estatutos do Partido Comunista Português (PCP), propõe-se a seguinte seção de tese, que explora o funcionamento interno de organizações não hierárquicas, com atenção especial aos mecanismos de segurança voltados para prevenir ameaças, como **"Representante Malicioso"**, **"Abuso de Poder em Emergências"**, **"Quórum Artificial por Colusão"** e **"Negação de Quórum por Colusão"**.

---

### Estrutura e Segurança Interna em Organizações Não Hierárquicas

Organizações não hierárquicas, como o PCP, estruturam-se de forma a promover a participação coletiva e descentralizada, ao mesmo tempo em que mantêm mecanismos de coordenação e coesão que asseguram sua funcionalidade e segurança interna. Esse modelo organizacional enfatiza a democracia interna, a transparência e a responsabilidade compartilhada como fundamentos para a construção de confiança mútua e interação eficaz com tecnologias de suporte. A seguir, detalham-se os principais elementos estruturais e os mecanismos de segurança que buscam prevenir riscos organizacionais e de governança.

#### Organização e Princípios Fundamentais

De acordo com seus estatutos, o PCP adota o princípio do centralismo democrático, que equilibra a autonomia das unidades organizacionais de base com a uniformidade de orientação política definida por organismos centrais. Sua estrutura baseia-se na participação ativa dos membros em células organizacionais, que atuam como elo direto com a comunidade. As decisões são tomadas coletivamente, respeitando a pluralidade de opiniões até que se alcance um consenso ou maioria, assegurando um processo inclusivo e deliberativo.

#### Mecanismos de Prevenção de Riscos

1. **Representante Malicioso:**  
   O processo de seleção e eleição dos representantes é regulado por normas claras, incluindo critérios de fidelidade aos princípios do partido e avaliação de conduta prévia. Adicionalmente, o direito de destituição pelo coletivo que os elegeu funciona como mecanismo preventivo contra abusos de poder.

2. **Abuso de Poder em Emergências:**  
   Situações extraordinárias são regulamentadas de forma a limitar a autonomia nas decisões emergenciais, com supervisão e revisão posterior pelos organismos superiores. Essa abordagem visa prevenir decisões arbitrárias ou desvios em momentos críticos.

3. **Quórum Artificial por Colusão:**  
   Para evitar a formação de quóruns artificiais, o PCP promove a rotatividade e a diversidade nos órgãos de direção. O princípio de distribuição de tarefas entre diferentes níveis organizacionais e a fiscalização contínua garantem que as decisões não sejam controladas por grupos restritos ou facções internas.

4. **Negação de Quórum por Colusão:**  
   A ampla mobilização dos membros nas decisões coletivas reduz o risco de bloqueios deliberados. O estímulo à participação regular e a garantia de acesso a informações relevantes fortalecem a transparência e impedem práticas que dificultem deliberações fundamentais.

#### Relação com Tecnologias

Organizações não hierárquicas como o PCP têm o potencial de interagir eficazmente com tecnologias horizontais, especialmente aquelas que suportam comunicação distribuída, deliberação coletiva e registro imutável de decisões (como blockchain). Essas tecnologias podem reforçar os valores de igualdade, transparência e segurança, mitigando riscos de centralização e abuso de poder.

#### Conclusão

A estrutura não hierárquica do PCP demonstra que é possível equilibrar a participação democrática com mecanismos robustos de segurança, assegurando a continuidade organizacional e a proteção contra ameaças internas. Esse modelo oferece uma base teórica e prática para o desenvolvimento de tecnologias de segurança alinhadas aos princípios de horizontalidade e coesão coletiva.

--- 

Se precisar de ajustes adicionais ou um foco mais específico, é só pedir!

\section{Lacunas na Literatura}
\label{sec:introduction}

Embora a modelagem de ameaças tenha sido amplamente estudada e aplicada em
diversos contextos, ainda existem lacunas significativas na literatura, especialmente
no que diz respeito à segurança em organizações não-hierárquicas. A maioria das
ferramentas e técnicas de segurança cibernética foi desenvolvida com base em pressupostos
hierárquicos, refletindo as necessidades de entidades militares ou corporativas, onde há uma
clara cadeia de comando e responsabilidades bem definidas. No entanto, essas
tecnologias não são adequadas para setores horizontais e participativos, como cooperativas
de trabalhadores e grupos ativistas, que operam com base em processos
democráticos e coletivos. 

Uma das principais lacunas identificadas é a falta de exploração na área de
segurança horizontal, ou seja, técnicas e tecnologias de segurança que utilizam a
participação democrática para a tomada de decisões de segurança. O trabalho COLBAC:
Shifting Cybersecurity from Hierarchical to Horizontal Designs destaca a necessidade
de desenvolver tecnologias que beneficiem a comunidade limitando os privilégios
dos poderosos dentro de uma organização por meio da participação democrática. No
entanto, o estudo também aponta que a implementação de tais sistemas requer mais opções
de configuração e interações não mediadas, o que pode ser um desafio
significativo. 

A pesquisa etnográfica também pode fornecer insights valiosos sobre como essas
organizações trabalham com sistemas centralizados e hierárquicos, e como essas práticas
estabelecidas podem ser usadas para gerar métodos de design de interfaces entre COLBAC e
sistemas mais centralizados. A observação de como a introdução de técnicas de segurança
horizontais afeta a organização das comunidades pode ajudar a refletir essas mudanças em
novas tecnologias. 

Por fim, é crucial garantir que as soluções desenvolvidas sejam utilizáveis. A
criação de um sistema de segurança horizontal, ou qualquer sistema, deve considerar a
usabilidade para garantir que os membros da organização possam efetivamente utilizar e
manter o sistema sem comprometer os princípios de participação e igualdade.

Essas lacunas na literatura destacam a necessidade de mais pesquisa e
desenvolvimento na área de segurança cibernética para organizações não-hierárquicas.
Abordar esses desafios permitirá a criação de sistemas de segurança que não apenas
protejam contra ameaças externas, mas também respeitem e reforcem a estrutura participativa
dessas organizações. 
