%!TEX root = ../template.tex
%%%%%%%%%%%%%%%%%%%%%%%%%%%%%%%%%%%%%%%%%%%%%%%%%%%%%%%%%%%%%%%%%%%%
%% chapter3.tex
%% NOVA thesis document file
%%
%% Chapter with the template manual
%%%%%%%%%%%%%%%%%%%%%%%%%%%%%%%%%%%%%%%%%%%%%%%%%%%%%%%%%%%%%%%%%%%%

\typeout{NT FILE chapter3.tex}%

\chapter{Related Work}
\label{cha:related_work}

\glsresetall

\section{Traditional Threat Modeling Approaches}
\label{sec:traditional_threat_modeling}

\subsection{STRIDE}
\label{subsec:stride}

O STRIDE, desenvolvido pela Microsoft, é uma metodologia sistemática
para a modelagem de ameaças, projetada para identificar potenciais
vulnerabilidades em sistemas de software. O acrônimo STRIDE representa
seis categorias principais de ameaças: Spoofing (falsificação),
Tampering (manipulação), Repudiation (repúdio), Information Disclosure
(divulgação de informações), Denial of Service (negação de serviço) e
Elevation of Privilege (elevação de privilégios)
\cite{ThreatModelingdesigningForSecurity}. Cada uma dessas categorias
reflete uma violação específica das propriedades desejadas de
segurança, como autenticidade, integridade, não-repúdio,
confidencialidade, disponibilidade e autorização
\cite{MicrosoftThreatModelingTechnique}.

A aplicação do STRIDE começa com a criação de Diagramas de Fluxo de
Dados (DFDs) para mapear a movimentação de informações no sistema
\cite{UncoverSecurityDesignFlawsSTRIDE}. Os DFDs ajudam a identificar
elementos como entidades externas, processos, fluxos de dados e
armazenamentos de dados. Para cada elemento do diagrama, as seis
categorias de ameaças do STRIDE são analisadas para identificar
possíveis vulnerabilidades \cite{ThreatModelingdesigningForSecurity}.

Cada ameaça no STRIDE tem uma definição clara e exemplos práticos para
auxiliar na identificação e mitigação. Por exemplo:

\begin{enumerate}
    \item \textbf{Spoofing:} Ameaças que envolvem a falsificação da
identidade de usuários ou processos, comprometendo a autenticidade.
    \item \textbf{Tampering:} Manipulação de dados em trânsito, no
armazenamento ou na memória, afetando a integridade.
    \item \textbf{Repudiation:} Cenários onde usuários negam ações
realizadas, muitas vezes devido à falta de mecanismos adequados de
registro.
    \item \textbf{Information Disclosure:} Exposição de informações
sensíveis para partes não autorizadas, comprometendo a
confidencialidade.
    \item \textbf{Denial of Service:} Ataques que sobrecarregam os
recursos do sistema, prejudicando a disponibilidade.
    \item \textbf{Elevation of Privilege:} Casos onde um ator
mal-intencionado obtém privilégios superiores aos autorizados,
comprometendo a autorização.
\end{enumerate}

A metodologia STRIDE pode ser adaptada para diferentes contextos. Por
exemplo, em sistemas ciberfísicos, é possível avaliar ameaças
relacionadas a componentes de hardware e software, como falhas em
sincronização ou comunicação
\cite{STRIDEthreatmodelingforcyberphysical}. Além disso, variantes
como STRIDE-per-Element e STRIDE-per-Interaction oferecem abordagens
mais focadas para identificar ameaças em elementos específicos ou em
interações entre componentes \cite{ThreatModelingdesigningForSecurity}.

Embora amplamente utilizada, o STRIDE tem limitações. Ele depende
significativamente da experiência dos analistas e pode não capturar
ameaças emergentes em sistemas descentralizados ou ambientes dinâmicos
\cite{SecurityDevelopmentLifecycle}. Por isso, é frequentemente
complementado com outros frameworks, como o DREAD, para priorizar
ameaças com base em impacto e probabilidade \cite{DREADful}.

Em resumo, o STRIDE fornece uma base sólida para a identificação de
ameaças, mas sua aplicação eficaz exige integração com outras
metodologias e adaptações para atender às necessidades específicas de
sistemas modernos e descentralizados
\cite{UncoverSecurityDesignFlawsSTRIDE}.

\subsubsection{Modelos Complementares ao STRIDE}
\label{subsubsec:stride_complementary_models}

Diversos modelos complementares foram derivados ou utilizados em
conjunto com o \gls{stride} para aprimorar sua eficácia e adaptabilidade em
diferentes contextos. Entre esses, destaca-se o modelo DREAD, que
complementa o \gls{stride} ao fornecer uma abordagem quantitativa para
priorização de ameaças \cite{DREADful}. O DREAD utiliza cinco
categorias principais: Damage Potential, Reproducibility,
Exploitability, Affected Users e Discoverability, permitindo que
analistas atribuam valores e criem escores para classificar ameaças de
acordo com sua severidade
\cite{SoftwareandattackcentricThreatModeling}.

O uso combinado do \gls{stride} e do DREAD pode melhorar a avaliação de
riscos em sistemas mais complexos. No entanto, a subjetividade
inerente à atribuição de valores no DREAD pode comprometer a
consistência das análises, especialmente em contextos colaborativos ou
descentralizados \cite{DREADful}. Para mitigar essas limitações,
algumas organizações têm integrado o \gls{stride} a frameworks mais
abrangentes, como o PASTA (Process for Attack Simulation and Threat
Analysis), que adota uma abordagem iterativa para identificar e
priorizar ameaças \cite{SoftwareandattackcentricThreatModeling}.

Adicionalmente, o uso de árvores de ataque tem se mostrado eficaz para
complementar o \gls{stride}, permitindo que equipes representem visualmente
cenários de ameaça complexos e identifiquem múltiplas vias de ataque.
Essa integração é particularmente útil em organizações horizontais,
onde a ausência de centralização aumenta a necessidade de mapeamentos
colaborativos de ameaças \cite{ThreatModelingdesigningForSecurity}.

Embora os modelos complementares e frameworks adicionais enriqueçam o
\gls{stride}, sua aplicação ainda requer adaptações específicas para atender
às particularidades de ambientes descentralizados, como organizações
horizontais. Nessas estruturas, frameworks colaborativos, como o
Security Cards e o CoReTM, têm demonstrado maior alinhamento com os
princípios de governança distribuída
\cite{SoftwareandattackcentricThreatModeling}.

\subsection{Attack Trees}
\label{subsec:attack_trees}

As árvores de ataque, introduzidas por Bruce Schneier
\cite{AttackTrees}, oferecem uma estrutura hierárquica para modelagem
de ameaças, onde o objetivo do ataque é representado pelo nó raiz, e
os subobjetivos e etapas necessárias para alcançá-lo são dispostos em
nós filhos. Cada nó pode ser detalhado com operadores lógicos como AND
e OR, representando condições que devem ser cumpridas conjuntamente ou
alternativamente \cite{FoundationsofAttackTrees}.

Uma das principais vantagens das árvores de ataque é sua capacidade de
decompor ameaças complexas em componentes menores e mais gerenciáveis,
permitindo uma análise sistemática \cite{Energytheftdetectionissues}.
Essa metodologia facilita a identificação de múltiplos caminhos de
ataque, permitindo que as organizações priorizem contramedidas com
base em métricas como custo, impacto e probabilidade
\cite{AnAttackTreeBasedRisk}.

Aplicações práticas das árvores de ataque incluem sua utilização em
redes de sensores sem fio para avaliar riscos à privacidade de
localização \cite{AnAttackTreeBasedRisk}, bem como na detecção de
roubo de energia em infraestruturas avançadas de medição, como redes
elétricas inteligentes \cite{Energytheftdetectionissues}. Em ambos os
casos, a abordagem possibilita que as organizações mapeiem cenários de
ameaça específicos e projetem contramedidas eficazes.

Além disso, estudos como \cite{FoundationsofAttackTrees} destacam a
reutilização de subárvores para aumentar a eficiência em sistemas
complexos. Essa prática permite que elementos compartilhados entre
diferentes cenários sejam modelados uma única vez e incorporados em
análises futuras, economizando tempo e recursos.

Embora sejam amplamente aplicáveis, as árvores de ataque apresentam
desafios relacionados ao esforço necessário para sua construção
inicial e à complexidade em sistemas de grande escala
\cite{AttackTrees}. A colaboração entre stakeholders, incluindo
especialistas técnicos e operacionais, é essencial para garantir que a
representação das ameaças seja precisa e abrangente
\cite{Energytheftdetectionissues}.

As árvores de ataque também se destacam como ferramentas
complementares a metodologias como STRIDE e podem ser utilizadas tanto
para identificar ameaças quanto para organizar as já descobertas. Além
disso, a reutilização de árvores existentes, como aquelas voltadas
para fraudes ou eleições, economiza tempo e fornece uma base sólida
para análise. Apesar de sua versatilidade, o uso eficaz das árvores
depende de representações claras de nós AND/OR e da avaliação contínua
para evitar excessos ou lacunas \cite{ThreatModelingdesigningForSecurity}.

\section{Emerging Methodologies}
\label{sec:emerging_methodologies}

\subsection{PASTA}
\label{subsec:pasta}

O PASTA (Process for Attack Simulation and Threat Analysis) é uma
metodologia de modelagem de ameaças centrada no risco, projetada para
integrar segurança ao longo do ciclo de vida do desenvolvimento de
software. Proposto por Tony UcedaVelez e Marco M. Morana
\cite{RiskCentricThreatModeling}, o framework consiste em sete
estágios sequenciais que permitem uma análise aprofundada e iterativa
de ameaças e vulnerabilidades.

O objetivo principal do PASTA é alinhar as preocupações de segurança
com os objetivos de negócio, garantindo que as medidas de mitigação
abordem tanto os riscos técnicos quanto os impactos organizacionais.
A metodologia promove uma abordagem orientada ao risco, integrando
simulações de ataques para avaliar a eficácia das contramedidas
propostas \cite{RiskCentricThreatModeling}.

\begin{enumerate}
    \item \textbf{Definition of the Objectives (DO):} Neste estágio
inicial, são definidos os requisitos de segurança, o perfil de risco e
os impactos potenciais nos negócios.
    \item \textbf{Definition of the Technical Scope (DTS):} Este
estágio detalha os aspectos técnicos, como usuários, componentes de
software, infraestrutura de terceiros e dependências externas.
    \item \textbf{Application Decomposition and Analysis (ADA):} A
aplicação é dividida em elementos funcionais básicos para identificar
fluxos de dados, tipos de usuários, e controles de segurança
existentes.
    \item \textbf{Threat Analysis (TA):} Identificação de possíveis
ameaças com base nos elementos e ativos analisados, considerando os
vetores de ataque mais prováveis.
    \item \textbf{Weakness and Vulnerability Analysis (WVA):} Neste
estágio, as ameaças são associadas a vulnerabilidades específicas,
avaliando a eficácia dos controles existentes e identificando
fraquezas.
    \item \textbf{Attack Modeling and Simulation (AMS):} Realização de
simulações para determinar os caminhos de ataque mais prováveis,
utilizando árvores de ataque e outros modelos para explorar cenários
de risco.
    \item \textbf{Risk Analysis and Management (RAM):} Identificação
dos impactos técnicos e de negócio, propondo medidas para mitigar os
riscos prioritários \cite{RiskCentricThreatModeling}.
\end{enumerate}

O PASTA se destaca por sua flexibilidade e profundidade analítica,
tornando-o particularmente eficaz em ambientes dinâmicos e
distribuídos. A metodologia incentiva a colaboração entre stakeholders
de diferentes áreas, promovendo uma compreensão unificada dos riscos e
das prioridades organizacionais. Além disso, a integração de
simulações de ataques permite que as organizações testem a eficácia de
suas estratégias de segurança em condições realistas, aprimorando sua
resiliência contra ameaças emergentes
\cite{RiskCentricThreatModeling}.

Um dos aspectos mais relevantes do PASTA é sua compatibilidade com
organizações horizontais. A abordagem colaborativa da metodologia, que
envolve múltiplos stakeholders em todas as etapas do processo, está
alinhada com os princípios de governança distribuída. Em estruturas
não hierárquicas, onde a responsabilidade pela segurança é
compartilhada, o PASTA oferece um framework estruturado para
identificar e mitigar riscos de forma participativa e eficiente. Além
disso, a análise iterativa do PASTA permite que organizações
horizontais adaptem suas estratégias de segurança às mudanças
constantes em seus ambientes operacionais, fortalecendo a resiliência
coletiva.

\subsection{Security Cards}
\label{subsec:security_cards}

Os Security Cards são uma ferramenta desenvolvida para facilitar o
brainstorming de ameaças de segurança, utilizando um baralho de cartas
que aborda diferentes aspectos de possíveis ataques
\cite{SecurityCardsToolkit}. Criados por Tamara Denning, Batya
Friedman e Tadayoshi Kohno, os cartões cobrem quatro dimensões
principais: motivações do adversário, recursos do adversário, métodos
do adversário e impacto humano \cite{KeepingAheadofOurAdversaries}.
Esta abordagem busca promover a criatividade e a colaboração entre
stakeholders, incentivando uma análise mais holística e abrangente das
ameaças \cite{CyberThreatModeling}.

Cada carta oferece exemplos e cenários relacionados à sua dimensão,
ajudando as equipes a explorar vulnerabilidades que poderiam não ser
identificadas por métodos tradicionais \cite{SecurityCardsToolkit}.
Por exemplo, na dimensão de "Impacto Humano", os cartões podem
destacar como violações de segurança podem afetar a privacidade, o
bem-estar emocional ou financeiro, fornecendo uma perspectiva mais
centrada no usuário \cite{KeepingAheadofOurAdversaries}.

Os Security Cards têm sido utilizados em diversas aplicações, como a
proteção de sistemas biométricos contra ataques de apresentação
\cite{AttackTreesforProtectingBiometric}. Sua estrutura flexível
permite adaptação a diferentes contextos organizacionais, incluindo
ambientes descentralizados. Em organizações horizontais, os Security
Cards facilitam a participação de diversos stakeholders, promovendo a
governança distribuída e reforçando a colaboração
\cite{CyberThreatModeling}.

Apesar de seu potencial, a metodologia pode gerar um número elevado de
falsos positivos, o que exige esforço adicional para filtrar ameaças
relevantes \cite{KeepingAheadofOurAdversaries}. No entanto, sua ênfase
na criatividade e inclusão de múltiplas perspectivas faz dos Security
Cards uma ferramenta valiosa para explorar ameaças emergentes e
fortalecer a resiliência organizacional.

\subsection{Personae Non Grata}
\label{subsec:personae_non_grata}

Personae Non Gratae (PnGs) representam uma abordagem inovadora para a
modelagem de ameaças, destacando-se por seu foco em usuários
mal-intencionados e comportamentos indesejáveis
\cite{PersonaeNonGratae}. Inspiradas pelas personas tradicionais do
design de experiência do usuário, as PnGs ajudam a antecipar como
adversários podem explorar vulnerabilidades em um sistema, fornecendo
uma perspectiva adversarial detalhada
\cite{PnGRequirementsPhaseThreatModeling}.

As PnGs são criadas por meio de técnicas como crowd-sourcing,
permitindo que diferentes stakeholders contribuam com insights para a
identificação de ameaças \cite{PnGRequirementsPhaseThreatModeling}.
Esta abordagem colaborativa aumenta a abrangência e a diversidade dos
perfis de atacantes considerados, permitindo uma modelagem mais
robusta e adaptada a diferentes contextos \cite{PersonaeNonGratae}.

Uma das principais vantagens das PnGs é sua capacidade de capturar
motivações, capacidades e comportamentos específicos de atacantes. Por
exemplo, uma PnG pode descrever um adversário que utiliza phishing
para obter credenciais ou explora falhas de segurança em transações
financeiras \cite{PersonaeNonGratae}. Este nível de detalhamento
auxilia na priorização de contramedidas e na alocação de recursos de
segurança \cite{PnGRequirementsPhaseThreatModeling}.

Além disso, as PnGs são particularmente eficazes em contextos onde as
ameaças internas e externas se sobrepõem. Em organizações horizontais,
onde a governança é distribuída e a responsabilidade é compartilhada,
as PnGs ajudam a mapear potenciais riscos que podem surgir de atores
internos, como colaboradores, ou externos, como competidores
\cite{PersonaeNonGratae}.

Apesar de seus benefícios, a implementação de PnGs requer esforço
significativo para garantir que os perfis sejam precisos e relevantes.
No entanto, quando integradas a outras metodologias, como árvores de
ataque ou \gls{stride}, as PnGs oferecem uma camada adicional de análise,
tornando-as uma ferramenta indispensável para organizações que buscam
compreender e mitigar ameaças de forma abrangente
\cite{PnGRequirementsPhaseThreatModeling}.

\section{Hybrid and Collaborative Approaches}
\label{sec:hybrid_collaborative_approaches}

As abordagens híbridas e colaborativas buscam integrar diferentes
metodologias para criar frameworks adaptáveis e eficazes em contextos
variados. Dentre essas, destaca-se o Hybrid Threat Modeling Method
(hTMM), que combina elementos de diferentes frameworks para uma
análise abrangente de riscos. Ele é particularmente útil em cenários
que envolvem múltiplos stakeholders e requerem alinhamento entre
objetivos de segurança e prioridades de negócio
\cite{AHybridThreatModelingMethod}.

Outro exemplo é o Collaborative and Remote Threat Modeling (CoReTM),
projetado para facilitar a modelagem de ameaças em equipes
distribuídas ou remotas. Utilizando ferramentas colaborativas, como
plataformas de anotação compartilhada, o CoReTM torna o processo mais
acessível e inclusivo, sendo ideal para organizações horizontais e
globais \cite{CoReTM}.

Por fim, o Participatory Threat Modeling (PTM) promove a inclusão de
uma ampla gama de stakeholders no processo de modelagem de ameaças.
Esta abordagem valoriza a diversidade de perspectivas, sendo
especialmente relevante em contextos descentralizados onde a
transparência e a participação coletiva são essenciais
\cite{ParticipatoryThreatModelling}. Essas metodologias reforçam a
governança distribuída e fortalecem a resiliência organizacional,
complementando o trabalho desenvolvido por frameworks mais
tradicionais.



\section{Decentralized Trust and Cryptographic Frameworks}
\label{sec:decentralized_cryptographic}
asdf

\subsection{COLBAC}
\label{subsec:colbac}
asdf

\subsection{ABCcrypto}
\label{subsec:abccrypto}
asdf

\subsection{PGP and the Web of Trust}
\label{subsec:pgp_web_of_trust}
asdf





\section{Comparative Perspectives}
\label{sec:comparative_perspectives}

\subsection{Criteria for Evaluation}
\label{subsec:criteria_evaluation}
Como avaliar as metodologias  

\subsection{Applicability in Non-Hierarchical Organizations}
\label{subsec:applicability_non_hierarchical}

asdf