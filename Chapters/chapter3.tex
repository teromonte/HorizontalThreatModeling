%!TEX root = ../template.tex
%%%%%%%%%%%%%%%%%%%%%%%%%%%%%%%%%%%%%%%%%%%%%%%%%%%%%%%%%%%%%%%%%%%%
%% chapter3.tex
%% NOVA thesis document file
%%
%% Chapter with a short latex tutorial and examples
%%%%%%%%%%%%%%%%%%%%%%%%%%%%%%%%%%%%%%%%%%%%%%%%%%%%%%%%%%%%%%%%%%%%

\typeout{NT FILE chapter3.tex}%

\chapter{Design}
\label{cha:design}

\glsresetall

\section{Desafios na Modelagem de Ameaças para Organizações Não-Hierárquicas}
\label{sec:introduction}

Organizações não-hierárquicas, como cooperativas de trabalhadores, sindicatos,
grupos ativistas e projetos de software de código aberto, enfrentam desafios únicos
em termos de segurança cibernética devido à sua estrutura participativa e
democrática. A maioria das ferramentas e técnicas de segurança cibernética foi desenvolvida
com base em pressupostos hierárquicos, refletindo as necessidades de entidades
militares ou corporativas, onde há uma clara cadeia de comando e responsabilidades bem
definidas. No entanto, essas tecnologias não são adequadas para setores horizontais e
participativos, como cooperativas de trabalhadores e grupos ativistas, que operam com base em
processos democráticos e coletivos. 

Um dos principais desafios enfrentados por organizações horizontais é a gestão
de segredos, como senhas e chaves de criptografia. Em uma organização
hierárquica, esses segredos são frequentemente controlados por um pequeno grupo de
administradores que têm autoridade para gerenciar o acesso. No entanto, em uma organização
horizontal, decidir quem deve ter acesso a esses segredos pode ser mais complicado. Se
todos os membros tiverem acesso, há um risco maior de abuso ou erro humano. Por
outro lado, restringir o acesso a um pequeno grupo pode criar uma hierarquia de
fato, minando os princípios de horizontalidade. 

Outro desafio significativo é a implementação de políticas de controle de
acesso de forma horizontal. Sistemas de controle de acesso tradicionais, como MAC
Mandatory Access Control, DAC Discretionary Access Control e RBAC Role-Based Access
Control, tendem a forçar a criação de uma hierarquia, onde certas entidades na
organização têm o poder de implementar as regras de controle de acesso que foram
democraticamente criadas pela organização. Se os indivíduos que têm a capacidade de aplicar as
regras decidirem não fazê-lo, as políticas de controle de acesso recém-criadas
tornam-se ineficazes, formando uma hierarquia com as entidades capazes de aplicar o
controle de acesso no topo. 

Além disso, a tomada de decisões coletivas em sistemas de segurança pode ser
vulnerável a ataques específicos, como o ataque Sybil, onde um usuário mal-intencionado
pode se passar por vários votantes legítimos, comprometendo a integridade do
processo democrático. A interrupção do sistema por um grupo de usuários
mal-intencionados também pode impedir o funcionamento adequado do sistema, interrompendo
operações de votação e outras atividades críticas. 

Esses desafios destacam a necessidade de desenvolver tecnologias e protocolos
de segurança que permitam uma organização ser flexível e dinâmica em sua
horizontalidade, sendo participativa ou hierárquica conforme necessário, sem comprometer a
capacidade de retornar à horizontalidade quando desejado. A criação de sistemas que
utilizem processos democráticos para a tomada de decisões de segurança pode ajudar a
resolver esses desafios, permitindo que as organizações horizontais mantenham seus
princípios fundamentais de participação e igualdade enquanto protegem seus ativos e dados
sensíveis. 

\section{Horizontality as an Asset}
\label{sec:horizontality_asset}

We also show some stuff which is not that common!
