%!TEX root = ../template.tex
%%%%%%%%%%%%%%%%%%%%%%%%%%%%%%%%%%%%%%%%%%%%%%%%%%%%%%%%%%%%%%%%%%%%
%% chapter3.tex
%% NOVA thesis document file
%%%%%%%%%%%%%%%%%%%%%%%%%%%%%%%%%%%%%%%%%%%%%%%%%%%%%%%%%%%%%%%%%%%%

\typeout{NT FILE chapter3.tex}%

\chapter{Trabalhos Relacionados}
\label{cha:related_work}

\glsresetall

\section{Abordagens Tradicionais de Modelagem de Ameaças}
\label{sec:traditional_threat_modeling}

\subsection{STRIDE}
\label{subsec:stride}

O \gls{stride}, desenvolvido pela Microsoft, é uma metodologia sistemática
para a modelagem de ameaças, projetada para identificar potenciais
vulnerabilidades em sistemas de software \cite{MicrosoftThreatModelingTechnique}.
O acrônimo \gls{stride} representa seis categorias principais
de ameaças \cite{ThreatModelingdesigningForSecurity}.
Cada uma dessas categorias reflete uma violação específica das
propriedades desejadas de segurança, como autenticidade,
integridade, não-repúdio, confidencialidade, disponibilidade e autorização
\cite{MicrosoftThreatModelingTechnique}.

A aplicação do \gls{stride} começa com a criação de \gls{dfds}
para mapear a movimentação de informações no sistema
\cite{UncoverSecurityDesignFlawsSTRIDE}. Os \gls{dfds} ajudam a identificar
elementos como entidades externas, processos, fluxos de dados e
armazenamentos de dados. Para cada elemento do diagrama, as seis
categorias de ameaças do \gls{stride} são analisadas para identificar
possíveis vulnerabilidades \cite{ThreatModelingdesigningForSecurity}.

Cada ameaça no \gls{stride} tem uma definição clara e exemplos práticos para
auxiliar na identificação e mitigação. Por exemplo:

\begin{enumerate}
    \item \textbf{Spoofing:} Ameaças que envolvem a falsificação da
identidade de usuários ou processos, comprometendo a autenticidade.
    \item \textbf{Tampering:} Manipulação de dados em trânsito, no
armazenamento ou na memória, afetando a integridade.
    \item \textbf{Repudiation:} Cenários onde usuários negam ações
realizadas, muitas vezes devido à falta de mecanismos adequados de
registro.
    \item \textbf{Information Disclosure:} Exposição de informações
sensíveis para partes não autorizadas, comprometendo a
confidencialidade.
    \item \textbf{Denial of Service:} Ataques que sobrecarregam os
recursos do sistema, prejudicando a disponibilidade.
    \item \textbf{Elevation of Privilege:} Casos onde um ator
mal-intencionado obtém privilégios superiores aos autorizados,
comprometendo a autorização.
\end{enumerate}

A metodologia \gls{stride} pode ser adaptada para diferentes contextos. Por
exemplo, em sistemas ciberfísicos, é possível avaliar ameaças
relacionadas a componentes de hardware e software, como falhas em
sincronização ou comunicação
\cite{STRIDEthreatmodelingforcyberphysical}. Além disso, variantes
como STRIDE-per-Element e STRIDE-per-Interaction oferecem abordagens
mais focadas para identificar ameaças em elementos específicos ou em
interações entre componentes \cite{ThreatModelingdesigningForSecurity}.

Embora amplamente utilizada, o \gls{stride} tem limitações. Ele depende
significativamente da experiência dos analistas e pode não capturar
ameaças emergentes em sistemas descentralizados ou ambientes dinâmicos
\cite{SecurityDevelopmentLifecycle}. Por isso, é frequentemente
complementado com outros frameworks, como o \gls{dread}, para priorizar
ameaças com base em impacto e probabilidade \cite{DREADful}.

Em resumo, o \gls{stride} fornece uma base sólida para a identificação de
ameaças, mas sua aplicação eficaz exige integração com outras
metodologias e adaptações para atender às necessidades específicas de
sistemas modernos e descentralizados
\cite{UncoverSecurityDesignFlawsSTRIDE}.

\subsubsection{Modelos Complementares ao STRIDE}
\label{subsubsec:stride_complementary_models}

Diversos modelos complementares foram derivados ou utilizados em
conjunto com o \gls{stride} para aprimorar sua eficácia e adaptabilidade em
diferentes contextos \cite{SoftwareandattackcentricThreatModeling}.
Entre esses, destaca-se o modelo \gls{dread}, que
complementa o \gls{stride} ao fornecer uma abordagem quantitativa para
priorização de ameaças \cite{DREADful}. O \gls{dread} utiliza cinco
categorias principais: Damage Potential, Reproducibility,
Exploitability, Affected Users e Discoverability, permitindo que
analistas atribuam valores e criem escores
para classificar ameaças de acordo com sua severidade
\cite{SoftwareandattackcentricThreatModeling, DREADful}.

O uso combinado do \gls{stride} e do \gls{dread} pode melhorar a avaliação de
riscos em sistemas mais complexos. No entanto, a subjetividade
inerente à atribuição de valores no \gls{dread} pode comprometer a
consistência das análises \cite{DREADful}. Para mitigar essas limitações,
algumas organizações têm integrado o \gls{stride} a frameworks mais
abrangentes, como o \gls{pasta}, que adota uma abordagem iterativa para identificar e
priorizar ameaças \cite{SoftwareandattackcentricThreatModeling}.

Adicionalmente, o uso de árvores de ataque tem se mostrado eficaz para
complementar o \gls{stride}, permitindo que equipes representem visualmente
cenários de ameaça complexos e identifiquem múltiplas vias de ataque
\cite{FoundationsofAttackTrees}.
Essa integração é particularmente útil em organizações horizontais,
onde a ausência de centralização aumenta a necessidade de mapeamentos
colaborativos de ameaças \cite{ThreatModelingdesigningForSecurity}. 

\subsection{Attack Trees}
\label{subsec:attack_trees}

As árvores de ataque, introduzidas por Bruce Schneier
\cite{AttackTrees}, oferecem uma estrutura hierárquica para modelagem
de ameaças, onde o objetivo do ataque é representado pelo nó raiz, e
os subobjetivos e etapas necessárias para alcançá-lo são dispostos em
nós filhos. Cada nó pode ser detalhado com operadores lógicos como AND
e OR, representando condições que devem ser cumpridas conjuntamente ou
alternativamente \cite{FoundationsofAttackTrees}.

Uma das principais vantagens das árvores de ataque é sua capacidade de
decompor ameaças complexas em componentes menores e mais gerenciáveis,
permitindo uma análise sistemática \cite{Energytheftdetectionissues}.
Essa metodologia facilita a identificação de múltiplos caminhos de
ataque, permitindo que as organizações priorizem contramedidas com
base em métricas como custo, impacto e probabilidade
\cite{AnAttackTreeBasedRisk}.

Aplicações práticas das árvores de ataque incluem sua utilização em
redes de sensores sem fio para avaliar riscos à privacidade de
localização \cite{AnAttackTreeBasedRisk}, bem como na detecção de
roubo de energia em infraestruturas avançadas de medição, como redes
elétricas inteligentes \cite{Energytheftdetectionissues}. Em ambos os
casos, a abordagem possibilita que as organizações mapeiem cenários de
ameaça específicos e projetem contramedidas eficazes.

Além disso, estudos como \cite{FoundationsofAttackTrees} destacam a
reutilização de subárvores para aumentar a eficiência em sistemas
complexos. Essa prática permite que elementos compartilhados entre
diferentes cenários sejam modelados uma única vez e incorporados em
análises futuras, economizando tempo e recursos.

Embora sejam amplamente aplicáveis, as árvores de ataque apresentam
desafios relacionados ao esforço necessário para sua construção
inicial e à complexidade em sistemas de grande escala
\cite{AttackTrees, Energytheftdetectionissues}. A colaboração entre stakeholders, incluindo
especialistas técnicos e operacionais, é essencial para garantir que a
representação das ameaças seja precisa e abrangente
\cite{Energytheftdetectionissues}.

As árvores de ataque também se destacam como ferramentas
complementares a metodologias como \gls{stride} e podem ser utilizadas tanto
para identificar ameaças quanto para organizar as já descobertas
\cite{FoundationsofAttackTrees, ThreatModelingASystematicLiteratureReview}. Além
disso, a reutilização de árvores existentes, como aquelas voltadas
para fraudes ou eleições, economiza tempo e fornece uma base sólida
para análise \cite{FoundationsofAttackTrees}. Apesar de sua versatilidade,
o uso eficaz das árvores depende de representações claras de nós AND/OR
e da avaliação contínua para evitar excessos ou lacunas
\cite{ThreatModelingdesigningForSecurity}.

\section{Metodologias Emergentes}
\label{sec:emerging_methodologies}

\subsection{PASTA}
\label{subsec:pasta}

O \gls{pasta} é uma
metodologia de modelagem de ameaças centrada no risco, projetada para
integrar segurança ao longo do ciclo de vida do desenvolvimento de
software. Proposto por Tony UcedaVelez e Marco M. Morana
\cite{RiskCentricThreatModeling}, o framework consiste em sete
estágios sequenciais que permitem uma análise aprofundada e iterativa
de ameaças e vulnerabilidades.

O objetivo principal do \gls{pasta} é alinhar as preocupações de segurança
com os objetivos de negócio, garantindo que as medidas de mitigação
abordem tanto os riscos técnicos quanto os impactos organizacionais.
A metodologia promove uma abordagem orientada ao risco, integrando
simulações de ataques para avaliar a eficácia das contramedidas
propostas \cite{RiskCentricThreatModeling}.

\begin{enumerate}
    \item \textbf{Definition of the Objectives (DO):} Neste estágio
inicial, são definidos os requisitos de segurança, o perfil de risco e
os impactos potenciais nos negócios.
    \item \textbf{Definition of the Technical Scope (DTS):} Este
estágio detalha os aspectos técnicos, como usuários, componentes de
software, infraestrutura de terceiros e dependências externas.
    \item \textbf{Application Decomposition and Analysis (ADA):} A
aplicação é dividida em elementos funcionais básicos para identificar
fluxos de dados, tipos de usuários, e controles de segurança
existentes.
    \item \textbf{Threat Analysis (TA):} Identificação de possíveis
ameaças com base nos elementos e ativos analisados, considerando os
vetores de ataque mais prováveis.
    \item \textbf{Weakness and Vulnerability Analysis (WVA):} Neste
estágio, as ameaças são associadas a vulnerabilidades específicas,
avaliando a eficácia dos controles existentes e identificando
fraquezas.
    \item \textbf{Attack Modeling and Simulation (AMS):} Realização de
simulações para determinar os caminhos de ataque mais prováveis,
utilizando árvores de ataque e outros modelos para explorar cenários
de risco.
    \item \textbf{Risk Analysis and Management (RAM):} Identificação
dos impactos técnicos e de negócio, propondo medidas para mitigar os
riscos prioritários \cite{RiskCentricThreatModeling}.
\end{enumerate}

O \gls{pasta} se destaca por sua flexibilidade e profundidade analítica,
tornando-o particularmente eficaz em ambientes dinâmicos e
distribuídos \cite{ThreatModelingASystematicLiteratureReview}.
A metodologia incentiva a colaboração entre stakeholders
de diferentes áreas, promovendo uma compreensão unificada dos riscos e
das prioridades organizacionais \cite{ParticipatoryThreatModelling}.
Além disso, a integração de simulações de ataques permite
que as organizações testem a eficácia de suas estratégias de segurança
em condições realistas, aprimorando sua resiliência contra ameaças emergentes
\cite{RiskCentricThreatModeling}.

Um dos aspectos mais relevantes do \gls{pasta} é sua compatibilidade com
organizações horizontais. A abordagem colaborativa da metodologia, que
envolve múltiplos stakeholders em todas as etapas do processo, está
alinhada com os princípios de governança distribuída \cite{Colbac}.
Em estruturas não hierárquicas, onde a responsabilidade pela segurança é
compartilhada, o \gls{pasta} oferece um framework estruturado para
identificar e mitigar riscos de forma participativa e eficiente.

\subsection{Security Cards}
\label{subsec:security_cards}

Os Security Cards são uma ferramenta desenvolvida para facilitar o
brainstorming de ameaças de segurança, utilizando um baralho de cartas
que aborda diferentes aspectos de possíveis ataques
\cite{SecurityCardsToolkit}. Criados por Tamara Denning, Batya
Friedman e Tadayoshi Kohno, os cartões cobrem quatro dimensões
principais: motivações do adversário, recursos do adversário, métodos
do adversário e impacto humano \cite{KeepingAheadofOurAdversaries}.
Esta abordagem busca promover a criatividade e a colaboração entre
stakeholders, incentivando uma análise mais holística e abrangente das
ameaças \cite{CyberThreatModeling}.

Cada carta oferece exemplos e cenários relacionados à sua dimensão,
ajudando as equipes a explorar vulnerabilidades que poderiam não ser
identificadas por métodos tradicionais \cite{SecurityCardsToolkit}.
Por exemplo, na dimensão de "Impacto Humano", os cartões podem
destacar como violações de segurança podem afetar a privacidade, o
bem-estar emocional ou financeiro, fornecendo uma perspectiva mais
centrada no usuário \cite{KeepingAheadofOurAdversaries}.

Os Security Cards têm sido utilizados em diversas aplicações, como a
proteção de sistemas biométricos contra ataques de apresentação
\cite{AttackTreesforProtectingBiometric, KeepingAheadofOurAdversaries}.
Sua estrutura flexível permite adaptação a diferentes contextos organizacionais, incluindo
ambientes descentralizados \cite{ParticipatoryThreatModelling}.
Em organizações horizontais, os Security Cards facilitam
a participação de diversos stakeholders, promovendo a
governança distribuída e reforçando a colaboração
\cite{CyberThreatModeling}.

Apesar de seu potencial, a metodologia pode gerar um número elevado de
falsos positivos, o que exige esforço adicional para filtrar ameaças
relevantes \cite{KeepingAheadofOurAdversaries}. No entanto, sua ênfase
na criatividade e inclusão de múltiplas perspectivas faz dos Security
Cards uma ferramenta valiosa para explorar ameaças emergentes e
fortalecer a resiliência organizacional \cite{CyberThreatModeling}.

\subsection{Personae Non Grata}
\label{subsec:personae_non_grata}

\gls{pngs} representam uma abordagem inovadora para a
modelagem de ameaças, destacando-se por seu foco em usuários
mal-intencionados e comportamentos indesejáveis
\cite{PersonaeNonGratae}. Inspiradas pelas personas tradicionais do
design de experiência do usuário, as \gls{pngs} ajudam a antecipar como
adversários podem explorar vulnerabilidades em um sistema, fornecendo
uma perspectiva adversarial detalhada
\cite{PnGRequirementsPhaseThreatModeling}.

As \gls{pngs} são criadas por meio de técnicas como crowd-sourcing,
permitindo que diferentes stakeholders contribuam com insights para a
identificação de ameaças \cite{PnGRequirementsPhaseThreatModeling}.
Esta abordagem colaborativa aumenta a abrangência e a diversidade dos
perfis de atacantes considerados, permitindo uma modelagem mais
robusta e adaptada a diferentes contextos \cite{PersonaeNonGratae}.

Uma das principais vantagens das \gls{pngs} é sua capacidade de capturar
motivações, capacidades e comportamentos específicos de atacantes
\cite{PnGRequirementsPhaseThreatModeling}.
Por exemplo, uma PnG pode descrever um adversário que utiliza phishing
para obter credenciais ou explora falhas de segurança em transações
financeiras \cite{PersonaeNonGratae}. Este nível de detalhamento
auxilia na priorização de contramedidas e na alocação de recursos de
segurança \cite{PnGRequirementsPhaseThreatModeling}.

Além disso, as \gls{pngs} são particularmente eficazes em contextos onde as
ameaças internas e externas se sobrepõem \cite{PersonaeNonGratae}.
Em organizações horizontais, onde a governança é distribuída
e a responsabilidade é compartilhada, as \gls{pngs} ajudam
a mapear potenciais riscos que podem surgir de atores
internos, como colaboradores, ou externos, como competidores
\cite{PersonaeNonGratae}.

Apesar de seus benefícios, a implementação de \gls{pngs} requer esforço
significativo para garantir que os perfis sejam precisos e relevantes
\cite{PnGRequirementsPhaseThreatModeling}.
No entanto, quando integradas a outras metodologias, como árvores de
ataque ou \gls{stride}, as \gls{pngs} oferecem uma camada adicional de análise,
tornando-as uma ferramenta indispensável para organizações que buscam
compreender e mitigar ameaças de forma abrangente
\cite{PnGRequirementsPhaseThreatModeling}.

\section{Abordagens Híbridas e Colaborativas}
\label{sec:hybrid_collaborative_approaches}

As abordagens híbridas e colaborativas buscam integrar diferentes
metodologias para criar frameworks adaptáveis e eficazes em contextos
variados \cite{AHybridThreatModelingMethod, CoReTM}.
Dentre essas, destaca-se o \gls{htmm}, que combina elementos
de diferentes frameworks para uma análise abrangente de riscos.
Ele é particularmente útil em cenários que envolvem múltiplos
stakeholders e requerem alinhamento entre objetivos de segurança
e prioridades de negócio \cite{AHybridThreatModelingMethod}.

Outro exemplo é o \gls{coretm},
projetado para facilitar a modelagem de ameaças em equipes
distribuídas ou remotas. Utilizando ferramentas colaborativas, como
plataformas de anotação compartilhada, o \gls{coretm} torna o processo mais
acessível e inclusivo, sendo ideal para organizações horizontais e
globais \cite{CoReTM}.

Por fim, o \gls{ptm} promove a inclusão de uma ampla gama de stakeholders
no processo de modelagem de ameaças \cite{ParticipatoryThreatModelling}. 
Esta abordagem valoriza a diversidade de perspectivas, sendo
especialmente relevante em contextos descentralizados onde a
transparência e a participação coletiva são essenciais
\cite{ParticipatoryThreatModelling}. Essas metodologias reforçam a
governança distribuída e fortalecem a resiliência organizacional,
complementando o trabalho desenvolvido por frameworks mais
tradicionais \cite{Colbac}.



\section{Confiança Descentralizada e Frameworks Criptográficos}
\label{sec:decentralized_cryptographic}

\subsection{COLBAC}
\label{subsec:colbac}

O \gls{colbac} é um modelo de controle de
acesso projetado para abordar as especificidades de organizações
horizontais, promovendo uma abordagem democrática e participativa para
a autorização de acessos. Sua proposta busca superar os desafios
impostos por modelos tradicionais de controle de acesso, como
\gls{dac}, \gls{mac} e \gls{rbac}, que frequentemente reforçam dinâmicas
hierárquicas inadequadas para estruturas horizontalizadas \cite{Colbac}.

Uma das características mais marcantes do \gls{colbac} é sua capacidade de
alinhar o controle de acesso às práticas de governança horizontal,
permitindo que decisões sejam tomadas coletivamente por meio de
processos democráticos \cite{ParticipatoryThreatModelling}.
O modelo organiza recursos e processos em três esferas
principais: a Esfera Coletiva, que concentra recursos críticos
sujeitos à aprovação coletiva; a Esfera do Usuário, que abrange
recursos gerenciados individualmente com base em controles
tradicionais; e a Esfera Imutável, responsável por armazenar logs e
registros de maneira inalterável, assegurando transparência e
rastreabilidade \cite{Colbac}.

No contexto do \gls{colbac}, interações com a Esfera Coletiva seguem um
processo estruturado em três fases: na Fase de Rascunho, o usuário
cria um token que especifica as permissões e objetivos da ação; na
Fase de Petição, o token é submetido à votação pelos membros da
organização; e, finalmente, na Fase de Autorização, os resultados da
votação determinam a aprovação ou rejeição da ação, com todos os
registros sendo armazenados na Esfera Imutável \cite{Colbac}. Essa
estrutura oferece flexibilidade ao permitir a adaptação do nível de
horizontalidade de acordo com as necessidades da organização,
inclusive em situações de crise que possam demandar centralizações
temporárias \cite{Colbac}.

Apesar de suas vantagens, o \gls{colbac} enfrenta desafios inerentes à sua
abordagem democrática \cite{Colbac}. Processos de votação frequentes podem resultar
em fadiga dos usuários, especialmente em organizações maiores
\cite{Colbac, EverydayRevolutions}.
Além disso, ataques democráticos, como a manipulação de quóruns ou o uso
abusivo de tokens de emergência, representam riscos significativos.
Tais problemas podem ser mitigados com a implementação de auditorias
independentes, ajustes dinâmicos nos critérios de quórum e mecanismos
que limitem o uso de tokens de emergência \cite{Colbac}. Outra questão importante é
a curva de aprendizado associada ao modelo, que requer familiaridade
com práticas democráticas e a compreensão do funcionamento dos tokens
\cite{Colbac}.

O \gls{colbac} oferece uma solução inovadora para organizações que desejam
alinhar sua governança horizontal com práticas robustas de segurança
digital \cite{Colbac}. Sua transparência, flexibilidade e compromisso com a
participação democrática o posicionam como uma ferramenta estratégica
para superar os desafios de segurança em estruturas descentralizadas,
transformando potenciais vulnerabilidades em oportunidades para
fortalecer a autonomia coletiva \cite{Colbac, EverydayRevolutions}.

\subsection{ABCcrypto}
\label{subsec:abccrypto}

O \gls{abc} é um framework de modelagem de
ameaças desenvolvido especificamente para abordar as peculiaridades de
criptomoedas e sistemas baseados em blockchain. Em contraste com
frameworks generalistas como o \gls{stride}, o \gls{abc} foi projetado para lidar
com os desafios de segurança únicos apresentados por sistemas
distribuídos e permissionless, onde atores desconfiam uns dos outros e
os incentivos econômicos desempenham um papel central
\cite{AbcCrypto}.

O principal diferencial do \gls{abc} é a introdução de matrizes de conluio
(collusion matrices), que permitem analisar cenários de ameaças que
envolvem colaborações entre diferentes atores maliciosos. Essa
abordagem sistemática reduz a complexidade do processo de modelagem ao
eliminar casos irrelevantes e agrupar cenários com efeitos
semelhantes \cite{AbcCrypto}. Além disso, o framework utiliza categorias de ameaças
específicas para criptomoedas, considerando não apenas os ativos
tangíveis, como blockchains e tokens, mas também ativos abstratos,
como privacidade e reputação \cite{AbcCrypto}.

Uma característica fundamental do \gls{abc} é sua capacidade de adaptar as
categorias de ameaças aos objetivos e ativos de cada sistema \cite{AbcCrypto}. O
processo começa com a caracterização detalhada do modelo de sistema,
identificando participantes, ativos e motivações financeiras. Em
seguida, categorias de ameaças são derivadas a partir das violações
potenciais das propriedades de segurança dos ativos, como corrupção de
serviços, roubo de pagamentos e inconsistências na blockchain. Por
fim, cenários concretos de ataque são enumerados e analisados por meio
da matriz de conluio, que considera todas as combinações possíveis de
atacantes e alvos \cite{AbcCrypto}.

O \gls{abc} também destaca a importância de incorporar análises econômicas e
incentivos financeiros no processo de mitigação de riscos. Por
exemplo, mecanismos de "detectar e punir" podem ser implementados para
desencorajar comportamentos desonestos ao torná-los financeiramente
inviáveis. Este uso de teoria dos jogos e modelagem econômica é
particularmente eficaz para endereçar ameaças que não podem ser
neutralizadas exclusivamente por meios criptográficos
\cite{AbcCrypto}.

A eficácia do \gls{abc} foi demonstrada em estudos de caso envolvendo
sistemas reais, como Bitcoin, Filecoin e CacheCash. No caso do
Filecoin, o framework revelou lacunas significativas no design
público, particularmente em cenários de conluio que não haviam sido
previamente considerados. Já no CacheCash, o \gls{abc} foi usado desde as
etapas iniciais do design para identificar 525 casos de conluio e
implementar contramedidas baseadas em incentivos \cite{AbcCrypto}.

Embora apresente benefícios claros, o \gls{abc} não é isento de desafios \cite{AbcCrypto}.
A criação de matrizes de conluio e a análise detalhada de categorias de
ameaças podem ser intensivas em termos de recursos, especialmente em
sistemas com múltiplos participantes e ativos complexos. No entanto,
esses esforços são recompensados pela identificação de ameaças
críticas e pela robustez das soluções propostas \cite{AbcCrypto}.

O \gls{abc} oferece uma abordagem avançada e adaptável para a modelagem de
ameaças em criptomoedas, demonstrando que frameworks especializados
podem melhorar significativamente a segurança e a resiliência de
sistemas distribuídos \cite{AbcCrypto}.


\subsection{PGP e o Web of Trust}
\label{subsec:pgp_web_of_trust}

\gls{pgp}, desenvolvido por Philip Zimmermann, é um sistema de
criptografia que combina privacidade, autenticação e conveniência para proteger
mensagens e arquivos \cite{Pgp}. O \gls{pgp} utiliza a criptografia de chave pública para
viabilizar a comunicação segura entre indivíduos, mesmo sem confiança prévia ou
troca de chaves em canais seguros \cite{Pgp}.

No modelo de funcionamento do \gls{pgp}, cada indivíduo possui um par de chaves — uma
pública, amplamente divulgada, e uma privada, mantida em sigilo. A chave pública
é utilizada para criptografar mensagens, enquanto a privada é usada para
descriptografá-las. Esse esquema não apenas garante a privacidade das mensagens,
mas também permite a autenticação por meio de assinaturas digitais, assegurando
a integridade e a origem de um conteúdo \cite{Pgp}.

Uma característica central do \gls{pgp} é o modelo de \gls{wot}, que adota
uma abordagem descentralizada para a validação de identidades. Diferentemente
das hierarquias centralizadas de Autoridades Certificadoras, o \gls{wot} permite
que qualquer usuário assine digitalmente a chave pública de outro, certificando
sua autenticidade. Essas assinaturas criam uma rede de confiança distribuída, na
qual a validação de uma chave pública depende da confiança acumulada das
assinaturas de outros usuários confiáveis \cite{Pgp}.

No \gls{pgp}, cada usuário pode atribuir diferentes níveis de confiança a outros
indivíduos para atuarem como "introduzidores confiáveis". Esse mecanismo permite
que a rede de confiança seja construída de forma orgânica, refletindo as
relações sociais naturais. Por exemplo, um usuário pode confiar plenamente em
outro para certificar chaves, ou apenas marginalmente, dependendo de sua
percepção sobre a competência e integridade do introduzidor \cite{Pgp}.

Além de promover a descentralização, a \gls{wot} também oferece resiliência
contra ataques \cite{Pgp}. Em vez de depender de um único ponto de falha, como ocorre em
sistemas centralizados, o \gls{wot} permite que os usuários validem chaves públicas
com base em múltiplas assinaturas, reduzindo o impacto de compromissos
individuais. No entanto, essa abordagem também apresenta desafios, como a
dificuldade de gerenciar grandes anéis de chaves e a subjetividade na atribuição
de níveis de confiança \cite{Pgp}.

A relevância do \gls{pgp} e da \gls{wot} para organizações horizontais é evidente \cite{Colbac}.
Estruturas não hierárquicas podem aproveitar a natureza descentralizada do \gls{wot}
para criar sistemas de segurança alinhados aos princípios de autonomia coletiva
e governança distribuída . Ao permitir que cada participante construa sua própria
rede de confiança, o \gls{pgp} reforça a segurança sem comprometer a horizontalidade
dessas organizações \cite{EverydayRevolutions, Colbac}.

\section{Perspectivas Comparativas}
\label{sec:comparative_perspectives}

\subsection{Critérios de Avaliação}
\label{subsec:criteria_evaluation}

A avaliação de frameworks de modelagem de ameaças requer critérios
objetivos para comparar sua eficácia, aplicabilidade e adequação a
diferentes contextos organizacionais \cite{EvaluationofCompetingThreatModeling}.
Critérios fundamentais incluem a capacidade de
identificar ameaças específicas, adaptabilidade às
mudanças no ambiente operacional, custos de implementação e a
integração de dimensões sociais, econômicas e técnicas no processo de
modelagem \cite{ThreatModelingASystematicLiteratureReview}.
Além disso, a escalabilidade e a habilidade de lidar com
estruturas organizacionais complexas são fatores críticos \cite{AbcCrypto}.

O \gls{stride}, por exemplo, é amplamente utilizado por sua simplicidade e
aplicabilidade em sistemas de software tradicionais
\cite{ThreatModelingdesigningForSecurity}. No entanto, ele
enfrenta limitações em ambientes descentralizados, como criptomoedas
ou organizações horizontais, devido à sua dependência de categorias de
ameaças predefinidas \cite{STRIDEthreatmodelingforcyberphysical}.
Em contraste, frameworks como o \gls{abc} oferecem uma abordagem especializada,
utilizando matrizes de conluio para explorar ameaças em
sistemas distribuídos e permissionless, enquanto o \gls{pasta} foca na
avaliação iterativa de riscos para sistemas dinâmicos
\cite{AbcCrypto, RiskCentricThreatModeling}.

Frameworks como o \gls{colbac}, por outro lado, destacam-se por integrar
processos democráticos à modelagem de ameaças, permitindo que
organizações horizontais alinhem segurança e governança participativa.
Apesar de sua inovação, o \gls{colbac} enfrenta desafios de usabilidade
devido à necessidade de familiarização com processos democráticos e o
risco de fadiga de votação em grandes grupos.

Por fim, a escalabilidade é essencial para organizações que lidam com
múltiplos participantes e interações complexas \cite{AbcCrypto}.
Técnicas como a fusão de cenários no \gls{abc} demonstram que frameworks especializados
podem mitigar custos operacionais enquanto mantêm a robustez
analítica \cite{AbcCrypto}. Isso os torna adequados para sistemas modernos que exigem
tanto precisão técnica quanto flexibilidade organizacional \cite{Colbac}.

\subsection{Aplicabilidade em Organizações Não-Hierárquicas}
\label{subsec:applicability_non_hierarchical}

A aplicabilidade de frameworks de modelagem de ameaças em organizações
não-hierárquicas depende de sua capacidade de abordar dinâmicas
específicas dessas estruturas \cite{Colbac, ThreatModelingASystematicLiteratureReview}.
Organizações horizontais operam com base em governança distribuída,
participação equitativa e ausência de centralização formal,
exigindo abordagens que respeitem e fortaleçam esses princípios \cite{EverydayRevolutions}.

Frameworks como o \gls{colbac} e o \gls{abc} demonstram forte
compatibilidade com organizações horizontais devido à sua ênfase em
processos colaborativos e adaptabilidade a contextos descentralizados.
O  \gls{colbac} utiliza tokens de autorização coletiva para alinhar segurança
e governança democrática, permitindo flexibilidade entre centralização
temporária e controle horizontal. Já o \gls{abc} incorpora análise
econômica e incentivos financeiros para mitigar riscos em ecossistemas
de blockchain, oferecendo ferramentas como matrizes de conluio para
mapear cenários de ameaça complexos.

Frameworks tradicionais, como \gls{stride} e árvores de ataque, fornecem uma
base sólida para a identificação de ameaças, mas sua aplicabilidade é
limitada em organizações horizontais devido à sua dependência de
hierarquias formais e pontos de controle centralizados
\cite{ThreatModelingdesigningForSecurity, AttackTrees}.
Em contraste, abordagens emergentes, como o \gls{ptm},
promovem maior alinhamento com as necessidades dessas organizações,
integrando stakeholders em todas as etapas do processo.

A escolha de um framework para organizações horizontais deve
equilibrar eficácia técnica com transparência e engajamento coletivo.
Soluções como o \gls{abc} e o  \gls{colbac} destacam-se ao integrar dimensões
sociais e econômicas, demonstrando que a segurança pode ser
fortalecida por meio de práticas colaborativas e governança inclusiva
\cite{AbcCrypto, Colbac}.
Essas abordagens são particularmente relevantes para organizações que
buscam alinhar segurança com autonomia e resiliência organizacional \cite{Colbac}.
