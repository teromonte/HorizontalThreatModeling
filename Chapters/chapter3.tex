%!TEX root = ../template.tex
%%%%%%%%%%%%%%%%%%%%%%%%%%%%%%%%%%%%%%%%%%%%%%%%%%%%%%%%%%%%%%%%%%%%
%% chapter3.tex
%% NOVA thesis document file
%%%%%%%%%%%%%%%%%%%%%%%%%%%%%%%%%%%%%%%%%%%%%%%%%%%%%%%%%%%%%%%%%%%%

\typeout{NT FILE chapter3.tex}%

\chapter{Related Work}
\label{cha:related_work}

The references discussed in this chapter were chosen based on an initial
selection suggested by the advisor of this work, Professor Kevin Gallagher. As a
starting point, the professor provided three fundamental articles related to his
research area: \gls{colbac}, \gls{abc} and \gls{pgp}'s \gls{wot}. In particular, the work
on \gls{colbac} is directly linked to Professor Gallagher's academic production,
offering a solid and highly relevant basis for the development of this research.
Based on these initial references, traditional and emerging methods for threat
modeling were explored, aiming to establish a comprehensive overview that would
allow a critical analysis of the applicability of these methods to the specific
context of non-hierarchical organizations.

\glsresetall

\section{Traditional Approaches to Threat Modeling}
\label{sec:traditional_approaches_threat_modeling}

\subsection{STRIDE}
\label{subsec:stride}

\gls{stride}, developed by Microsoft, is a systematic threat modeling
methodology designed to identify potential vulnerabilities in software systems
\cite{MicrosoftThreatModelingTechnique}. The acronym \gls{stride} stands for six
main categories of threats \cite{ThreatModelingdesigningForSecurity}. Each of
these categories reflects a specific violation of desired security properties
such as authenticity, integrity, non-repudiation, confidentiality, availability,
and authorization \cite{MicrosoftThreatModelingTechnique}.

Applying \gls{stride} begins with the creation of \gls{dfds} to map the movement
of information within the system \cite{UncoverSecurityDesignFlawsSTRIDE}.
\gls{dfds} help identify elements such as external entities, processes, data
flows, and data stores. For each element in the diagram, the six \gls{stride}
threat categories are analyzed to identify potential vulnerabilities
\cite{ThreatModelingdesigningForSecurity}.

Each threat in \gls{stride} has a clear definition and practical examples to aid
in identification and mitigation. For example:

\begin{enumerate}
    \item \textbf{Spoofing:} Threats that involve falsification of the
identity of users or processes, compromising authenticity.
    \item \textbf{Tampering:} Manipulation of data in transit, in
storage, or in memory, affecting integrity.
    \item \textbf{Repudiation:} Scenarios where users deny actions
performed, often due to the lack of adequate logging mechanisms.
    \item \textbf{Information Disclosure:} Exposure of
sensitive information to unauthorized parties, compromising
confidentiality.
    \item \textbf{Denial of Service:} Attacks that overload
system resources, impairing availability.
    \item \textbf{Elevation of Privilege:} Cases where a malicious
actor obtains privileges above those authorized, compromising authorization.
\end{enumerate}

The \gls{stride} methodology can be adapted to different contexts. For example,
in cyber-physical systems, it is possible to assess threats related to hardware
and software components, such as failures in synchronization or communication
\cite{STRIDEthreatmodelingforcyberphysical}. In addition, variants such as
STRIDE-per-Element and STRIDE-per-Interaction offer more focused approaches to
identify threats in specific elements or in interactions between components
\cite{ThreatModelingdesigningForSecurity}.

Although widely used, \gls{stride} has limitations. It relies heavily on analyst
experience and may not capture emerging threats in decentralized systems or
dynamic environments \cite{SecurityDevelopmentLifecycle}. Therefore, it is often
complemented with other frameworks, such as \gls{dread}, to prioritize threats
based on impact and probability \cite{DREADful}.

In summary, \gls{stride} provides a solid foundation for threat identification,
but its effective application requires integration with other methodologies and
adaptations to meet the specific needs of modern, decentralized systems
\cite{UncoverSecurityDesignFlawsSTRIDE}.

\subsubsection{STRIDE Complementary Models}
\label{subsubsec:stride_complementary_models}

Several complementary models have been derived from or used in conjunction with
\gls{stride} to improve its effectiveness and adaptability in different contexts
\cite{SoftwareandattackcentricThreatModeling}. Notable among these is the
\gls{dread} model, which complements \gls{stride} by providing a quantitative
approach to threat prioritization \cite{DREADful}. \gls{dread} uses five main
categories: Damage Potential, Reproducibility, Exploitability, Affected Users,
and Discoverability, allowing analysts to assign values and create scores to
rank threats according to their severity
\cite{SoftwareandattackcentricThreatModeling, DREADful}.

The combined use of \gls{stride} and \gls{dread} can improve risk assessment in
more complex systems. However, the inherent subjectivity in assigning values
in \gls{dread} can compromise the consistency of \cite{DREADful} analyses. To
mitigate these limitations, some organizations have integrated \gls{stride} with
more comprehensive frameworks, such as \gls{pasta}, which takes an iterative
approach to identifying and prioritizing threats
\cite{SoftwareandattackcentricThreatModeling}.

Additionally, the use of attack trees has proven effective in complementing
\gls{stride}, allowing teams to visually represent complex threat scenarios and
identify multiple attack paths \cite{FoundationsofAttackTrees}. This integration
is particularly useful in horizontal organizations, where the lack of
centralization increases the need for collaborative threat mapping
\cite{ThreatModelingdesigningForSecurity}.

\subsection{Attack Trees}
\label{subsec:attack_trees}

Attack trees, introduced by Bruce Schneier \cite{AttackTrees}, provide a
hierarchical framework for threat modeling, where the attack objective is
represented by the root node, and the sub objectives and steps required to
achieve it are arranged in child nodes. Each node can be detailed with logical
operators such as AND and OR, representing conditions that must be met either
together or alternatively \cite{FoundationsofAttackTrees}.

A key advantage of attack trees is their ability to decompose complex threats
into smaller, more manageable components, enabling systematic analysis
\cite{Energytheftdetectionissues}. This methodology makes it easier to identify
multiple attack paths, allowing organizations to prioritize countermeasures
based on metrics such as cost, impact, and likelihood
\cite{AnAttackTreeBasedRisk}.

Practical applications of attack trees include their use in wireless sensor
networks to assess location privacy risks \cite{AnAttackTreeBasedRisk}, as well
as in energy theft detection in advanced metering infrastructures such as smart
grids \cite{Energytheftdetectionissues}. In both cases, the approach enables
organizations to map specific threat scenarios and design effective
countermeasures.

In addition, studies such as \cite{FoundationsofAttackTrees} highlight the reuse
of subtrees to increase efficiency in complex systems. This practice allows
shared elements across different scenarios to be modeled once and incorporated
into future analyses, saving time and resources.

Although broadly applicable, attack trees present challenges related to the
effort required for their initial construction and the complexity in large scale
systems \cite{AttackTrees, Energytheftdetectionissues}. Collaboration between
stakeholders, including technical and operational experts, is essential to
ensure that threat representation is accurate and comprehensive
\cite{Energytheftdetectionissues}.

Attack trees also stand out as complementary tools to methodologies such as
\gls{stride} and can be used both to identify threats and to organize those
already discovered \cite{FoundationsofAttackTrees,
ThreatModelingASystematicLiteratureReview}. Furthermore, reusing existing trees,
such as those focused on fraud or elections, saves time and provides a solid
foundation for analysis \cite{FoundationsofAttackTrees}. Despite their
versatility, effective use of trees depends on clear representations of AND/OR
nodes and continuous evaluation to avoid oversights or gaps
\cite{ThreatModelingdesigningForSecurity}.

\section{Emerging Methodologies}
\label{sec:emerging_methodologies}

\subsection{PASTA}
\label{subsec:pasta}

\gls{pasta} is a risk centric threat modeling methodology designed to integrate
security throughout the software development lifecycle. Proposed by Tony
UcedaVelez and Marco M. Morana \cite{RiskCentricThreatModeling}, the framework
consists of seven sequential stages that allow for in depth and iterative
analysis of threats and vulnerabilities.

The main goal of \gls{pasta} is to align security concerns with business
objectives, ensuring that mitigation measures address both technical risks and
organizational impacts \cite{RiskCentricThreatModeling}. The methodology promotes
a risk oriented approach by integrating attack simulations to evaluate the
effectiveness of proposed countermeasures \cite{RiskCentricThreatModeling}.

\begin{enumerate}
    \item \textbf{Definition of the Objectives (DO):} In this initial
stage, the security requirements, risk profile, and potential
business impacts are defined.
    \item \textbf{Definition of the Technical Scope (DTS):} This stage
details the technical aspects, such as users, software components,
third party infrastructure, and external dependencies.
    \item \textbf{Application Decomposition and Analysis (ADA):} The
application is broken down into basic functional elements to identify
data flows, user types, and existing security controls.
    \item \textbf{Threat Analysis (TA):} Identification of potential
threats based on the analyzed elements and assets, considering the most
likely attack vectors.
    \item \textbf{Weakness and Vulnerability Analysis (WVA):} At this
stage, threats are associated with specific vulnerabilities,
evaluating the effectiveness of existing controls and identifying
weaknesses.
    \item \textbf{Attack Modeling and Simulation (AMS):} Performing
simulations to determine the most likely attack paths,
using attack trees and other models to explore risk
scenarios.
    \item \textbf{Risk Analysis and Management (RAM):} Identifying
technical and business impacts, proposing measures to mitigate
priority risks \cite{RiskCentricThreatModeling}.
\end{enumerate}

\gls{pasta} stands out for its flexibility and analytical depth, making it
particularly effective in dynamic and distributed environments
\cite{ThreatModelingASystematicLiteratureReview}. The methodology encourages
collaboration between stakeholders from different areas, promoting a unified
understanding of risks and organizational priorities
\cite{ParticipatoryThreatModelling}. In addition, the integration of attack
simulations allows organizations to test the effectiveness of their security
strategies under realistic conditions, improving their resilience against
emerging threats \cite{RiskCentricThreatModeling}.

One of the most relevant aspects of \gls{pasta} is its compatibility with
horizontal organizations. The collaborative approach of the methodology is
aligned with the principles of distributed governance \cite{Colbac}. In non-hierarchical
structures, \gls{pasta} offers a structured framework to identify and mitigate
risks in a participatory and efficient way.

\subsection{Security Cards}
\label{subsec:security_cards}

Security Cards are a tool designed to facilitate brainstorming of security
threats, using a deck of cards that address different aspects of potential
attacks \cite{SecurityCardsToolkit}. Created by Tamara Denning, Batya Friedman,
and Tadayoshi Kohno, the cards cover four main dimensions: adversary
motivations, adversary capabilities, adversary methods, and human impact
\cite{KeepingAheadofOurAdversaries}. This approach aims to foster creativity and
collaboration among stakeholders, encouraging a more holistic and comprehensive
analysis of threats \cite{CyberThreatModeling}.

Each card provides examples and scenarios related to its dimension, helping
teams explore vulnerabilities that might otherwise be missed by traditional
methods \cite{SecurityCardsToolkit}. For example, in the “Human Impact”
dimension, cards can highlight how security breaches can affect privacy,
emotional or financial well being, providing a more user centric perspective
\cite{KeepingAheadofOurAdversaries}.

Security Cards have been used in a variety of applications, such as protecting
biometric systems against presentation attacks
\cite{AttackTreesforProtectingBiometric, KeepingAheadofOurAdversaries}. Their
flexible structure allows them to be adapted to different organizational
contexts, including decentralized environments
\cite{ParticipatoryThreatModelling}. In horizontal organizations, Security Cards
facilitate the participation of multiple stakeholders, promoting distributed
governance and reinforcing collaboration \cite{CyberThreatModeling}.

Despite their potential, the methodology can generate a high number of false
positives, which requires additional effort to filter relevant threats
\cite{KeepingAheadofOurAdversaries}. However, their emphasis on creativity and
inclusion of multiple perspectives makes Security Cards a valuable tool for
exploring emerging threats and strengthening organizational resilience
\cite{CyberThreatModeling}.

\subsection{Personae Non Grata}
\label{subsec:personae_non_grata}

\gls{pngs} represent an innovative approach to threat modeling, notable for
their focus on malicious users and undesirable behaviors
\cite{PersonaeNonGratae}. Inspired by traditional user experience design
personas, \gls{pngs} help anticipate how adversaries might exploit
vulnerabilities in a system, providing a detailed adversarial perspective
\cite{PnGRequirementsPhaseThreatModeling}.

\gls{pngs} are created through techniques such as crowd sourcing, allowing
different stakeholders to contribute insights to threat identification
\cite{PnGRequirementsPhaseThreatModeling}. This collaborative approach increases
the breadth and diversity of attacker profiles considered, allowing for more
robust modeling that is adapted to different contexts \cite{PersonaeNonGratae}.

One of the main advantages of \gls{pngs} is their ability to capture specific
attacker motivations, capabilities, and behaviors
\cite{PnGRequirementsPhaseThreatModeling}. For example, a PnG might describe an
adversary who uses phishing to obtain credentials or exploits security flaws in
financial transactions \cite{PersonaeNonGratae}. This level of detail helps in
prioritizing countermeasures and allocating security resources
\cite{PnGRequirementsPhaseThreatModeling}.

In addition, \gls{pngs} are particularly effective in contexts where internal
and external threats overlap \cite{PersonaeNonGratae}. In flat organizations,
where governance is distributed and responsibility is shared, \gls{pngs} help
map potential risks that may arise from internal actors, such as employees, or
external actors, such as competitors \cite{PersonaeNonGratae}.

Despite its benefits, implementing \gls{pngs} requires significant effort to
ensure that profiles are accurate and relevant
\cite{PnGRequirementsPhaseThreatModeling}. However, when integrated with other
methodologies such as attack trees or \gls{stride}, \gls{pngs} provide an
additional layer of analysis, making them an indispensable tool for
organizations seeking to comprehensively understand and mitigate threats
\cite{PnGRequirementsPhaseThreatModeling}.

\section{Hybrid and Collaborative Approaches}
\label{sec:hybrid_collaborative_approaches}

Hybrid and collaborative approaches seek to integrate different methodologies to
create adaptable and effective frameworks in different contexts
\cite{AHybridThreatModelingMethod, CoReTM}. Among these, \gls{htmm} stands out,
combining elements of different frameworks for a comprehensive risk analysis. It
is particularly useful in scenarios involving multiple stakeholders and
requiring alignment between security objectives and business priorities
\cite{AHybridThreatModelingMethod}.

Another example is \gls{coretm}, designed to facilitate threat modeling in
distributed or remote teams. Using collaborative tools, such as shared
annotation platforms, \gls{coretm} makes the process more accessible and
inclusive \cite{CoReTM}.

Finally, \gls{ptm} promotes the inclusion of a wide range of stakeholders in the
threat modeling process \cite{ParticipatoryThreatModelling}. This approach
values the diversity of perspectives, being especially relevant in
decentralized contexts where transparency and collective participation are
essential \cite{ParticipatoryThreatModelling}. These methodologies reinforce
distributed governance and strengthen organizational resilience, complementing
the work developed by more traditional frameworks \cite{Colbac}.

\section{Decentralized Trust and Cryptographic Frameworks}
\label{sec:decentralized_trust_crypto_frameworks}

\subsection{COLBAC}
\label{subsec:colbac}

\gls{colbac} is an access control model designed to address the specificities of
horizontal organizations, promoting a democratic and participatory approach to
access authorization. Its proposal seeks to overcome the challenges imposed by
traditional access control models, such as \gls{dac}, \gls{mac} and \gls{rbac},
which often reinforce hierarchical dynamics that are inadequate for
horizontalized structures \cite{Colbac}.

One of the most striking features of \gls{colbac} is its ability to align access
control with horizontal governance practices, allowing decisions to be made
collectively through democratic processes \cite{ParticipatoryThreatModelling}.
The model organizes resources and processes into three main spheres: the
Collective Sphere, which concentrates critical resources subject to collective
approval; the User Sphere, which encompasses individually managed resources
based on traditional controls; and the Immutable Sphere, responsible for storing
logs and records in an unalterable manner, ensuring transparency and
traceability \cite{Colbac}.

In the context of \gls{colbac}, interactions with the Collective Sphere follow a
process structured in three phases: in the Draft Phase, the user creates a token
that specifies the permissions and objectives of the action; in the Petition
Phase, the token is submitted to a vote by the members of the organization;
in the Authorization Phase, the results of the vote determine the
approval or rejection of the action, with all records being stored in the
Immutable Sphere \cite{Colbac}. This structure offers flexibility by allowing
the adaptation of the level of horizontality according to the needs of the
organization, including in crisis situations that may require temporary
centralizations \cite{Colbac}.

Despite its advantages, \gls{colbac} faces challenges inherent to its democratic
approach \cite{Colbac}. Frequent voting processes can result in user fatigue,
especially in larger organizations \cite{Colbac, EverydayRevolutions}. In
addition, democratic attacks, such as quorum manipulation or the abuse of
emergency tokens, pose significant risks. Such issues can be mitigated by
implementing independent audits, dynamic adjustments to quorum criteria, and
mechanisms that limit the use of emergency tokens \cite{Colbac}. Another
important issue is the learning curve associated with the model, which requires
familiarity with democratic practices and an understanding of how tokens work
\cite{Colbac}.

\gls{colbac} offers an innovative solution for organizations that want to align
their horizontal governance with robust digital security practices
\cite{Colbac}. Its transparency, flexibility, and commitment to democratic
participation position it as a strategic tool to overcome security challenges in
decentralized structures, transforming potential vulnerabilities into
opportunities to strengthen collective autonomy \cite{Colbac,
EverydayRevolutions}.

\subsection{ABCcrypto}
\label{subsec:abccrypto}

\gls{abc} is a threat modeling framework specifically developed to address the
peculiarities of cryptocurrencies and blockchain based systems. In contrast to
generalist frameworks such as \gls{stride}, \gls{abc} is designed to address the
unique security challenges presented by distributed and permissionless systems,
where actors distrust each other and economic incentives play a central role
\cite{AbcCrypto}.

The main differentiator of \gls{abc} is the introduction of collusion matrices,
which allow the analysis of threat scenarios involving collaborations between
different malicious actors. This systematic approach reduces the complexity of
the modeling process by eliminating irrelevant cases and grouping scenarios with
similar effects \cite{AbcCrypto}. Furthermore, the framework uses threat
categories specific to cryptocurrencies, considering not only tangible assets,
such as blockchains and tokens, but also abstract assets, such as privacy and
reputation \cite{AbcCrypto}.

A key feature of \gls{abc} is its ability to tailor threat categories to the
objectives and assets of each \cite{AbcCrypto} system. The process begins with a
detailed characterization of the system model, identifying participants, assets,
and financial motivations. Threat categories are then derived based on potential
violations of the security properties of the assets, such as service corruption,
payment theft, and blockchain inconsistencies. Finally, concrete attack
scenarios are enumerated and analyzed using the collusion matrix, which
considers all possible combinations of attackers and \cite{AbcCrypto} targets.

\gls{abc} also highlights the importance of incorporating economic analysis and
financial incentives into the risk mitigation process. For example, “detect and
punish” mechanisms can be implemented to discourage dishonest behavior by making
it financially unviable. This use of game theory and economic modeling is
particularly effective for addressing threats that cannot be neutralized by
cryptographic means alone \cite{AbcCrypto}.

The effectiveness of \gls{abc} has been demonstrated in case studies involving
real systems such as Bitcoin, Filecoin, and CacheCash. In the case of Filecoin,
the framework revealed significant gaps in the public design, particularly in
collusion scenarios that had not previously been considered. In CacheCash,
\gls{abc} was used from the early design stages to identify 525 cases of
collusion and implement incentive based countermeasures \cite{AbcCrypto}.

While \gls{abc} has clear benefits, it is not without its challenges
\cite{AbcCrypto}. Creating collusion matrices and analyzing threat categories in
detail can be resource intensive, especially in systems with multiple
participants and complex assets. However, these efforts are rewarded by the
identification of critical threats and the robustness of the proposed solutions
\cite{AbcCrypto}.

\gls{abc} offers an advanced and adaptable approach to cryptocurrency threat
modeling, demonstrating that specialized frameworks can significantly improve
the security and resilience of distributed systems \cite{AbcCrypto}.

\subsection{PGP and the Web of Trust}
\label{subsec:pgp_web_of_trust}

\gls{pgp}, developed by Philip Zimmermann, is a cryptographic system that
combines privacy, authentication, and convenience to protect messages and files
\cite{Pgp}. \gls{pgp} uses public key cryptography to enable secure
communication between individuals, even without prior trust or key exchange over
secure channels \cite{Pgp}.

In the \gls{pgp} operating model, each individual has a pair of keys — a public
key, which is widely disseminated, and a private key, which is kept secret. The
public key is used to encrypt messages, while the private key is used to decrypt
them. This scheme not only ensures the privacy of messages, but also allows
authentication through digital signatures, ensuring the integrity and origin of
a content \cite{Pgp}.

A central feature of \gls{pgp} is the \gls{wot} model, which adopts a
decentralized approach to identity validation. Unlike centralized hierarchies of
Certificate Authorities, \gls{wot} allows any user to digitally sign another's
public key, certifying its authenticity. These signatures create a distributed
trust network, in which the validation of a public key depends on the
accumulated trust of the signatures of other trusted users \cite{Pgp}.

In \gls{pgp}, each user can assign different levels of trust to other
individuals to act as "trusted introducers". This mechanism allows the trust
network to be built organically, reflecting natural social relationships. For
example, a user may fully trust another to certify keys, or only marginally,
depending on their perception of the introducer's competence and integrity
\cite{Pgp}.

In addition to promoting decentralization, \gls{wot} also provides resilience
against \cite{Pgp} attacks. Rather than relying on a single point of failure, as
is the case in centralized systems, \gls{wot} allows users to validate public
keys based on multiple signatures, reducing the impact of individual
compromises. However, this approach also presents challenges, such as the
difficulty of managing large key rings and the subjectivity in assigning trust
levels \cite{Pgp}.

The relevance of \gls{pgp} and \gls{wot} to horizontal organizations is evident.
Non-hierarchical structures can take advantage of the
decentralized nature of \gls{wot} to create security systems aligned with the
principles of collective autonomy and distributed governance . By allowing each
participant to build their own web of trust, \gls{pgp} strengthens security
without compromising the horizontality of these organizations
\cite{EverydayRevolutions, Colbac}.

\section{Comparative Perspectives}
\label{sec:comparative_perspectives}

\subsection{Evaluation Criteria}
\label{subsec:evaluation_criteria}

The evaluation of threat modeling frameworks requires objective criteria to
compare their effectiveness, applicability, and suitability to different
organizational contexts \cite{EvaluationofCompetingThreatModeling}. Key criteria
include the ability to identify specific threats, adaptability to changes in the
operational environment, implementation costs and the integration of social,
economic, and technical dimensions into the modeling process
\cite{ThreatModelingASystematicLiteratureReview}. In addition, scalability and
the ability to handle complex organizational structures are critical factors
\cite{AbcCrypto}.

\gls{stride}, for example, is widely used for its simplicity and applicability
in traditional software systems \cite{ThreatModelingdesigningForSecurity}.
However, it faces limitations in decentralized environments due to its reliance on
predefined threat categories \cite{STRIDEthreatmodelingforcyberphysical}. In contrast,
frameworks such as \gls{abc} offer a specialized approach, using collusion
matrices, while \gls{pasta} focuses on iterative risk assessment for dynamic systems
\cite{AbcCrypto, RiskCentricThreatModeling}.

\subsection{Applicability in Non-Hierarchical Organizations}
\label{subsec:applicability_nonhierarchical_orgs}

The applicability of threat modeling frameworks to non-hierarchical
organizations depends on their ability to address specific dynamics of these
structures \cite{Colbac, ThreatModelingASystematicLiteratureReview}. Horizontal
organizations operate on the basis of distributed governance, equitable
participation, and the absence of formal centralization, requiring approaches
that respect and reinforce these principles \cite{EverydayRevolutions}.

Frameworks such as \gls{colbac} and \gls{pasta} demonstrate strong compatibility
with horizontal organizations due to their emphasis on collaborative processes
and adaptability to decentralized contexts. \gls{colbac} uses collective
authorization tokens to align security and democratic governance. \gls{pasta},
with its iterative and participatory approach, facilitates the involvement of
stakeholders at all levels, ensuring that diverse perspectives are considered
in the threat modeling process.

The choice of a framework for horizontal organizations must balance technical
effectiveness with transparency and collective engagement. Solutions such as
\gls{abc} and \gls{colbac} excel at integrating social and economic dimensions,
demonstrating that security can be strengthened through collaborative practices
and inclusive governance \cite{AbcCrypto, Colbac}.

\begin{table}[]
    \caption{Comparison of Threat Modeling Methods}
    \label{tab:threat-modeling-comparison}
    \scriptsize
    \resizebox{\textwidth}{!}{
        \begin{tabular}{|p{0.06\textwidth}|p{0.2\textwidth}|p{0.21\textwidth}|p{0.25\textwidth}|p{0.25\textwidth}|}
            \toprule
            Method
                &Threat categories considered
                &Scope
                &Key Features
                &Limitations\\
            \midrule
            STRIDE
                &Spoofing, Tampering, Repudiation, Information Disclosure, Denial of
                    Service, Elevation of Privilege
                &Software and application security (design phase threat analysis)
                &Six category mnemonic covers major attack types (acts as a
                    checklist). Applied to \gls{dfds} to systematically identify threats
                    per component. Simple and widely adopted; easy for teams new to threat modeling.
                &Focused on technical threats, may miss threats outside its six
                    categories (e.g., collusion or social attacks). No built in risk ranking, so a
                    long list of threats may need additional prioritization. Considered less
                    detailed for complex, business specific threats.\\
            Attack Trees
                &No fixed set, any threat goal can be modeled (e.g., malware
                    attack, insider abuse, social engineering) as the root, with sub goals as attack
                    steps.
                &Broadly applicable (software, cyber-physical systems, critical
                    infrastructure) for visualizing attack paths.
                &Graphical tree diagrams map out attacker goals and all possible
                    paths (leaf nodes are specific attack methods). Can capture complex multi step
                    attacks (including technical exploits or human tactics) by branching logic.
                    Often used in combination with other methods (e.g., STRIDE or risk scoring) for
                    comprehensive analysis.
                &Time consuming to create for large or complex systems (trees can
                    become very large). No inherent risk scoring, requires supplemental analysis
                    (like CVSS) to prioritize threats. Maintenance can be difficult as systems
                    evolve (manual updates needed for tree changes).\\
            PASTA
                &Not predefined by categories, addresses all threat types
                    identified through its stages, from technical vulnerabilities to misuse and
                    fraud (prioritized by business impact).
                &Enterprise and critical systems where business impact and risk
                    alignment are key (e.g., fintech, large applications). Integrates with SDLC and
                    organizational risk management.
                &Seven stage, risk centric methodology (from defining scope to
                    attack simulation) providing in depth analysis. Aligns threats with business
                    objectives, focuses on likely attacks that matter most to the organization's
                    mission. Emphasizes risk prioritization: analyzes likelihood/impact of each
                    threat to guide resource allocation.
                &Complex and resource intensive, the 7 step process is lengthy and
                    requires cross functional expertise. Not widely used compared to simpler models;
                    has a steep learning curve and longer implementation time. Geared toward
                    comprehensive analysis, which may be overkill for small projects or early phase
                    designs.\\
            Trike
                &Doesn't use classic threat categories, focuses on unauthorized
                    actions on assets. Threats are modeled as any action (Create, Read, Update,
                    Delete) by any actor that violates assigned permissions (e.g., an insider
                    modifying data they shouldn't, or an outsider gaining illicit access).
                &Primarily for security audit and risk management in software
                    systems. Suitable for organizations wanting to set and verify acceptable risk
                    levels for each asset/user role.
                &Risk based audit approach, assigns a risk score to each asset's
                    threat scenarios, ensuring risk is acceptable to stakeholders. Uses a unique
                    actor-asset matrix (CRUD matrix) to identify where an actor's permitted actions
                    differ from potential malicious actions. Provides a structured way to involve
                    stakeholders in deciding which risks to mitigate vs. accept (tying into
                    governance).          
                &Can be too granular/complex in large IT environments, requires
                    detailed mapping of all assets, roles, and permissions. Less mainstream and fewer
                    tooling options and community support compared to STRIDE/PASTA. Focuses on
                    internal policy violations; may require augmentation to cover threats like
                    external social engineering or collusion not captured by role permission
                    analysis.\\
            ABC
                &Collusion based threats (multiple actors cooperating maliciously)
                    and financial attacks specific to crypto systems. Derives custom threat
                    categories for new blockchain assets (e.g., double spend fraud, consensus
                    manipulation) beyond traditional threat lists.
                &Blockchain and decentralized systems (cryptocurrencies,
                    decentralized organizations). Tailored for systems with economic incentives and
                    distributed trust, but also applied to other large scale distributed systems
                    (e.g., cloud native architectures).
                &Introduces collusion matrices, a novel tool to systematically
                    enumerate complex collusion scenarios among actors (forcing analysis of
                    combinations of insider/outsider attacks). Asset centric: identifies unique
                    assets (crypto tokens, consensus, smart contracts) and derives system specific
                    threat categories incorporating financial impact.      
                &Specialized scope, designed for cryptocurrency and blockchain context, so it
                    may require adaptation to use in other domains. Complexity: Collusion analysis
                    can become intricate (though the matrix approach manages complexity, it still
                    demands detailed domain knowledge).\\
            \bottomrule
        \end{tabular}%
    }
\end{table}

\section*{} 
Chapter 3 analyzed traditional threat modeling approaches, highlighting
established methodologies such as \gls{stride}, as well as their applications and
limitations in decentralized and collaborative environments. The fundamental
categories used in vulnerability identification were described and the necessary
adaptations for non-hierarchical contexts were discussed. Based on these
analyses, Chapter 4 presents the preliminary design of a protocol specifically
adapted for horizontal organizations, describing the fundamental security and
governance requirements necessary to maintain organizational resilience and
ensure democratic participation in threat analysis and mitigation processes.